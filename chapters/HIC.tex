\section{Relativistic heavy-ion collision}
Heavy ion collisions are major experimental tools to study nuclear matter at high energy density.
In this chapter, I shall introduce the basics of collider physics to be used extensively in the rest of this thesis. 
This includes concepts and terminologies from both hadron and nuclear collisions. 
Then, I will discuss on the status experimental observations and the theoretical understandings and challenges.

High energy colliders are important tools for high energy physics study. 
Hadron such as protons and anti-protons and various nuclei are accelerated to ultra-relativistic energies and collide, depositing a huge amount of energy in the center-of-mass (CoM) frame for particle production.

It is advantages think of the relavistic collision in a new set of coordinates that is related to the Cartesian coordinates by,
\begin{eqnarray}
x_\perp &=& x_\perp\\
\tau &=& \sqrt{t^2 - z^2}\\
\eta_s &=& \frac{1}{2}\ln\frac{t+z}{t-z}
\end{eqnarray}
where the $z$ direction aligns with the accelerated beam direction.
$\tau$ is called the ``proper time" and $\eta_s$ called the space-time rapidity.
One advantage of using this set of coordinates is that $\tau$ and $\eta_s$ transforms much simpler than $t$ and $z$ under Lorentz boost ($\beta_z$) in the beam direction,
\begin{eqnarray}
\tau' &=& \tau,\\
\eta_s' &=& \eta_s + \frac{1}{2}\ln\frac{1+\beta_z}{1-\beta_z}
\end{eqnarray}
Similarly, for momentum the transverse mass is defined as $m_T^2 = m^2 + p_T^2 = E^2 - p_z^2$ and the rapidity is $y = \frac{1}{2}\ln\frac{E+p_z}{E-p_z}$.
Besides, pseudo-rapidity is often used experimentally,
\begin{equation}
    \eta = \frac{1}{2}\ln\frac{|p|+p_z}{|p|-p_z} = \frac{1}{2}\ln\frac{1+\cos\theta}{1-\cos\theta}
\end{equation}
It has the merits that it is directly related to the polar angle of final state particles and that it is an approximate for rapidity when the transverse mass is negligible.

In the center-of-mass frame of the colliding nuclei, due to the Lorentz contraction in the beam direction, the nuclei ``shrinks" in the $z$ direction by the factor $\gamma = (1-v^2)^{-1/2} = E/M$. 
For example, at top RHIC energy the energy per nucleon in the center-of-mass frame is 100 GeV, and the rest mass of nucleon is about 1 GeV, yielding a contraction factor of 100. 
At top energy for Pb+Pb collision at the LHC, this factor can be as large as 2500.
Therefore, the high energy nuclei ``look like pancakes" with transverse spatial extend about the nucleus radii, but a highly concentrated profile in the beam direction $\delta z \sim 2R/\gamma$.
As the result the penetration of the colliding nuclei happened almost instantaneously compared transverse dynamics.
Energy is deposited in the overlapped area and entropy is rapidly produced, creating an expanding fireball in the middle while the nuclear remnants recede.
This highly excited fireball of fields undergoes complex dynamics and cools down.
Eventually, it reaches a point that the strong fields is again confined into hadrons. The hadrons may decay into other hadrons, photons and lepton, or free-stream into the detectors.

Nuclei are extended objects, and collision geometry can fluctuate from complete overlap (small impact-parameter) to peripheral interaction (large impact-parameter).
It is impossible to determine precisely the impact-parameter by analyzing the final state. 
However, an approximate handle on the the collision geometry is the ``centrality" meter. 
Centrality can be defined in different ways at different kinematic cuts, but the idea is to reflect the nuclear collision geometry by the amount of particle production activity.
It is not hard to anticipate the number of charged particles produced or the total transverse energy observed by the detector increases as the impact parameter decreases, up-to fluctuations.
These approximate correspondence between centrality measure and impact-parameter gives a powerful handle on selecting events with particular nuclear collision geometry.


It is our goal to understand this state of highly excited nuclear matter and probe its properties.
Experimentally, it is found that produced charged particles has a wide distribution in rapidity / pseudo-rapidity with a plateau around the middle rapidity $\eta = 0$.
The formation of such a plateau is also observed in proton proton collision and can be explained by the way QCD produces particles.
The transverse momentum $p_T$ spectra of produced particles are very steep, meaning that the majority of particles produced in one event are relatively soft (involving small $p_T$) with $p_T \lesssim 3$ GeV.
These particles forms the ``soft" observables. 
Very occasionally, particles with large transverse momentum ($p_T\gtrsim 10$ GeV) are produced, and there are referred to as the ``hard" particles.
By uncertainty principal, these hard particles can only be produced in the initial collisions on a time scale $\delta t \sim 1/p_T$.
Therefore, a hard process can only be produced at the very beginning of the nuclear collision and interacts with the nuclear medium that surrounds it.
Due to the large $p_T$ and relatively small coupling constant, the production of these hard particles can be studied in a perturbative framework.
The interaction with the medium modifies the initial production and leaves finger prints of the nuclear matter properties in the hard observables. 
So, these hard particles can be used as self-generated probes of the system.

In the following sections, I shall introduce the key observations from soft physics and the current ``standard model" for describing the bulk medium evolution.
This focus on particularly the anisotropic flows and the hydrodynamic based modeling.
After that I devote a section to discuss one of the remaining challenge of the medium evolution: the initial stage, and explain my contribution to this problem.
The rest of this chapter will be focusing on the hard probes, including the production in hadronic collision, the jet quenching phenomenon in nuclear collisions, and finally introduce the main focus of this thesis: heavy-flavors as hard probes.

\subsection{Anisotropic flows and QGP transport coefficient}
One of the most striking discovery from RHIC and LHC heavy-ion program is the large anisotropic flows in the final charged particle distribution function.
Decomposing the invariant spectra of particle production into Fourier series of the azimuthal angle,
\begin{eqnarray}
E\frac{dN}{p_T dp_T d\phi dy} = \frac{1}{2\pi}\frac{dN}{p_T dp_T dy}\left(1 + \sum_{n=1}^{\infty}v_n(p_T)\cos\left[n(\phi-\Psi_n)\right]\right).
\end{eqnarray}
The first term in the expansion is azimuthal angle averaged yield.
The terms in the sum expands the angular dependence. 
The $n=1$ term is a modulation of the center of mass, which we neglect by virtue of symmetry at mid-rapidity.
From $n=2$, a non-zero $v_n$ characterize the anisotropy in momentum-space of various order.
If the particle production in high energy nuclear collisions is simply an independent sum of elementary nucleon binary collisions, then the anisotropy would be zero after averaging over many events. 
However, experiments observe a surprisingly larger elliptic flow ($v_2$) in Au+Au collisions and later in Pb+Pb collisions.
Higher order flows harmonics and in particular odd order $v_n$ were discovered later.
The large anisotropic flow suggests a substantial role of final state interaction among excitations after the initial production.
The ideal hydrodynamic model that assumes immediate realization of local thermal equilibrium, i.e., strong and frequent final state interaction, explains elliptic flow very well.
In the hydrodynamic picture, the deformed fireball created in non-central nuclear collision and expands hydrodynamically. 
The initial spatial anisotropy results in a pressure gradient differences in the long- and short-axis directions.
The higher pressure gradient in the short axis direction drives a fast expansion than the perpendicular direction, explaining the quadrupole azimuthal modulation.
The existence of odd order harmonics seemed odd but was explained by the simple idea that the nuclear wave function fluctuates on such a short time scale. 
The randomized nucleon positions results in non-zero odd-order initial eccentricities that drives the respective anisotropic flows.
Improving ideal hydrodynamics, relativistic viscous hydrodynamics was developed to include off-equilibrium effects, which includes important QCD transport properties as the shear and bulk viscosity.
The QGP shear viscosity and bulk viscosity are of fundamental importance. 
The shear viscosity to entropy ratio is related to the stong / weak coupling nature of the relevant dynamics. 
The bulk viscosity is shown to be directly replaced to the QCD trace anomoly and scale-invariance breaking.
Study has shown that anisotropic flows are sensitive probes of the QCD shear viscosity.
Stronger constraining power for both shear and bulk viscosities can be achieved by global fitting to a variety of bulk observables.
It is found that small but non-vanishing values of shear- and bulk-viscosity-to-entropy ratio is needed to quantitatively describe the data quantitatively.

\subsection{``Standard model" of bulk evolution}


\subsection{Hard probes}

\subsection{Jet quenching}

\subsection{Physics of heavy-flavors}