\documentclass[PhD]{dukethesis2006}
%\UseRawInputEncoding
%preamble here for options
\usepackage[left=1.5in, top=1in, right=1in, bottom=1in]{geometry}
\usepackage[numbers,compress]{natbib}
\setcitestyle{square}
\usepackage[T1]{fontenc}
\usepackage{lmodern}
\usepackage{bbold}
\usepackage{wrapfig}
%\usepackage[light]{roboto}

\usepackage{mathtools}
\usepackage{appendix}
\usepackage{amsmath}
\usepackage{amssymb}
\usepackage{amsfonts}
\usepackage{tabularx}
\usepackage{booktabs}
\usepackage{color}
\usepackage{xcolor}
\usepackage{multirow}
\usepackage{verbatim}
\usepackage[inline]{enumitem}
\usepackage{slashed}
\usepackage{enumerate}
\usepackage{mfirstuc}
\usepackage{array}
\usepackage{epsfig}
\usepackage{graphicx}
\graphicspath{{images/}{images/LBT/}{images/Feynman/}{images/Lido/}{images/New-analysis/}{images/Citing/}}
\usepackage{xy}

%----  macros to save typing; added by hartley, 2002 ----
\newcommand{\fig}{\begin{figure}[htbp]\centering}
\newcommand{\pic}[1]{\includegraphics[width=#1]}
\newcommand{\efig}{\end{figure}} 
%\newcommand{\efig}{\end{figure} \clearpage}
% Use \clearpage to cure ``too many floats'' problem (1 fig. per page).
\newcommand{\fr}[1]{Figure \ref{fig:#1}}
\newcommand{\FR}[1]{Figure \ref{fig:#1}}
\newcommand{\er}[1]{equation \ref{eq:#1}}
\newcommand{\ER}[1]{Equation \ref{eq:#1}}
\newcommand{\eq}{\begin{equation}}
\newcommand{\eeq}{\end{equation}}
\newcommand{\eqa}{\begin{eqnarray}}
\newcommand{\eeqa}{\end{eqnarray}}
\newcommand{\etal}{\nobreak\mbox{\it et al.}}



\newcommand{\singlespacing}{%
  \let\CS=\small\renewcommand{\baselinestretch}{1.0}\CS}
\newcommand{\doublespacing}{%
  \let\CS=\small\renewcommand{\baselinestretch}{1.6}\CS}
\newcommand{\normalspacing}{\doublespacing}

%\newcommand{\singlespacingplus}{%
%  \let\CS=\@currsize\renewcommand{\baselinestretch}{1.25}\tiny\CS}
%\newcommand{\realdoublespacing}{%
%  \let\CS=\@currsize\renewcommand{\baselinestretch}{2}\tiny\CS}
%\newcommand{\footnotespacing}{\singlespacing}
%\newcommand{\changespacing}[2]{%
%  \renewcommand{#1}{%
%    \let\CS=\@currsize\renewcommand{\baselinestretch}{#2}\tiny\CS}%
%}
%\newcommand{\changenormalspacing}[1]{\renewcommand{\normalspacing}{#1}}




%---- units ----
\newcommand{\days}{\nobreak\mbox{$\;$days}}
\newcommand{\hrs}{\nobreak\mbox{$\;$hrs}}
\newcommand{\mins}{\nobreak\mbox{$\;$min}}
\newcommand{\s}{\nobreak\mbox{$\;$s}}
\newcommand{\ms}{\nobreak\mbox{$\;$ms}}
\newcommand{\us}{\nobreak\mbox{$\;\mu$s}}
\newcommand{\inch}{\nobreak\mbox{$\;$in}}
\newcommand{\meter}{\nobreak\mbox{$\;$m}}
\newcommand{\cm}{\nobreak\mbox{$\;$cm}}
\newcommand{\mm}{\nobreak\mbox{$\;$mm}}
\newcommand{\um}{\nobreak\mbox{$\;\mu$m}}
\newcommand{\nm}{\nobreak\mbox{$\;$nm}}
\newcommand{\cmpers}{\nobreak\mbox{$\;$cm\,s$^{-1}$}}
\newcommand{\mmpers}{\nobreak\mbox{$\;$mm\,s$^{-1}$}}
\newcommand{\umpers}{\nobreak\mbox{$\;\mu$m\,s$^{-1}$}}
\newcommand{\g}{\nobreak\mbox{$\;$g}}
\newcommand{\kg}{\nobreak\mbox{$\;$kg}}
\newcommand{\hz}{\nobreak\mbox{$\;$Hz}}
\newcommand{\mhz}{\nobreak\mbox{$\;$mHz}}
\newcommand{\uhz}{\nobreak\mbox{$\;\mu$Hz}}


%---- oft-used complicated symbols ----
\newcommand{\rms}{\nobreak\mbox{$\sigma_{rms}$}}
\newcommand{\sat}{\nobreak\mbox{$\sigma_{\textrm{sat}}$}}
\newcommand{\DG}{\nobreak\mbox{$\Delta G^2$}}
\newcommand{\DS}{\nobreak\mbox{$\Delta\sigma$}}
\newcommand{\DQ}{\nobreak\mbox{$\Delta\theta$}}
\newcommand{\DSDQ}{\nobreak\mbox{$\Delta\sigma/\Delta\theta$}}
\newcommand{\dq}{\nobreak\mbox{$d\theta$}}
\newcommand{\ds}{\nobreak\mbox{$d\sigma$}}
\newcommand{\dsdq}{\nobreak\mbox{$d\sigma/d\theta$}}
\newcommand{\YY}{$\curlyvee\curlywedge$}
\newcommand{\GG}{\nobreak\mbox{$G^2$}}

\newcommand{\fmc}{\ensuremath{\text{fm}/c}}
\newcommand{\sqrts}{\sqrt{s_\mathrm{NN}}}
\newcommand{\sigmann}{\sigma^\text{inel}_\mathrm{NN}}
\newcommand{\nch}{N_\text{ch}}
\newcommand{\ntrk}{N_\text{trk}^\text{offline}}
\newcommand{\Np}{N_\text{part}}
\newcommand{\Nc}{N_\text{coll}}
\newcommand{\np}{n_\text{part}}
\newcommand{\nc}{n_\text{coll}}
\newcommand{\avg}[1]{\langle #1 \rangle}
\newcommand{\vnk}[2]{v_{#1}\{#2\}}
\newcommand{\order}[1]{\ensuremath{\mathcal{O}(#1)}}
\newcommand{\del}{\nabla}
\newcommand{\mn}{^{\mu\nu}}
\newcommand{\trento}{T\raisebox{-0.5ex}{R}ENTo}
\newcommand{\T}{\tilde{T}}
\newcommand{\dmin}{d_\text{min}}
\newcommand{\tfs}{\tau_\text{fs}}
\newcommand{\Tsw}{T_\text{switch}}
\newcommand{\dv}{\mathbf d}
\newcommand{\kv}{\mathbf k}
\newcommand{\vv}{\mathbf v}
\newcommand{\xv}{\mathbf x}
\newcommand{\yv}{\mathbf y}
\newcommand{\zv}{\mathbf z}
\newcommand{\tran}{^\mathrm{T}}
\newcommand{\note}{[{\color{deepred} ?}]}

\newcommand{\paddedhline}{\noalign{\smallskip}\hline\noalign{\smallskip}}
\newcommand{\dnchdy}{dN_\text{ch}/d\eta}
\newcommand{\dndypP}{dN_\text{pPb}/d\eta}
\newcommand{\dndyPP}{dN_\text{PbPb}/d\eta}
\newcommand{\x}{\mathbf x}
\newcommand{\y}{\mathbf y}
\newcommand{\z}{\mathbf z}
\newcommand{\trans}{^\intercal}

\newcommand{\Raa}{R_{AA}}
\newcommand{\vvv}{v_3\{2\}}
\newcommand{\vnv}{v_n\{2\}}
\newcommand{\ppi}{\frac{\partial}{\partial p_i}}
\newcommand{\ppj}{\frac{\partial}{\partial p_j}}
\newcommand{\ppl}{\frac{\partial}{\partial p_l}}
\newcommand{\Kpara}{\kappa_{\|}}
\newcommand{\Kperp}{\kappa_{\perp}}
\newcommand{\Ppara}{\hat{P}^{\|}}
\newcommand{\Pperp}{\hat{P}^{\perp}}

\DeclareFontFamily{OMS}{oasy}{\skewchar\font48 }
\DeclareFontShape{OMS}{oasy}{m}{n}{%
         <-5.5> oasy5     <5.5-6.5> oasy6
      <6.5-7.5> oasy7     <7.5-8.5> oasy8
      <8.5-9.5> oasy9     <9.5->  oasy10
      }{}
\DeclareFontShape{OMS}{oasy}{b}{n}{%
       <-6> oabsy5
      <6-8> oabsy7
      <8->  oabsy10
      }{}
\DeclareSymbolFont{oasy}{OMS}{oasy}{m}{n}
\SetSymbolFont{oasy}{bold}{OMS}{oasy}{b}{n}

\DeclareMathSymbol{\smallleftarrow}     {\mathrel}{oasy}{"20}
\DeclareMathSymbol{\smallrightarrow}    {\mathrel}{oasy}{"21}
\DeclareMathSymbol{\smallleftrightarrow}{\mathrel}{oasy}{"24}
\newcommand{\tensor}[1]{\overset{\scriptscriptstyle\smallleftrightarrow}{#1}}


\DeclareMathOperator{\expi}{Ei}
\DeclareMathOperator{\acosh}{acosh}
\DeclareMathOperator{\diag}{diag}
\DeclareMathOperator{\cov}{cov}
\DeclareMathOperator{\SC}{SC} 
%\usepackage{sectsty}
%\allsectionsfont{\singlespacing\raggedright\Large}
% This is where the shortened versions of Latex environments like figure, equation, etc 
% are defined. See the file for the shortcuts.


\author{Weiyao Ke}
\advisor{Steffen Bass}
\member{Berndt M\"uller}
\member{Christopher Walter}
\member{Robert Wolpert}
\member{Thomas Sch\"afer}

\department{Department of Physics}
%\subject{xxx} If this is used, "subject" has to be un-commented in the cls file in several places
\title{Partonic Transport Model Application to Heavy Flavor}

%end of preamble, beginning of printable document

\begin{document}


\maketitle

\makeabstract

\Copyright

\abstract
\doublespacing
Heavy-flavor particles are excellent probes of the properties of the hot and dense nuclear medium created in the relativistic heavy-ion collisions.
Heavy-flavor transport coefficients in the quark-gluon plasma (QGP) stage of the collisions are particularly interesting, as they contain important information on the strong interaction at finite temperatures.
Studying the heavy-flavor evolution in a dynamically evolving medium requires a comprehensive multi-stage modeling approach of both the medium and the probes, with an accurate implementation of the physical ingredients to be tested.
For this purpose, I have developed a new partonic transport model (Linear-Boltzmann-plus-Diffusion-Transport-Model) LIDO and applied it to heavy quark propagation inside a QGP.
The model has an improved implementation of parton in-medium bremsstrahlung and a flexible treatment of the probe-medium interactions, combining both large angle scatterings and diffusion processes.
The model is then coupled to a high-energy event-generator, a hydrodynamic medium evolution and a hadronic transport model.
Finally, applying a Bayesian analysis, I extract the heavy quark transport coefficients from a model-to-data comparison.
The results, with uncertainty quantification, are found to be consistent with earlier extraction of the light-quark transport coefficients at high momentum and with first-principle calculations of the heavy flavor diffusion constant at low momentum.

\acknowledgements
This work was supported by the U.S. Department of Energy (DOE) grant number DE-FG02-05ER41367 and by National Science Foundation (NSC) grant number OAC-1550225.
The computational resources were provided by the Open Science Grid (OSG) funded by DOE and NSF, and by the National Energy Research Scientific Computing Center (NERSC), which is the primary scientific computing facility for the Office of Science in the DOE.


%A table of contents file is automatically generated in the same folder as the .tex file when
%the \tableofcontents is used
\singlespacing
\tableofcontents
\listoffigures
\listoftables

%-----------------------------------------------------------------------------%
% replace FILE in \input{FILE} with name of tex file
% containing the given chapter, eg. for the introduction one could
% have FILE = intro if stored in intro.tex (.tex extension is assumed!).
%-----------------------------------------------------------------------------%

\doublespacing
\chapter{Introduction}
\section{Nuclear matter under extreme conditions}
The quest for the matter and its properties under extreme highly density and temperature has fundamental importance.
Understanding nuclear matter and under these conditions help to understand questions from the early dynamics of the universe to the neutron star merger simulation for the gravitational wave studies.
It also help to enrich the understanding of the fundamental theory of the strong interaction and basics to the nuclear physics: the Quantum Chromodynamics (QCD).

\paragraph{Quantum Chromodynamics}
QCD describes the interaction of objects that carries ``color'' charges.
Quarks (fermions) and gluons (bosons) are its elementary degrees of freedom. 
The QCD Lagrangian (with one flavor of quark) is,
\begin{eqnarray}
\mathcal{L} = \bar{\psi_i} \left(i\gamma_\mu D^\mu_{ij} -m \delta_{ij} \right)\psi_j - \frac{1}{4}G_{\mu\nu}^a G^{\mu\nu,a},
\end{eqnarray}
where $\psi_i$ the Dirac spinor of the quark field with $i$ the color index.
\begin{eqnarray}
D_{ij}^\mu = \partial^\mu - i g T_{ij}^a A^{\mu, a}
\end{eqnarray}
is the covariant derivative, containing the interaction between quark field and the gluon field with coupling strength $g$.
Here $T_{ij}^a$ are the generators of the SU(3) group in the fundamental representation and the generators satisfies the commutation relation,
\begin{eqnarray}
[T^a, T^b] = i f^{abc} T^c
\end{eqnarray}
where $f^{abc}$ are known as the structure constants of SU(3).
The field tensor of gluon with color $a$ is,
\begin{eqnarray}
G^{\mu\nu,a} = \partial^\mu A^{\nu, a} - \partial^\nu A^{\mu, a} + g f^{abc} A^{\mu,b}A^{\mu,c}
\end{eqnarray}
The first term is the kinetic term, and the second term is the gluon field self-interation (also with strength $g$) is a unique feature of the non-Abelian gauge field.

\paragraph{Asymptotic freedom and confinement}
Due to quantum fluctuations, the effective coupling constant $g$ changes with the energy scale of a process. 
The rate of change of $g$ with respect to the scale parameter is called the $\beta$-function,
\begin{eqnarray}
\frac{\partial g}{\partial \ln\mu} = \beta(g),
\end{eqnarray}
which can be evaluated as a perturbation series of $g$.
At leading order, the QCD $\beta$ function with number of colors $N_c$ and $n_f$ flavors of fermion is,
\begin{eqnarray}
\beta(g) = - \left( \frac{11}{3}N_c - \frac{2}{3}n_f \right) \frac{g^3}{16\pi^2}.
\end{eqnarray}
This $\beta$ function is negative for QCD ($N_c=3$) using realistic numbers of quark flavors $n_f = 2\cdots 6$, meaning the effective coupling constant decreases with increasing energy scale.
The property is known as the asymptotic freedom of QCD because the interaction becomes small at asymptotically high energy, which also makes possible the use of perturbation theory at high energy.

Often a strong coupling constant is defined as $\alpha_s = g^2/4\pi$.
Using the leading order $\beta$-function, its scale dependence is
\begin{eqnarray}
    \alpha_s(Q^2) = \frac{4\pi}{\left(\frac{11}{3}N_c - \frac{2}{3}n_f\right)\ln\left(\frac{Q^2}{\Lambda^2}\right)}.
\end{eqnarray}
The integration constant has been absorbed into the QCD scale parameter $\Lambda$.
Therefore, at least in perturbation theory, $\Lambda$ becomes the only parameter of QCD. 
Its value is determined by anchoring $\alpha_s(\mu)$ to the experimental measurement at a fixed scale, for example, $\alpha_s(M_z) = 0.1185$ at the scale equal to the $Z$ boson mass.
The leading order $\Lambda$ is then around $200$ MeV.

The decreasing of $\alpha_s(Q)$ is logarithmic slow at high energy, but it rises quickly approaching with $Q$ approaching $\Lambda$ from above.
Even before reaching this scale, the coupling constant is already too large for a reliable perturbative calculation.
Near the $\Lambda$ scale, QCD enters the non-perturbative region,
and at this long distances only hadrons exists as colorless bound states of quarks and gluons.
The fact that colors not directly observed at large distances is knowns as ``color confinement'' of QCD. 
To pull a quark out of the hadron, the color field becomes so strong that eventually more quark-anti-quark pairs populated in-between the pulled quark and the remnant and forms new colorless hadrons.

Depending on its valance quarks (quarks that carry the net quantum number of the hadron) content, hadrons are generally categorized as baryons and mesons.
Baryon has three valance quarks or anti-quarks, such as neutrons and protons.
Meson has a valance quark and an anti-quark, such as pion and kaon.
Hadrons are also populated with sea-quarks and gluons that are constantly produced and annihilated as quantum fluctuations.
The momentum of hadron is mostly carried by the valance quarks.
The sea quarks and gluons together share the rest fraction of the total momentum, but their abundance number at high energy is very important for the particle production in relativistic hadron / heavy-ion collisions.

Nowadays, the only reliable ab initio theoretical tool for non-perturbative QCD is lattice field theory technique, where the QCD Lagrangian is discretized on a finite lattice and studied on a computer.

\paragraph{The phase-diagram of the QCD matter}
At zero temperature ($T$), protons and neutrons (together as nucleons) form bound states of atomi nuclei that build the ordinary matter.
One can also define the baryon chemical potential $\mu_b$, and for ordinary matter $\mu_b$ is around $1$ GeV, close to the proton mass.
The ordinary nuclei is indicated as the white dot on the (partly conjectured) phase diagram in figure \ref{fig:phase-diagram} [].
Increasing the temperature of the system, nucleons start escape from nuclear potential and other hadrons can be created from collision and resonance formation and decay.
This system is know as the hadron gas (cyan region in figure \ref{fig:phase-diagram}).

Because QCD has asymptotic freedom at high energy and confinement at low energy scale, there can be so-called deconfinement phase-transition when temperatures crosses the QCD non-perturbative scale.
At asymptotically high temperature, the weakening of the coupling should lead to the transit from the color confined hadronic matter to a system of transporting quarks and gluons, termed the quark-gluon plasma (QGP). 
Frist principle lattice QCD calculations have studied this transition at zero baryon chemical potential with 2+1 flavors (up / down plus strange quark) QCD.
Figure \ref{fig:qcd_eos} quotes the equation of state computed by  the HotQCD Collaboration [].
It shows the pressure $P$, energy density ($\epsilon$) and entropy density ($s$) of QCD system.
These thermodynamic quantities are scaled by powers of temperature, so that the ratio can be loosely related the effective number of degrees of freedom (DoF) of the system
The Stefan-Boltzmann limit (non-interacting gas of quarks and gluons) is denoted as the dashed lines in the up-right corner.
It is observed that the effective number of DoF converges to the expectation from a hadron resonance gas model (solid lines) at low temperature and rapidly increases to a value close to the Stefan-Boltzmann limit in a narrow temperature window.
This suggests a releasing of the quark and gluon degrees of freedom at high temperature.
More close study indicate that this is not a real phase-transition at ($\mu_b = 0$), and sometimes people also refer to it as a ``cross-over'' phase transition, where the thermaldynamical quantity is smooth across this region of phase-diagram.
Nevertheless, a pseudo critical temperature is defined to be $T_c \approx 150 $ MeV, corresponding to 1.5 trillion Kelvin.

\begin{figure}
    \centering
    \includegraphics[width=.8\textwidth]{phase-diagram.png}
    \caption{Caption}
    \label{fig:phase-diagram}
\end{figure}

Moving to wards finite baryon chemical potential, the lattice approach runs into the fermion sign problem, though recent studies has been pushing the realm of lattice QCD into small $\mu/T$ regions [].
Effective models [] and Dyson-Schwinger equation studies [] have suggested the existence of a first order phase transition at large $\mu_B/T$.
If true, the first-order coexistence line must end at a point on the phase-diagram at lower $\mu_b$, beyond which the phase-transition is the cross-over type.
Such a point, called the critical end point (CEP), has attached a great interest of both theoretical confirmation / exclusion and experimental search.

At higher chemical potential and low temperature, another phase of nuclear matter known as the ``color superconducting phase'' is proposed, where the quarks forms Cooper pairs in analogy to the superconductors [].

\begin{figure}
    \centering
    \includegraphics[width=.8\textwidth]{qcd-eos.png}
    \caption{Caption}
    \label{fig:qcd_eos}
\end{figure}

It is believed that the QCD high-temperature phase-transition is a stage of the universe around a microsecond after ``the Big Bang'', when the temperature drops down to QCD scale.
Compact starts are ``celestial laboratories'' to test the QCD equation-of-state in the high density and low temperature region, providing an important physical input for simulating the recently discovered gravitational wave from neutron star mergers.
In laboratories, we create hot and dense nuclear matter by colliding heavy nuclei at ultra-relativistic high energies.
Though the created matter is so transient and tiny compared to the cosmic nuclear matter, we can learn not only thermodynamic properties but also essential dynamical properties of the QCD in these experiments.

\section{Phenomenology of relativistic heavy-ion collision}
Relativistic heavy-ion collision is currently the only tool to access high energy density QCD medium in laboratory.
Since 2005, the Relativistic Heavy-ion Collider (RHIC) at the Brookheaven National Laboratory (BNL) started colliding gold nuclei at 200 GeV []. 
The Large Hadron Collider (LHC) started its heavy ion programs later, colliding lead nuclei at 2.76 TeV and 5.02 TeV [].
Since then, many evidences have been pointing to the existence a new state-of-matter: the strongly coupled quark-gluon plasma (sQGP).

In this section, I shall introduce useful concepts and terminology in heavy-ion collisions.
Then I will review a few important experimental evidences and how they can help the understanding of the properties of the sQGP.

\subsection{Kinematics}
In ultra relativistic collisions, it is advantages to use a new set of coordinates, related to the Cartesian coordinates by,
\begin{eqnarray}
x_\perp &=& x_\perp\\
\tau &=& \sqrt{t^2 - z^2}\\
\eta_s &=& \frac{1}{2}\ln\frac{t+z}{t-z}
\end{eqnarray}
where the $z$ direction aligns with the accelerated beam direction.
$\tau$ is called the ``proper time" and $\eta_s$ called the space-time rapidity.
One advantage of using this set of coordinates is that $\tau$ and $\eta_s$ transforms much simpler than $t$ and $z$ under Lorentz boost ($\beta_z$) in the beam direction,
\begin{eqnarray}
\tau' &=& \tau,\\
\eta_s' &=& \eta_s + \frac{1}{2}\ln\frac{1+\beta_z}{1-\beta_z}
\end{eqnarray}
Similarly for the four momentum $p^\mu$ is parametrized as 
\begin{eqnarray}
p_x &=& p_T\cos\phi\\
p_y &=& p_T\sin\phi\\
m_T &=& \sqrt{m^2 + p_T^2}\\
y &=& \frac{1}{2}\ln\frac{E+p_z}{E-p_z}.
\end{eqnarray}
$p_T$ is transverse momentum relative to the beam ($z$) direction, $\phi$ is the azimuth angle of particle emission. 
$m_T$ is referred as the transverse mass, and $y$ is the rapidity of a particle.
Besides, pseudorapidity is often used in experiments,
\begin{equation}
\eta = \frac{1}{2}\ln\frac{|p|+p_z}{|p|-p_z} = \frac{1}{2}\ln\frac{1+\cos\theta}{1-\cos\theta}
\end{equation}
It has the merits that it is directly related to the polar angle  $\theta$ of particle emission.
When the transverse mass is small compared to $p_z$, the pseudorapidity is also a good proxy of rapidity.

\subsection{Nuclear collision geometry}
Nuclei are extended objects.
The radius of heavy nuclei approximately scales like $A^{1/3}$ fm, where  $A$ is the atomic number; therefore, the collision geometry plays far more important role than it is in the proton-proton collision.
In the center-of-mass frame,  nuclei ``shrink" in the $z$ direction by the factor $\gamma = (1-v^2)^{-1/2} = E/M$ because of the Lorentz contraction.
$\gamma$ is over $100$ for gold nuclei at top RHIC energy and is more than $2500$ for lead nuclei at the LHC.
As a result, the approaching nuclei takes a short time to penetrating each other $t_L = 2R/\gamma$, while dynamics in the transverse direction can only propagate within a causal circle of $r < t_L$ that is much smaller than the nuclear extend.

\paragraph{Impact-parameter and centrality} Defining the impact parameter $\vec{b}$ as the transverse separation between the center-of-mass of the two approach nuclei,
the initial deposition of the energy largely depends on $\vec{b}$.
The collision geometry is a useful handle to study QGP dynamics; however,
it is impossible control $b$ directly in high energy experiments.
What is used as an approximate geometry indicator is the so-called centrality.
Centrality are defined in different ways (detector response / multiplicity / transverse energy) and with different kinematic cuts, but the idea that  nuclear collision geometry strongly correlates with the particle production activity.
It is reasonable to anticipate that the average number of charged particles produced or the total transverse energy deposited within a certain detector's acceptance is higher if the collision is more central (small impact parameter), and is lower for peripheral collision (large impact parameter).
Of course the relation between centrality and impact-parameter is not exact, as dynamical fluctuations smear out the one-to-one correspondence between the impact parameter and the ``centrality meter''.
Correctly accounting for these fluctuations is particularly important for small collision system,s such as proton-lead and deutron-gold collisions.

\paragraph{Centrality selection} Experimentally, a minimum-biased (a minimum set of event triggers) sample of recorded events are sorted according to the centrality definition (multiplicity, e.g.) and the events are binned by percentile.
Then, for example, the top 0--5\% highest multiplicity events are associated to the central collisions with centrality range 0--5\%. 
And the details of the collision geometry can be studied through a model.
Usually, the models is one of the many variants of the Glauber model [], we shall explained it in detail in section \ref{simulation}).
It computes the number of binary nucleon-nucleon ($N_{\textrm{bin}}$) collisions and number of participant nucleons ($N_{\textrm{part}}$, nucleons that suffers at least one binary collisions) at a given impact parameter.
Experimentally, $N_{\textrm{part}}$ is often used as the centrality estimator of the model as it is roughly proportional to the bulk particle production; while the cross-section of hard processes that involves large momentum transfer $Q \gg \Lambda$ scales like the $N_{\textrm{bin}}$.
Though this correspondence can be model dependent, but the uncertainty can be quantified and its prediction can be validated by the production of colorless probes.

\subsection{Particle production at low-$p_T$ and collective flow} 
Immediately after the nuclei penetration through each other $t\sim 2R/\gamma$, huge amount of energy is deposited into the overlapped area and entropy is produced, creating a fireball in the middle while the nuclear remnants recede.
This highly excited fireball of fields undergoes complex dynamics and cools down rapidly because of the longitudinal and transverse expansion.
Eventually, the system hardronizes, and the hadrons can have further interactions and may decay into other hadrons, photons and lepton that are seen by the detectors.

It is observed that the particle production in relativistic nuclear collisions distributes across a wide (pseudo)rapidity range, and has steep falling transverse momentum spectra [].
The majority of the particles are soft hadrons with relatively small transverse momentum $p_T \lesssim 3$ GeV and their creation are consequences of final state interactions [].
One of the most striking discoveries from RHIC and LHC heavy-ion program is that these soft particles displays a strong collectivity and the patterns can be described by hydrodynamic-based models to a very high precision [].
This success of the hydrodynamic model reveals the strongly coupled nature of the matter produced with a temperature several times above $T_c$ and it has been entitled the name strongly coupled quark-gluon plasma.
This is in contrary to a weakly coupled gas of quarks and gluons that emits independently.

\begin{figure}
\centering
\includegraphics[width=.6\textwidth]{ALICE-chg-vn.png}
\caption{•}
\label{fig:intro:vn}
\end{figure}

One manifestation of collectivity is the momentum space anisotropy or collective flow of the bulk medium.
Decompose the charged particle spectra into Fourier series of the azimuth angle,
\begin{eqnarray}
E\frac{dN}{p_T dp_T d\phi dy} = \frac{1}{2\pi}\frac{dN}{p_T dp_T dy}\left(1 + 2\sum_{n=1}^{\infty}v_n(p_T)\cos\left[n(\phi-\Psi_n)\right]\right).
\end{eqnarray}
The first term is averaged yield, and subsequent terms in the sum encode the angular dependence. 
The $n=1$ term is a shift of the center-of-mass momentum
From $n=2$, $v_n$s are momentum anisotropy coefficients of $\cos({n\phi})$ modulations.
If the particle production is simply an independent sum of elementary nucleon binary collisions, then the anisotropy would be zero after average. 
However, experiments observe a surprisingly larger elliptic flow ($v_2$), triangular flow ($v_3$) and higher order $v_n$ at both RHIC and LHC in nuclear collisions.
Figure \ref{fig:intro:vn} shows the variation of the $v_n$ as function of centrality from ALICE measurements [].
The $v_2$ coefficients first increases from central to mid-central collisions and slightly decreases at peripheral collisions, while $v_3$, $v_4$ signal are smaller and varies slower with centrality.

In the hydrodynamic picture, a non-central collision deposits initial energy density in an almond shape and the initial fireball has a finite spatial eccentricity $\epsilon_2$.
The energy density is higher in the middle and lower at the boundary, so a hydrodynamic pressure builds up and drives the transverse expansion of the fireball.
Because the pressure gradient is larger in the short axis-direction and the long-axis direction, the matter flows in an isotropic way, creating the observed momentum space second-order anisotropy $v_2$.
The existence of higher order of flow harmonics and non-zero $v_2$ in the most central collisions is because the nuclear configuration fluctuations, such as randomized nucleon positions, create all order of eccentricity $\epsilon_n$.
In short, a hydrodynamic expansion transfer initial geometry eccentricities $\epsilon_n$ into final state momentum anisotropy $v_n$ of the particle.

\paragraph{Extracting the QGP transport coefficients}
A relativistic ideal hydrodynamic model assumes an infinitely strong interaction that the medium always stages in local thermal equilibrium.
A more sophisticated treatment is the relativistic viscous hydrodynamics to account for deviations from the local thermal equilibrium due to large gradients in the expansion.
The response of hydrodynamic evolution to the gradients are characterized by the shear viscosity $\eta$ and bulk viscosity $\zeta$.
The QGP shear viscosity and bulk viscosity are of fundamental importance. 
The shear viscosity to entropy ratio $\eta/s$ is an indicator of the stong / weak coupling nature of the QGP. 
And the bulk viscosity to entropy ratio $\zeta/s$ is directly related to  the scale-invariance breaking of QCD.

Dynamical quantities such as viscosity are extremely hard to be computed from first principle, so currently, the pinning down of these numbers and their temperature dependence resorts to the phenomenological extraction from experiments.
The flow harmonics $v_n$ are particular sensitive to the viscous effects, as a finite viscosity damps the development of anisotropic flows, reducine the transition efficiency from $\epsilon_n$ to $v_n$.
Global comparisons of the state-of-the-art medium modeling to a collection of soft observables have corroborated the need of a small $\eta/s$ that are likely to be slowly increasing with temperature and a non-vanishing $\zeta/s$.

\subsection{Probing sQGP using hard probes}
Very occasionally, an initial collision involves large momentum transfer creates high-$p_T$ particles in the system ($p_T\gtrsim 10$ GeV), are referred to as the ``hard" particles.
By uncertainty principal, they can only be produced at the beginning of the nuclear collision on a time scale $\delta t \sim 1/p_T$, then they pass through and interact with the surrounding bulk medium.
Hard particles can be used as self-generated probes of the system.
Due the asymptotic freedom, the initial production of hard processes can be computed in the perturbative framework, granting a good theoretical control of its initial state.
The final state interaction with the medium then modifies the initial production and leaves finger prints of the medium on the hard probes.

\paragraph{Jet and jet quenching}
Initial hard partons (gluons, quarks) undergoes complex QCD dynamics, radiating more partons which then hadronizes into a collimated bunch of hadrons and decay products.
This final collection of particles is observed in detector as jets.
In the proton-proton collisions, perturbative calculations and Monte-Carlo simulations are able to explains the production cross-section of the jet and leading hadrons (the hardest hadron in the jet) reasonably well.
In the nuclear collisions, the initial parton and the radiative daugther partons interacts with the medium, causing energy loss and triggering medium-induced radiation.
As a result, one expects the jet / leading parton yield at high-$p_T$ is reduced compared to the reference in proton-proton.
To focus on the difference caused by medium effects, the reference has to cancel out a na\"ive difference that simply rises because there are more effective nucleon-nucleon collisions than the proton-proton collisions.
One defines the so-called ``nuclear modification factor'',
\begin{eqnarray}
R_{AA}(y, p_T) = \frac{\frac{dN_{AA}}{dy dp_T}}{\langle N_{\textrm{coll}}\rangle \frac{dN_{pp}}{dy dp_T}} = \frac{\frac{dN_{AA}}{dy dp_T}}{\langle T_{AA} \rangle \frac{d\sigma_{pp}}{dy dp_T}}.
\end{eqnarray}
It is the average $N_{\textrm{bin}}$ ``normalized'' ratio between the yield in AA collisions and pp collision.
The number of binary collision is sometimes replaced by the average nuclear overlapping function $\langle N_{\textrm{coll}}\rangle = \langle T_{AA} \rangle /\sigma_{pp}^{\textrm{inel}}$, and the yield is replaced by the inclusive cross-section for the proton-proton collision $\frac{dN_{pp}}{dy dp_T}\rightarrow \frac{d\sigma_{pp}}{dy dp_T}$.
These two expressions are equivalent.
The ratio is expected to be unity if there is no medium effect, though we do remind the readers that $N_{\textrm{coll}}$ is not a directly observed quantity and has to be estimated in a model-dependent way.

At both RHIC and the LHC, colored probes as measured by the $R_{AA}$ of leading hadron and jet are found to be significantly below unity in nuclear collisions, while  $R_{AA}$ of color neutral probe such as $Z$-boson is consistent with unity.
These discoveries indicate the creation of a color deconfined medium that strongly modifies to hard parton propagation.
The interaction strength between the hard parton and the medium is theoretically quantified as the jet transport parameter $\hat{q}$, which is defined as the momentum broadening per unit path-length in the transverse direction of the propagation,
\begin{eqnarray}
\frac{d\langle p_\perp^2 \rangle}{dL} = \hat{q}
\end{eqnarray}
The jet transport coefficients is another quantity of fundamental interest  in heavy-ion collisions, and there has been a great effort in both first principal computation and phenomenological extraction.
The hard parton / jet probes the medium at different scale at different stage of its evolution, and more sophisticated observables and theoretical tools are being constructed to answer more microscopic questions such as the actually degrees of freedom of the strongly coupled QGP [].

\paragraph{Heavy-flavor probes}
Heavy flavor is a charming probe in studying heavy-ion collisions, and is also the focus of this work.
Heavy quarks have mass well above the QCD non-perturbative scale.
The charm quark ($M=1.3$ GeV) and bottom quark (M=$4.2$ GeV) are the most relevant heavy quarks for the present study.
The reason that top quark ($M = 173$ GeV) is out of our discussion is due to its extremely short life time ($\sim 5\times 10^{-25} \approx 0.15$  fm/$c$ in its rest frame) so it barely interacts with the QGP before it decays into, predominantly, bottom quarks.
Though there has also been proposal that takes the advantage of this short life-time to probe the temporal structure of the QGP evolution, we only focus on the charm and bottom flavor in this work.

\begin{figure}
\centering
\includegraphics[width=.7\textwidth]{CMS-HF.png}
\caption{}
\label{fig:intro:Raa}
\end{figure}

A large mass guarantees a negligible thermal production contribution, at least for the present top LHC beam energies (there are estimations that thermal contribution can play a row in future FCC collider []).
Therefore, heavy flavors, regardless of its $p_T$, are almost always created in initial hard processes.
Moreover, the tiny population of heavy flavors in the collision also suppresses the chances that they annihilates / recombine with their anti-particles.
Certainty heavy mesons have a long life time such that the decay vertices are outside of the fireball and can be resolved by experiments.
Therefore, the number of heavy-flavor particles is almost conserved from the beginning to the end of the entire medium evolution including both the QGP phase and the hadronic phase.

\begin{figure}
\centering
\includegraphics[width=\textwidth]{ALICE-CMS-D-vn.pdf}
\caption{}
\label{fig:intro:D-vn}
\end{figure}

Heavy flavors are of physical interests in many ways.
The mass effects are negligible at high $p_T$ where heavy flavors merge into the topic of the jet energy loss physics.
At intermediate $p_T$, the mass effect is expected to suppress the medium-induce radiation, which dominates the energy loss of light quark.
And there may be a competing between the collisional energy loss and radiation processes.
Experimentally, these differences lead to a hierarchy in the amount of nuclear suppression.
For example, figure \ref{fig:intro:Raa} citing from the CMS collaboration shows a collection of $R_{AA}$ measurements, for charged (mostly light) hadron, prompt D meson, prompt B meson and D and J/$\Psi$ from $b$-decays.
All the $R_{AA}$ tends to collapse onto the same trend at very high-$p_T$, while for $p_T$ range aounrd 5 to 20 GeV, despite the large uncertainty, there is a suggestive hierarchy of $R_{AA}(\textrm{light}) < R_{AA}(\textrm{charm}) < R_{AA}(\textrm{bottom})$.
It would be interesting to study whether the theoretical framework can explain this difference quantitatively and distinguish different energy loss mechanisms.

At low $p_T$, collisional process is thought to dominates over radiative process, and the description of the heavy quark dynamics under the influence of medium ``kicks'' reduces to a succinct diffusion equation.
In a diffusion picture, the large inertia $M\gg T$ delays the relaxation time $\tau_{\textrm{th}} \propto M/T^2$ and is the ideal probe to study the thermalization process in a QGP.
The degree of thermalization of charmness can be seen in its momentum anisotropy.
Figure \ref{fig:intro:D-vn} quotes the results from the ALICE collaboration and CMS collaboration.
The $v_2$ of charged pions and prompt $D$ mesons are plotted as a function of transverse momentum.
At low $p_T$, the large $v_2$ of the charged pions is due to the collective expansion; but the initially perturbatively produced charm flavors also catch up a large amount of momentum anisotropy.
How does charm flavor catches this huge amount of flow during its non-equilibrium evolution is still puzzling, and it may eventually reveal the strongly coupled nature of the QGP in the soft realm.
As a remark, the finite $v_2$ of D meson at high $p_T$ is not directly related to the flow phenomena, but as a result of anisotropic energy loss in a spatial eccentric medium.
Finally, the unique flavors of heavy quarks help to tag certain processes of interest. 
For example, the study of recombination hadronization mechanism, strange ness enhancement, and implementing a selection bias on the quark / gluon-initiated jet ratio.

To understand all these phenomena related to the heavy flavors requires a comprehensive modeling of the non-equilibrium dynamics of heavy quarks and heavy mesons.
At high $p_T$, a heavy-quark transport coefficient $\hat{q}_Q$ can be defined similarly to the light parton.
For the close-to-thermalization diffusion dynamics, the heavy flavor community often uses the spatial diffusion constant $D_s$.
$D_s$ is related to the momentum diffusion coefficient by $D_s = 4T^2/\hat{q}_Q(p=0)$, and this transport coefficients has been evaluated on the lattice by several group.
Phenomenological extraction of this parameter from transport modeling is close to the lattice results and is significantly below the perturbative estimation. 

\section{Extract sQGP properties from model-to-data comparison}
The dynamics of QGP only lasts for $O(10) fm/$c, while we can only observe the remnants at the end of the collisions by detectors meters away.
Therefore, the learning of any intrinsic properties of QGP can be cast into a parameter inference problem:
given the set of measurement, a model, and parameters of QGP such as $\eta/s, \hat{q}$, etc, then find out what are the range of parameters favored by the data.
Then, one can make further comparison to theoretical calculations (if possible) of these parameters. 
Finally, by comparing the theory expectation, and the experimental constraints, one evaluate the assumptions of the theory, which gives out new information of the physics system.

This inversion from observables to parameters is not as simple as it appears, because of the following problems
\begin{itemize}
\item The dynamical models are complex and computational intense.
\item Model takes many parameters or unknown functions that have infinitely many degrees of freedom.
\item Better constraining power comes from global comparison to many experimental observables.
\item The quantification of uncertainty: including experimental uncertainty, model uncertainty and theoretical uncertainty.
\end{itemize}
Most of these issues can be solved by the Bayes analysis for model parameter calibration, and its key ingredients will be explained in the detailed chapter \ref{bayes}.
A remaining issue is the theoretical uncertainties, which is hard to quantify.
The solution is two folded. 
First, if there are more than one theoretical assumptions without appealing reasons to disfavor either of them, then this difference should be propagated into the extracted model parameters.
For example, whether the coupling between the heavy quark is strongly coupled or weakly coupled; is the dynamics diffusion-like or scattering-like; etc.
Including these uncertainty will certainly decrease the constraining power on the transport parameters, but it prevents biasing the estimated number from imposing a too strong assumption.
Second, existing theoretical calculations are often worked out in certain idealized limits, while a dynamical modeling is much more complicated and may not closely follow what the underlying theory.
This will certainly obscure the interpretation of the extracted parameters.
Therefore, as an important practice for dynamical modeling, the model should be able to calibrate to theoretical calculations at least in those idealized limits, and then generalized to the more complex scenario.
We devoted chapter \ref{transport} to improve the accuracy of hard parton transport model.

\section{Outline of this Thesis}
In this thesis, I focuses on the extraction of the heavy quark transport coefficients from experiments using a newly developed transport model for hard parton in a quark-gluon plasma.
In chapter \ref{simulation}, I introduce a hydrodynamic-based medium for the medium evolution. 
As an application of this simulation framework, I review my project on reverse engineering a three-dimensional initial entropy deposition of the heavy-ion collision from experimental data. 
In chapter \ref{transport}, we develop the transport model for hard parton (including heavy quark)  propagation inside a quark-gluon plasma.
This model interpolates a small-angle diffusion picture and a large-angel scattering picture of the probe-medium interaction.
I show the limitation of the semi-classical transport approach in implementing parton branching (radiation) processes at high energy, and demonstrate how to modified the semi-classical approach to treat it properly.
Chapter \ref{coupling} builds a comprehensive simulation workflow that couples the initial production and in-meidum transport of heavy flavors to the bulk medium evolution.
Benchmark calculations with simple guessed parameters are compared to the experimental measurements.
Chapter \ref{bayes} is a brief description of the Bayes methodology of model parameter calibration.
Applying the Bayes method, in chapter \ref{results}, a systematic model-to-data comparison is performed, extracting the heavy flavor transport properties.
Finally, chapter \ref{conclusion} summarizes the work and discuss possible future improvements.



\chapter{Bulk medium evolution and initial condition}
In this chapter, I shall introduce a hydrodynamic based model for describing the bulk medium evolution in the heavy-ion collision.
After the basic description of the modeling, I will show an application of this simulation framework in reverse engineering the three-dimensional initial entropy deposition, which is one of my published research projects.

\section{A Hydrodynamics-based dynamical modeling}
\subsection{The hydrodynamic equations}
There has been vast experimental evidences that supports the use of an relativistic viscous hydrodynamics in describing the dynamical evolution of the medium created in the heavy-ion collision.
At mid-rapidity, the hydrodynamic-based model can achieve a high precision of agreement with a global comparison to experimental data. 
This data also support the use of a small but nonzero specific shearviscosity {Muronga:2004sf, Chaudhuri:2006jd, Romatschke:2007mq, Dusling:2007gi, Song:2007ux, Luzum:2008cw}, and more recent studies and global Bayesian analysis also suggest the use of a finite bulk viscosity.

\paragraph{Ideal hydrodynamics} Hydrodynamics is a macroscopic description that propagates the energy momentum tensor of the system without an microscopic knowledge of the degrees of freedom.
The first four equations comes from the the energy-momentum conservation,
\begin{eqnarray}
\partial_\mu T^{\mu\nu} = 0
\end{eqnarray}
where $T^{\mu\nu}$ is the energy momentum tensor and $\partial_\mu = \partial/\partial x^\mu$. 
Here we have choose the metric as $g^{\mu\nu} = \diag\{1, -1, -1, -1\}$.
If one assumes ideal hydrodynamics that the system always relax to local thermal equilibrium fast enough compared to other time scale, then $T^{\mu\nu}$ can be expressed as
\begin{eqnarray}
\partial_\mu T^{\mu\nu} = e u^\mu \nu - P (g^{\mu\nu}-u^{\mu\nu})
\end{eqnarray}
where $u^\mu$ is the flow velocity such that in the co-moving frame  $T^{\mu\nu}$ has the diagonal form $T^{\mu\nu} = \diag\{e, -P, -P, -P\}$.
Therefore $e$ and $P$ are the energy density and pressure of the thermalized matter in the rest frame.
There are five unknowns $e, P, u_x, u_y, u_z$ ($u_t$ is determined by $u^2 = 1$), but conservation laws only provide four equations.
A fifth equation is the equation of state (EoS) $P = P(e)$ that encodes thermodynamic properties of the medium (in this case, the nuclear medium at finite temperature) which closes the set of ideal-hydrodynamic equation.
\paragraph{Inputs from first principal calculation} The lattice QCD simulation has determined the QCD equation of state to high precision with physical masses of the light flavor.
It is not a prior known that the lattice QCD EoS computed for an infinite matter in infinite time is the best choice for describing a transient, finite size system, however, using the lattice input does result in reasonable agreement with the data.
Moreover, there has been a study that try to constrain the from of EoS from experimental data and the ``calibrated'' EoS is very close the Lattice prediction.
Most of the currently hydrodynamic studies already uses the state-of-the-art lattice results.
\paragraph{Viscosity correction and QCD transport coefficient}
The medium created in heavy-ion collision undergoes fast longitudinal and transverse expansion, and the thermalization rate may not be fast enough to keep the system in fully thermal equilibrium,
\begin{eqnarray}
T^{\mu\nu} = u^\mu u^\nu e - (g^{\mu\nu}- u^\mu u^\nu) (P+\Pi) + \pi^{\mu\nu}
\end{eqnarray}
Here $T_0$ is the thermal part, and $\pi$ is the shear viscosity correction, and $\Pi$ is the bulk viscosity correction to the pressure.
Relativistic viscosity hydrodynamics that expands the transport description in terms of the Knudsen number is developed to taken into account viscous effects.
At first order, this is well-known Naiver-Stokes equations, where the $\pi$, $\Pi$ are given by the constitutive relations,
\begin{eqnarray}
\pi^{\mu\nu} &=& 2\eta\sigma^{\mu\nu}\\
\Pi &=& -\zeta\theta
\end{eqnarray}
Here the corrections to the energy momentum tensor are proportional to the first gradient of the macroscopic field $\sigma^{\mu\nu} = \partial^{\langle \mu} u^{\nu\rangle}, \theta = \partial\cdot u$.
The proportional constant are known as the (first order) transport coefficient of the QCD matter, which encodes the dynamical information of the system. 
Especially their dimensionless ratio to the entropy density $\eta/s$ and $\zeta/s$ are important indicator of the interaction strength and the scale-violation of the QCD matter and are of great physical importance.
There have been many effort to either computing these quantities from first principles and effective models or extracting these numbers in a model-to-data comparison.
Coming back to the hydrodynamic modeling, it is shown that one has to go to second order in the expansion to the render the correction compatible with special relativity and $\pi$, $\Pi$ becomes dynamical quantities which tends to relax to the Naiver-Stokes limit.
\begin{eqnarray}
\nonumber
\tau_\pi \dot{\pi}^{\langle\mu\nu\rangle}+\pi^{\mu\nu} &=& 2\eta\sigma^{\mu\nu}- \delta_{\pi\pi}\pi^{\mu\nu}\theta + \phi_7 \pi_{\alpha}^{\langle\mu}\pi^{\nu\rangle\alpha}-\tau_{\pi\pi}\pi_{\alpha}^{\langle\mu}\sigma^{\nu\rangle\alpha} + \lambda_{\pi\Pi}\Pi\sigma^{\mu\nu},
\\
\nonumber
\tau_{\Pi}\dot{\Pi} + \Pi &=& -\zeta\theta - \delta_{\Pi\Pi}\Pi\theta + \lambda_{\Pi\pi}\pi^{\mu\nu}\sigma_{\mu\nu}.
\end{eqnarray}
The $\delta, \phi, \lambda$ are known as second order transport coefficients.
This complicated set of equations together with the conservation law and the EoS forms the viscosity hydrodynamic equations.
Nowadays, well tested numerical packages have been developed solving these equations in the context of heavy-ion collisions.

\paragraph{Boost-invariance approximation and beyond}
In general, the hydrodynamic equations have to be solved as a 3+1 dimensional problem.
But a reduction to a 2+1 dimension is possible, if one assume the fast expansion along the beam-axis (longitudinal) direction has an approximated symmetry near mid-rapidity.
This approximate symmetry is first proposed by Bjorken in [] that the system at different rapidity looks similar upto a longitudinal boost, so that one may first obtain the solution to the 2+1 dimension problem at one space-time rapidity ($\eta_s = 0$) and then boost the solution to other $\eta_s$.
A direct consequences of this symmetry is that the observables should not depends on space-time rapidity.
The $\eta_s$-dependence cannot be directly detected, but can be related to the the rapidity / pseudo-rapidity for the reason below.
Suppose that the Lorentz contracted nuclei interact at $z=0$ and produce excitations that free-stream in the longitudinal direction, then for each of these excitation
\begin{equation}
  \frac{z}{t} = \frac{p_z}{E}.
\end{equation}
This approximation the equivalence of $\eta_s$ and $y$ at early stages of the collision:
\begin{equation}
  \eta_s = \frac{1}{2}\log\frac{t+z}{t-z} \sim y = \frac{1}{2}\log\frac{E+p_z}{E-p_z}.
\end{equation}
If one further assumes these initial excitations are mass-less partons, then the rapidity can be approximated by pseudorapidity as $E\approx |p|$.
The event-averaged rapidity-distribution of charged particles $dN_{\textrm{ch}}/dy$ in both proton-proton and symmetric nuclei-nuclei collisions at the LHC energy falls at large rapidity but has a slowly varying profile at mid-rapidity within $|y|<2$, which is a motivation to apply the boost-invariant at mid-rapidity as a first approximation.

However, since $dN_{\textrm{ch}}/dy$ are event-averaged quantities, its being flat within $|y|<2$ does not rule out that event-by-event particle production fluctuation can break the approximation, and these fluctuations, local in transverse plane, can be different at different transverse location.
Moreover, asymmetric nuclear collision such as $p+Pb, p+Au, d+Au, He+Au$, etc clearly breaks the boost-invariance even on an event averaged level. 
Therefore, the study of longitudinal fluctuation related observables or the search for hydrodynamic  behavior in small collision system clearly requires one to go beyond the boost-invariance approximation.

\subsection{Particularization and microscopic transport}
The longitudinal expansion and the transverse expansion driven by the pressure difference cools down the system temperature and density rapidly and eventually the relaxation rate is too low for the the hydrodynamic approach to apply.
When this happens, one can switch to the microscopic transport description of the system.
This switching is usually done near or below the pseudo-critical temperature $T_{sw} \lesssim T_c$ so that the energy momentum tensor can be particularized as an ensemble of hadrons .
This choice $T_{sw} \lesssim T_c$ avoids the problem of modeling quark / gluon dynamics and hadronization in the strong coupled regime near $T_c$, and whether a hydrodynamics with lattice EoS and a hadronic transport model with cross-sections as inputs are both valid in this regime is another question.
The hydrodynamic $T^{\mu\nu}$ is usually particlized at a constant energy density / temperature hypersurface using the Cooper-Frye formula,
\begin{eqnarray}
dN_i^a(p) = \frac{g^a f^a(p) dp^3}{(2\pi)^3}  \frac{p^{\mu}}{E} \Delta \sigma_{i,\mu} 
\end{eqnarray}
Where the particle yield of specie ``$a$'' with momentum $p$ from the $i^{\textrm{th}}$ surface element $\sigma_{i,\mu}$ equals its phase space density times the surface area (with units of a 3D volume) parallel to the  four velocity $p^\mu/E$.
This distribution function should include both a thermal part and a viscous correction, $f = f_0 + \delta f$.
There are more than one way to construct the viscous correction $\delta f$  from $e, P, n, \Pi$ and $\pi^{\mu\nu}$ based on different assumptions of the form of corrections.
In this work, we shall use a non-additive $\delta f$ correction that has been developed and implemented by [] and [], and please refer to the references for the original formulation and numerical implementations details.

The particlized hadronic system is then solved by the Ultra-relativistic Quantum Molecular Dynamics (UrQMD) model until the system is dilute enough and interactions ceases (kinetic freezeout). 
The UrQMD in the context of hadronic cascade (afterburner) includes processes like resonance decays, elastic and inelastic scatterings, and string formations and fragmentations.

\subsection{Pre-equilibrium stage}
At very early times of the collision, the system is off equilibrium. 
However, viscous hydrodynamic assumes a closeness to the local thermal equilibrium  (there are also recent efforts in studying the effectiveness of hydrodynamics outside of its traditional range of application []).
A successful prediction using an early onset of hydrodynamic evolution at $\tau_0\lesssim 1$fm/$c$ suggests a fast hydrodynamization, whose exact mechanism is still a debatable question. 
There are different modelings of this pre-equilibrium stage, including solving the classical Yang-Mills equation [], partonic transport models [], a collision-less Boltzmann equation [], and the linear response method of the effective kinetic approach [].

Such a pre-equilibrium stage was not included in my study of the 3D initial condition, where the simulation starts at $\tau_0 = 0.6$ fm/$c$ assuming local thermal equilibrium.
For my later study of the heavy flavor dynamics, a 2+1D collision-less Boltzmann equation implemented by [] was later used in obtaining the medium evolution history.
In such a model, the initial energy density ($\tau = 0^+$) at mid-rapidity  is thought to be carried by mass-less partons that propagates at the speed of light in the transverse direction. 
The initial distribution function is assumed to have a factorized form $f(x_\perp, p_\perp, \tau=0) = n(x, \tau=0) \times dN/dp_\perp^2$.
The momentum distribution $dN/dp_\perp^2$ do not evolve as the collisions are neglected, while the spatial density evolves as
\begin{eqnarray}
n(\vec{x}, \tau) = \int n(\vec{x}', \tau=0) \delta^{(2)}(\vec{x} - \vec{x}'- \tau) d\vec{x}'^2
\end{eqnarray}
Then, the model assumes a sudden hydrodynamization at time $\tau_{\textrm{hydro}}$, where the free-streamed,
\begin{eqnarray}
T^{\mu\nu}(x_\perp, \tau_{\textrm{hydro}}) = \int f(x_\perp, p_\perp, \tau=0) \frac{p^\mu p^\nu}{E} dp^3
\end{eqnarray}
is used for initializing the hydrodynamic equations by the Landau matching procedure [].

\section{Initial condition model}
Unlike the dynamical models that are governed by a few equations / laws with a few parameters, the initial condition model parameterizes many more unknowns.
For different initial condition models, these unknowns can be initial color density of the nuclear wave function, the effective size of a nucleon, the form factor of nucleon-nucleon inelastic cross-section, and the amount of fluctuations in particle production / energy deposition, etc.
There are two classes of initial condition models:
\begin{itemize}
\item Models that takes into the particle production dynamics. Such as minijet production, strings productions and flux-tubes, hadronic transport,and color-glass condensate (CGC) effective field theory based models {Wang:1991hta, Zhang:1999bd, Werner:2010aa, Petersen:2008dd, Schenke:2016ksl, Dumitru:2011wq, Hirano:2012kj}.
\item Parametric models that provides macroscopic initial conditions without a dynamical component. Such as Monte-Carlo Glauber models and its extensions [Bozek:2015bna], and the \trento model to be explained [Scott].
\end{itemize}
I shall focus on the development of a parametric model \trento to help the understanding of the initial three-dimensional entropy deposition.


\subsection{The original (boost-invariant) \trento\ model}
The original \trento\ model is proposed as a flexible ansatz to investigate a family of entropy / energy deposition behaviors at mid-rapidity.

First, the impact parameter $\vec{b}_{AB}$ between the two colliding nuclei $A$ and $B$ is sampled at random.
Then, the 3D nucleons positions inside each nuclei are sampled according to the Woods-Saxon distribution (for heavy nuclei, for light nuclei such as Deuteron, Helium, Oxygen where the nucleon distribution is not well approximated by a Woods-Saxon form, the sampling are different),
\begin{eqnarray}
\frac{df_N}{r^2 dr d\phi d\cos\theta} = \frac{1}{\exp\{\frac{r-R(1+\beta_2 Y_{20}(\theta)+\beta_4 Y_{40}(\theta))}{a}\}+1}
\end{eqnarray}
including the quadrupole and hexadecapole deformation of certain nuclei.
The randomized nucleon position is critical to explain the odd order of flow harmonics observed in experiments [].

Then, the collision between the two nuclei is analyzed at the nucleon level. 
Every nucleon pair $\{i, j\}$ with $i$ from nuclei $A$ and $j$ from nuclei $B$ has a certainty probability to undergo an inelastic collision, given the impact parameter $\vec{b} = \vec{b}_{AB} + \vec{x}_{i, \perp} -  \vec{x}_{j, \perp}$,
\begin{eqnarray}
P(b; \sigma) = 1 - \exp\left[-\sigma T_{pp}(b)\right],
\label{dsigma_db}
\end{eqnarray}
where $T_{pp}(b)$ is the overlapping function between the density of the nucleon,
\begin{eqnarray}
T_{pp}(b) = \int d\vec{x}_\perp^2 T_p(\vec{x}_\perp-\vec{b}/2) T_p(\vec{x}_\perp+\vec{b}/2)
\end{eqnarray}
And each nucleon is assumed to have a Gaussian density profile, 
\begin{eqnarray}
T_p(\vec{x}_\perp^2) = \frac{1}{2\pi w^2} \exp\left(-\frac{\vec{x}^2}{2w^2}\right)
\end{eqnarray}
with the width of the nucleon $w$ a free parameter.
The $\sigma$ is an effective constituent cross-section that is determined by reproducing the experimental measured proton-proton inelastic cross-section at a given beam energy,
\begin{eqnarray}
\sigma_{pp}^\text{inel}(\sqrt{s}) = \int d\vec{b}^2 P(b; \sigma(\sqrt{s}))
\end{eqnarray}
Applying the probabilistic collision criteria of equation \ref{dsigma_db} to each pair of nucleons, the nucleons that suffer at least one inelastic collisions are called participant and the total number of binary inelastic collisions are denoted as $N_{\textrm{bin}}$.
The minimum-biased event sample in the \trento model are defined all events that has at least one binary collision, 

The above procedure is similar to the that of an Monte-Carlo Glauber model in determine the nuclear inelastic cross-section.
The key step of \trento is an ansatz that maps from the participants in a particular event to the energy / entropy density deposited at the mid-rapidity.
We fist define the participant densities,
\begin{equation}
T_{A, B}(\x) = \sum_{i\in \textrm{Parts}_{A, B}} w_i\, T_p(\x - \x_i).
\end{equation}
The summation goes over the participants in nuclei $A$ and $B$, and each participant contribute a fluctuating contribution proportional to its transverse density $T_p$.
The fluctuating weight $w_i$ follows a $\Gamma$-distribution with unit mean and variance $1/k$ where $k$ is a parameter.
This fluctuating contribution is put in to mimic the multiplicity fluctuation in the proton proton collisions.
The entropy / energy density deposit at mid-rapidity is assumed to be a function of $T_A$ and $T_B$ only at each transverse location,
\begin{eqnarray}
\frac{dS(\vec{x})}{d\eta dx_\perp^2} \textrm{ or } \frac{dE(\vec{x})}{d\eta dx_\perp^2} = f(T_A(\vec{x}), T_B(\vec{x})).
\end{eqnarray}
This simplification is possible because at $\tau=0^+$, causality requires that the entropy production at one location cannot be correlated with the information at a different transverse location. 
Also, the partons that contribute to the bulk low-$p_T$ particle production at high $\sqrt{s}$ are predominately low energy gluons whose longitudinal wavelength is longer than the contracted proton radius in the $z$-direction; therefore, the entropy production should not be sensitive to the details of how the participants aligned but only its $z$-integrated density.
\trento\ parametrizes this mapping from $T_A$ and $T_B$ to the energy / entropy deposition using a ``generally mean'' ansatz,
\begin{eqnarray}
f(T_A, T_B; p) = \left(\frac{T_A^p + T_B^p}{2}\right)^{1/p}.
\end{eqnarray}
$p$ is a tunable parameter and with a few special values, it reduces to the well known average procedures as in table \ref{tab:trento-p}
\begin{table}
\centering
\caption{\trento\ $p$-parameter}\label{tab:trento-p}
\begin{tabular}{ccc}
\hline
$p\in \mathbb{R}$ & $f(x, y)$ & Entropy / energy production\\
\hline
$-\infty$ & $\min\{x, y\}$ &  dominated by the thinner target\\
$-1$ & $2xy/(x+y)$ &   the harmonic mean scaling\\
$0$ & $\sqrt{xy}$ &  the geometric mean scaling\\
$1$ & $(x+y)/2$ &  the arithmetic mean (participant) scaling\\
$+\infty$ & $\max\{x, y\}$ &  dominated by the thicker target \\
\hline
\end{tabular}
\end{table}
This way the model is able to parametrize a class of entropy / energy production scheme and includes certain type of initial condition uncertainty.
Through a global model-to-data comparison, this $p$ parameter has been calibrated to be very close to 0, suggesting the data favors a mid-rapidity entropy / energy deposition that scales as $(T_A T_B)^{0.5}$.
A similar scaling is also found in the EKRT initial condition model based on pQCD plus saturation physics.

\subsection{Parametrize the longitudinal dependence in \trento}
The \trento\ model has been doing successful phenomenology for obserables at mid-rapidity [].
My later work focus on extending the parametrization to a rapidity-dependent initial condition, and seek for a reverse-engineered 3D entropy production from the rapidity-dependent observables.

One constrain in extending the model is that it should reduces to the boost-invariant version of the model at mid-rapidity; therefore, I take the following decomposition of $s(\x, \eta_s)$ at the hydrodynamic starting time $\tau_0 = 0.6$ fm/c,
\begin{equation}
  s(\x, \eta_s)\vert_{\tau=\tau_0} \propto f(T_A(\x), T_B(\x)) \times g(T_A(\x), T_B(\x), \eta_s),
  \label{factorized}
\end{equation}
$f$ is the entropy production at midrapidity as explain above.
Now $g$ parametrize the rapidity-dependence and is always normalized such that $g(T_A(\x), T_B(\x)), 0)=1$.
Qualitatively, the breaking of boost-invariance can come from the initial asymmetry of the incoming nuclear matter $T_A \neq T_B$ and various sources of fluctuation in particle production.
We listed the sources of these contributions and indicated what contribution has been included and what is not,
\begin{itemize}
\item In asymmetric collisions like $p$-$A$ and non-central $A$-$A$, the local thickness functions are imbalanced $T_A \neq T_B$.
\item Even in central $A$-$A$ collision, nuclear / nucleon configuration fluctuations also contribute to the asymmetry. These are included as the randomized nucleon position fluctuation and the $\gamma$-fluctuation of the nucleon thickness function.
\item Initial stage dynamical evolution which can introduce dynamical fluctuations and non-vanishing flow in the $z$-direction is not included.
\end{itemize}
Therefore, the asymmetry in the extended \trento\ model only comes from the imbalanced and fluctuating participant density function.
We parametrize this function in terms of rapidity and then transformed to the space-time rapidity.
\begin{eqnarray}
g(\x, \eta) &=& g(y; T_A(\x), T_B(\x)) \frac{J \cosh \eta_s}{\sqrt{1 + J^2 \sinh^2 \eta_s}},
\label{jacobian}
\end{eqnarray}
where the species-dependent factor $J$ is replaced with an effective value $J \approx \langle p_T \rangle / \langle m_T \rangle$.
Instead of determining a single number at mid-rapidity, we need to map $T_A$ and $T_B$ to a function $g(y; T_A(\x), T_B(\x))$ that has infinite degrees of freedom.
To do these, we choose to parametrize how the $y$-cumulents of $g$ changes with $T_A$ and $T_B$, and reconstruct the function using its first few cumulants (mean $\mu$, standard deviation $\sigma$, and skewness $\gamma$)
\begin{eqnarray}
g(\x, y) &=& \mathcal{F}^{-1}\{\tilde{g}(\x, k)\}, \\
\log \tilde{g} &=&  i \mu k - \frac{1}{2}\sigma^2 k^2 - \frac{1}{6} i \gamma \sigma^3 k^3  e^{-\frac{1}{2}\sigma^2 k^2} + \cdots \label{cgf}
\end{eqnarray}
where for the skewness term, we have included an exponential that systematically includes higher order cumulants to regulate the behavior of the function at large $y$.
Numerically, we have found that within the range of $|y| < 3.3\sigma$, the reconstructed function has a good behavior and remains positive definite.

The remaining task is to parametrize the way these cumulants depends on the participant density,
\begin{itemize}
\item For the mean parameter, we assume it is proportional to the center-of-mass rapidity of the local incoming nuclear matter $\mu = \mu_0\eta_\text{cm}$,
\begin{equation}
  \eta_\text{cm}=\frac{1}{2} \log \left[\frac{T_A e^{y_b}+T_Be^{-y_b}}{T_A e^{-y_b}+T_B e^{y_b}}\right]
\end{equation}
where $y_b$ is the beam rapidity.

\item For the standard deviation, currently we leave it as a global parameter independent on the transverse location $\sigma = \sigma_0$, but only a function of the center-of-mass energy.
\item Finally, for the skewness, there is no clear reason to choose a particular form, so we tested two parametrizations. And in the end, we will check whether the 3D initial condition extracted from data is sensitive to the different choice of parametrizations.
These two choices are termed ``relative skewness'' and ``absolute skewness''.
\begin{itemize}
\item The ``relative skewness'' parametrization assume a skewness proportional to the relative difference of $T_A$ and $T_B$,
\begin{equation}
  \mathcal{A}(T_A, T_B) = \gamma_r\frac{T_A - T_B}{T_A + T_B},
\end{equation}
\item The ``absolute skewness'' parametrization assume a skewness proportional to the direct difference of $T_A$ and $T_B$,
\begin{equation}
  \mathcal{A}(T_A, T_B) = \gamma_a \frac{T_A - T_B}{T_0},
\end{equation}
where a unit for thickness function $T_0=1$~fm$^{-2}$ restores dimensionless of the $\gamma$ parameter.
\end{itemize}
\end{itemize}
We have summarized the two parametrizations in table \ref{tab:parametrization}, and $\mu_0$, $\sigma_0$, $\gamma_r$ or $\gamma_a$, along with the effective Jacobian $J$ are the four additional parameters one introduced for the three-dimensional extended \trento model.
\begin{table}
\centering
\caption{Table}\label{tab:parametrization}
\begin{tabular}{lccc}
\paddedhline
Model & mean $\mu$ & std.\ $\sigma$ & skewness $\gamma$ \\
\paddedhline \noalign{\smallskip}
Relative  & $\frac{1}{2} \mu_0 \ln\left(\frac{T_A e^{y_b}+T_B e^{-y_b}}{T_A e^{-y_b} + T_B e^{y_b}}\right)$ & $\sigma_0$ & $\gamma_r \dfrac{T_A - T_B}{T_A + T_B}$ \smallskip\\
Absolute & $\frac{1}{2} \mu_0 \ln\left(\frac{T_A e^{y_b}+T_B e^{-y_b}}{T_A e^{-y_b} + T_B e^{y_b}}\right)$  & $\sigma_0$ & $\gamma_a (T_A - T_B)/T_0$\smallskip\\
\paddedhline 
\end{tabular}
\end{table}

In figure \ref{fig:3d-example}, we show two sample events from a $Pb$-$Pb$ collision (top plots) and a $p$-$Pb$ collision (bottom plots) generated by \trento\ .
The 3D initial entropy densities are sliced at mid-rapidity $\eta_s=0$ (left plots) and at the $x=0$ plane (right plots).
The mid-rapidity results are identical to the one predicted in the original \trento\ model.
The model is capable of generating fluctuating longitudinal structures that are local in the transverse plane, and breaks the boost-invariance both locally and globally.
For the $p$-$Pb$ event, one can clearly sees that one hot spot extends into the proton going side $\eta_s >0$, while the participant clusters from the lead nuclei pushes the entropy production into the lead going side $\eta_s <0$.

\begin{figure}
\centering
\includegraphics{trento3d_example}
\caption{Initial entropy density in sample Pb+Pb (upper) and p+Pb (lower) events for cross sections of the $\eta=0$ and $x=0$ planes (left and right columns). Event is constructed using the relative skewness model in Table~\ref{tab:parametrization} with $\mu_0=1$, $\sigma_0=3$ and $\gamma_0=6$ along with midrapidity parameters from {Bernhard:2016tnd}.}
\label{fig:3d-example}
\end{figure}

\section{Reverse engineering the 3D initial condition}
In the final section of this chapter, I shall demonstrate applying both the hydrodynamic-based bulk simulation and the flexible \trento-3D initial condition to reverse engineer the 3D entropy deposition at the onset of hydrodynamics at the LHC energies.
Experimentally, one can only measure rapidity-dependent observables on an event averaged level, which already integrates the contribution of particle production over the transverse plane; while our parametrization in the \trento\ model only involves local functions of the participant density function.
Therefore, it is indeed a nontrivial task to infer the functional form of local entropy production $s(\x, \eta_s)$ from these ``global'' measurements.
The suitable statistical technique for parameter inference is the Bayesian 
analysis, which will be introduced in a great detail in chapter \ref{}.

\subsection{Sensitive observables to the initial entropy deposition}
\paragraph{The single particle spectra}
The most direct observable is the charged particle pseudo-rapidity density $\dnchdy$ measured for different collision systems and centralities.
The ALICE collaboration and the ATLAS collaboration has measured this quantity for both $Pb$-$Pb$ system ($-3.5<\eta<5.0$) and $p$-$Pb$ system ($|\eta| < 2.7$) {Abbas:2013bpa,ALICE:2015kda,Aad:2015zza}.
$\dnchdy$ can very well constrain the global rapidity profile and the centrality dependence of the mode, while the limitation being that it is less sensitive to the amount of longitudinal fluctuation.

\paragraph{Two particle pseudo-rapidity correlation}
A good probe of event-by-event longitudinal fluctuations is the two-particle pseudorapidity correlation observable $C(\eta_1, \eta_2)$.
The long range part of $C(\eta_1, \eta_2)$ can be shown to sensitive to initial state.
This is because correlation between particles separated by a large gap at proper time $\tau$ in rapidity can only come from proper time before $\tau e^{-|\eta_1-\eta_2|/2}$ even if the information travels at the speed of light.
For example, if one assumes the two particles separated by 4 units of rapidity are emitted at a constant $\tau = 8$ fm/$c$ hydrodynamic freeze-out hyper-surface and neglects the long range correlation that is built by the hadronic cascade, then any correlation must have come from before the proper time $ 8 e^{-2}\approx 1$ fm/c.

To see how $C(\eta_1, \eta_2)$ is related to the longitudinal fluctuation of the entropy deposition / particle production, we follow the analysis in  {Bzdak:2012tp, Jia:2015jga, ATLAS:2015kla} to decompose $\dnchdy$ for each event in a finite pseudo-rapidity window $[-Y, Y]$ using the normalized Legendre polynomials basis,
\begin{eqnarray}
T_n(x) &=& \sqrt{n + \frac{1}{2}} P_n(x)
\end{eqnarray}
And the event-wise charged particle distribution is then,
\begin{eqnarray}
\frac{dN}{d\eta} &=& \biggl\langle\frac{dN}{d\eta}\biggr\rangle \biggl[1 + \sum_{n=0}^\infty a_n T_n\left(\frac{\eta}{Y}\right) \biggr]
\end{eqnarray}
Where $\biggl\langle\frac{dN}{d\eta}\biggr\rangle$ is the reference multiplicity at mid-rapidity for an ensemble of events.
$a_0$ is the total multiplicity fluctuation and $a_1$ controls how the multiplicity distribution is tilted in rapidity in each event and so on.
Two-particle correlation $C(\eta_1, \eta_2)$ measures the variance of these $a_n$ coefficients.
>>>HERE<<< Because to have one  $R(\eta) = dN/d\eta /\langle dN/d\eta\rangle$ and is decomposed into symmetrized polynomials $T_{mn}(\eta_1, \eta_2)$
\begin{align}
  C(\eta_1, \eta_2) &= \left\langle R(\eta_1) R(\eta_2)\right\rangle \\
  &= 1 + \sum_{m, n}\langle a_m a_n\rangle  T_{mn}(\eta_1, \eta_2),  \\
  T_{mn}(\eta_1, \eta_2) &= \frac{T_n(\eta_1)T_m(\eta_2) + T_m(\eta_1)T_n(\eta_2)}{2}.
\end{align}
Centrality fluctuations introduce nonzero $\langle a_0 a_n\rangle$ and are removed by renormalizing $C(\eta_1, \eta_2)$,
\begin{align}
  C_N(\eta_1, \eta_2) &= \frac{C(\eta_1, \eta_2)}{C_1(\eta_1)C_2(\eta_2)},\\[.5ex]
  C_{1,2}(\eta_{1,2}) &= \int_{-Y}^{Y}C(\eta_1, \eta_2)\frac{d\eta_{2,1}}{2Y}.
\end{align}
The forward-backward multiplicity fluctuations characterized by $\langle a_m a_n\rangle$ with $m, n > 0$ can then be projected out from the renormalized correlation function as
\begin{align}
  C_N(\eta_1, \eta_2) \sim 1 + \frac{3}{2}\langle a_1 ^2 \rangle \frac{\eta_1\eta_2}{Y^2} + \cdots.
\end{align}

The ATLAS collaboration has measured the centrality dependence of various $\langle a_m a_n\rangle$ in Pb+Pb collisions {ATLAS:2015kla, SoorajRadhakrishnanfortheATLAS:2015eqq} using particles with $p_T > 0.5$~GeV.
Recent theoretical work on these observables includes a longitudinal extension of IP-Glasma initial conditions {Schenke:2016ksl} and a rapidity-dependent constituent-quark MC-Glauber model which was embedded in three-dimensional hydrodynamic simulations {Denicol:2015bnf, Monnai:2015sca}.
It was shown that short range correlations (SRC) from resonance decays are a significant contribution to the $\langle a_m a_n\rangle$ signal, while variations in the transport coefficients have a much smaller effect {Denicol:2015bnf}.
The contributions from LRC and SRC were estimated in a subsequent ATLAS analysis {Jia:2016jlg} using correlations of same- and opposite-signed charged particles with $p_T > 0.2 \textrm{ GeV}$.
The isolated LRC, which were measured with a lower $p_T$ cut, are a cleaner observable for the study of initial state fluctuations, but they are not yet calculated for central Pb+Pb collisions.
We therefore perform the calculation {ATLAS:2015kla} with proper modeling of the SRC using the UrQMD model.

\section{Model-to-data comparison}

\begin{table}[t]
  \caption{Input parameter ranges for the rapidity-dependent parametric initial condition model.}

    \begin{tabular}{lll}
      Parameter & Description	& Range \\
      \paddedhline
      $N_{\textrm{p+Pb}}$    & Overall p+Pb normalization      & 140.0--190.0 \\
      $N_{\textrm{Pb+Pb}}$   & Overall Pb+Pb normalization     & 150.0--200.0  \\
      $p^\dagger$	                   & Generalized mean parameter      & -0.3--0.3  \\
      $k$	                   & Multiplicity fluct.\ shape      & 1.0--5.0  \\
      $w$	                   & Gaussian nucleon width     & 0.4--0.6  \\
      $\mu_0$                & Rapidity shift mean coeff.\     & 0.0--1.0  \\
      $\sigma_0$             & Rapidity width std.\ coeff.\    & 2.0--4.0  \\
      \multirow{2}{*}{$\gamma_0$}             & \multirow{2}{*}{Rapidity skewness coeff.\ }      & 0.0--10.0 (rel) \\
                  &        & 0.0--3.6 (abs)  \\
      $J$	                   & Pseudorapidity Jacobian param.  & 0.6--0.9
    \end{tabular}
   \raggedright{$\dagger$ Priori probability distributions fitted from {Bernhard:2016tnd} are applied on this parameter independently within the given ranges.}
  \label{tab:parameters}
\end{table}

The aforementioned rapidity extension introduces several new model parameters which necessitate rigorous optimization.
For this purpose, we apply established Bayesian methodology {OHagan:2006ba, Dave:pca, Higdon:2014tva, Wesolowski:2015fqa} to constrain the proposed parametric initial conditions and extract intrinsic, local features of the QGP fireball using macroscopic event-averaged quantities.
Ideally, one would run the full model calculation at each design point and calibrate the model to fit a comprehensive list of experimental measurements.

Unfortunately, three-dimensional viscous hydrodynamic simulations require an order of magnitude more computing resources than the boost-invariant models previously used in Bayesian analyses.
This makes it difficult to calibrate on statistically intensive observables such as multiparticle flow correlations which require tens of thousands of minimum bias events at each design point. 

Work is underway to solve these technical challenges by migrating three-dimensional viscous hydrodynamic simulations to graphics cards which offer dramatic performance enhancements over processors at a fraction of the cost {Bazow:2016yra}.
We instead omit higher order flow observables and calibrate only on the rapidity-dependent charged particle yields and two-particle pseudorapidity correlations which can be calculated with a few thousand events. 

It should also be noted that the data we use for Pb+Pb and p+Pb collisions are taken at different beam energies, ${\sqrts = 2.76}$~TeV and 5.02~TeV respectively.
Because phenomenological model parameters generally change with beam energy, it is not fully consistent to optimize a single set of parameters to fit experimental observables at two different energies.
Here we assume that all model parameters, except for the overall entropy normalizations, do not change drastically with the beam energy of the two datasets since the beam rapidity changes less than $8\%$ and perform a simultaneous multi-parameter fit using both measurements.

We now briefly summarize the procedure used to apply Bayesian methodology to the newly constructed parametric initial condition model. For a more comprehensive explanation see {Novak:2013bqa, Bernhard:2015hxa, Bernhard:2016tnd}. All steps are repeated for both the relative and absolute skewness models described in Table~\ref{tab:parametrization}.

\subsection{Parameter design}

The three-dimensional parametric initial conditions are specified using nine model parameters.
Five control entropy deposition at midrapidity:
\begin{itemize}[itemsep=0pt]
  \item[1--2.] two normalization factors for Pb+Pb and p+Pb collisions at $\sqrts=2.76$~TeV and 5.02~TeV beam energies,
  \item[3.] the generalized mean parameter $p$ which modulates entropy deposition at midrapidity,
  \item[4.] a gamma shape parameter $k$, which controls the variance of proton-proton multiplicity fluctuations,
  \item[5.] and a Gaussian nucleon width $w$, which determines initial state granularity;
\end{itemize}
the remaining four parameters add rapidity dependence to the model:
\begin{itemize}[itemsep=0pt]
  \setcounter{enumi}{6}
  \item[6--8.] three coefficients which modulate the local rapidity distribution's shift $\mu_0$, width $\sigma_0$, and skewness $\gamma_0$,
  \item[9.] and a Jacobian factor $J$ for the conversion from rapidity to pseudorapidity.
\end{itemize}

Given the large number of iterations required by multidimensional Monte Carlo optimization methods and the non-negligible computation time needed to sample initial condition events and evolve them through hydro, it is intractable to explore the aforementioned parameter space using direct model evaluation.
To circumvent this issue, we train emulators using a limited number of parameter configurations to reproduce the charged-particle pseudorapidity density and the two-particle pseudorapidity correlations predicted by evolving the initial condition model through hydrodynamics.
These emulators interpolate the predictions of the model between training points and provide essentially instant predictions at uncharted regions of parameter space.

The emulators are trained using 100 unique parameter configurations sampled from the parameter ranges listed in Table~\ref{tab:parameters}.
Each parameter design point is distributed in the nine dimensional space using a maximin Latin hypercube design---a space-filling algorithm that maximizes the minimum distance between pairs of points in the multidimensional space.

With the parameter design in hand, we run $4\times 10^3$ Pb+Pb and $10^4$ p+Pb initial condition events through hydrodynamics at each of the 100 points and calculate the charged-particle pseudorapidity density and two-particle pseudorapidity correlations.
The centrality bins are defined by charged particle multiplicity using the same kinematic cuts used by experiments: $|\eta|<0.8$ for Pb+Pb collisions and ${-4.9 < \eta < -3.1}$ for p+Pb collisions.
The resulting $\dnchdy$ and rms $a_1$ are concatenated to an observable array for each input parameter set.
Loosely speaking, the physics model maps the $m\times n$ parameter design matrix $X$ to an $m \times p$ observable matrix $Y$:
\begin{equation}
  \begin{pmatrix}
    x_{1,1} & \cdots & x_{1,n} \\
    \vdots  & \ddots & \vdots \\
    x_{m,1} & \cdots & x_{m,n} \\
  \end{pmatrix}
  \xrightarrow{\text{Model}}
  \begin{pmatrix}
    y_{1,1} & \cdots & y_{1,p} \\
    \vdots  & \ddots & \vdots \\
    y_{m,1} & \cdots & y_{m,p} \\
  \end{pmatrix},
  \label{design-obs}
\end{equation}
where $m=100$ is the number of design points, ${n=9}$ is the number of input parameters, and $p$ is the number of measured outputs.

\subsection{Calibrated observables}
The 3+1D relativistic viscous hydrodynamics code \mbox{vHLLE} {Karpenko:2013wva} is used for the QGP medium evolution. 
The equation-of-state (EoS) is obtained by interpolating a state-of-the-art lattice-QCD EoS {Bazavov:2014pvz} at high temperature with vanishing baryon density and a hadron resonance gas EoS at low temperature.
We use a switching energy density $\varepsilon_s = 0.322$~GeV/fm$^3$ ($T_s\sim0.154$~GeV) at which the hydrodynamic description is switched to the UrQMD transport description. 
The switching temperature $T_s$ is the same as the EoS pseudo-critical temperature $T_c = 0.154$~GeV. 
The hydrodynamic transport coefficients are given by:
\begin{align}
  (\eta/s)(T>T_s)  &=  \text{0.17--0.28}, \\
  (\zeta/s)(T>T_s) &=  0.0.
\end{align}
For simplicity, there is no bulk viscosity and the shear viscosity to entropy ratio is a constant.
Below $T_s$, the hydrodynamic energy density is particlized into hadrons and UrQMD takes over the time evolution of the hadronic system.
No additional inputs for the transport coefficients are needed.

To investigate the performance of the calibrated models, we show in Fig.~\ref{fig:post_obs} the resulting observables calculated from each model's calibrated emulators.
The bands are centered around the mean prediction, and their spread denotes $\pm 2$ standard deviations.
Both calibrated models are able to simultaneously describe $\dnchdy$ for the two collision systems as functions of rapidity and centrality, illustrating the flexibility of the generating function approach.

The results for rms $a_1$ are compared to preliminary data from ATLAS {ATLAS:2015kla} in Fig.~\ref{fig:post_obs}.
Both models capture the increasing trend of rms $a_1$ as function of centrality.
Hybrid model calculations agree with experimental measurements within $20\%$ for $0$--$50\%$ centralities ($N_{\textrm{part}} \gtrsim 75$) but underestimate the data at more peripheral centralities.
We notice that in {ATLAS:2015kla}, \mbox{HIJING} calculations reproduce rms $a_1$ for $N_{\textrm{part}} \lesssim 80$ but overpredict the signal at larger $N_{\textrm{part}}$.
This suggests that hydrodynamic calculations and microscopic models are complementary in understanding longitudinal fluctuations.

Averaging the likelihood function over an ensemble of posterior parameter sets for each model gives the model likelihood, from which the Bayes factor is calculated,
\begin{eqnarray}
K = \frac{\text{Relative-skewness model}}{\text{Absolute-skewness model}} = 2.5 \pm 0.2. 
\end{eqnarray}
This value is too close to unity to make a decisive statement regarding the preference of one model over the other {Jeffreys:1961}.
Indeed, the absolute-skewness model is slightly better in capturing the asymmetries in p+Pb collisions; while the relative-skewness model exhibits a larger curvature for rms $a_1$, closer to experiment.
This is not a surprise since these two models give effectively the same local entropy profile as shown in the previous subsection.

In summary, both models describe the p+Pb and Pb+Pb charged-particle pseudorapidity densities in all centrality bins to 10\% accuracy.
It also describes the rms $a_1$ from central to mid-central Pb+Pb collisions.
Both models fail to describe the rms $a_1$ in peripheral collisions which suggests that additional sources of fluctuation are needed in addition to nuclear thickness function fluctuations.
Relevant sources could include initial dynamical fluctuations such as string fragmentation, subnucleonic fluctuations, and finite-particle effects.

\begin{figure*}
  \includegraphics{posterior}
  \caption{Posterior distributions of the model parameters, listed in Table~\ref{tab:parameters}, for the relative-skewness (blue lower diagonal) and absolute-skewness (red upper diagonal) models. The diagonal panels are the marginal likelihood distributions of individual model parameters, while off-diagonal panels are joint distributions for pairs of model parameters.}
  \label{fig:posterior}
\end{figure*}

\subsection{Posterior distribution of model parameters}
Fig.~\ref{fig:posterior} presents the Bayesian posterior probability distributions for the relative- and absolute-skewness models (blue lower- and red upper-triangular matrices respectively).
Diagonal panels show the marginal posterior distribution of individual model parameters (all other parameters integrated out), while off-diagonal panels show the joint distribution for pairs of model parameters, reflecting their correlations.

The posterior distributions contain a wealth of information; here we summarize a few key observations:
\begin{itemize}[itemsep=0pt, leftmargin=2\parindent]
  \item Both models prefer the entropy deposition parameter $p$ close to $0$, consistent within the range of the prior distribution extracted from {Bernhard:2016tnd}.
  \item The p+p multiplicity fluctuation parameter is well constrained and distributed about $k=2.0$ for both models.
    These $k$ values are also consistent with the range of the previous estimates obtained from fits to p+p, p+Pb, and Pb+Pb multiplicity distributions at midrapidity {Moreland:2014oya}.
  \item The relative-skewness model prefers a larger nucleon width than the absolute-skewness model. 
  For future studies, one may also use more granular protons with subnucleonic structure instead of Gaussian protons.
  \item The calibrated relative-skewness model exhibits almost zero shift about the mean and large skewness, while the absolute-skewness model prefers a shift close to the center-of-mass rapidity and a moderate skewness. 
\end{itemize}
Superficially, it appears that the models prefer qualitatively different mechanisms for longitudinal entropy deposition; however, for realistic values of the nuclear thickness function in heavy-ion collisions, the behavior of the two calibrated models is nearly identical, despite the use of two possible skewness parametrizations, as is shown in Fig.~\ref{fig:post_dsdy}.
The lines and bands in Fig.~\ref{fig:post_dsdy} correspond to the mean and $1\sigma$ uncertainty of the calibrated models predictions. 
We vary the nuclear thickness functions $T_A$ and $T_B$ from $0.2~\text{fm}^{-2}$ to $2.6~\text{fm}^{-2}$. The maximum of the nuclear thickness function for a Pb nucleus in an optical Glauber model is about $2.2~\text{fm}^{-2}$. However, the event-by-event $T_A$ and $T_B$ may exceed this value in the presence of nucleonic fluctuations.
The calibrated relative- and absolute-model predictions agree within $1\sigma$ uncertainty band.
This observation has an important implication.
The two models adapt their parameters independently to describe the data and they coincide on one functional form of initial entropy deposition in terms of $T_A$ and $T_B$.
Therefore, with experimental inputs from the charged particle pseudorapidity densities and two-particle pseudorapidity correlations, a systematic model-to-data comparison can extract the form of the three-dimensional initial entropy distribution for relativistic heavy-ion collisions at the LHC.

\begin{figure}
  \includegraphics{post_dsdy}
  \caption{Applying parameters sampled from the posterior probability distribution to Eq.~\eqref{cgf} along with Eq.~\eqref{regulateEq}, this plot shows the resulting local entropy profile $ds/d\eta$ varying $T_A$ and $T_B$ from $0.2~\text{fm}^{-2}$ to $2.6~\text{fm}^{-2}$. The lines are the mean predictions and the bands denote $1\sigma$ model uncertainties.
  }
  \label{fig:post_dsdy}
\end{figure}


\subsection{Predictions for novel observables}
Both the relative- and absolute-skewness models provide comparable descriptions of multiplicity observables in p+Pb and Pb+Pb collisions which makes sense given that they predict effectively identical three-dimensional initial entropy profiles when calibrated to fit experimental data.
In this section, we proceed to investigate whether they can describe azimuthally sensitive observables such as flows, event-plane decorrelations and symmetric cumulants, and for completeness, we shall conduct the calculation with both models.
Here we use selected initial condition parameters around the peaks of the posterior distributions for each model (Table \ref{tab:chosen_parameters}) and perform viscous 3+1D hydrodynamic evolution with UrQMD as an afterburner.
These observables are a nontrivial test of the proposed model as they resolve azimuthal correlations which have not been included in the calibration process.

\begin{table}[t]
  \caption{Selected high-probability parameter sets}
    \begin{tabular}{lll}
      Parameter & rel-skew	& abs-skew \\
      \paddedhline
      $N_{\textrm{Pb+Pb}}^\dagger$   & 150.0     & 154.0  \\
      $p$	    & 0.0      & 0.0  \\
      $k$	    & 2.0     & 2.0  \\
      $w$	    & 0.59     & 0.42  \\
      $\mu_0$   & 0.0     & 0.75  \\
      $\sigma_0$ & 2.9    & 2.9  \\
   	  $\gamma_0$ & 7.3		& 1.0	\\
      $J$	     & 0.75 & 0.75	\\
    \end{tabular}
  \raggedright{$\dagger$ Normalization tuned with ideal hydro is reduced when using viscous hydro.}
  \label{tab:chosen_parameters}
\end{table}

\subsection{Anisotropic flows} 
As a preliminary check, we first verify that previous results for the elliptic and triangular flow harmonics $v_2\{2\}$ and $v_3\{2\}$ obtained using \trento\ initial conditions at midrapidity {Bernhard:2016tnd} are indeed recovered by the rapidity-dependent model extension.
Fig.~\ref{fig:vn_cen} shows the centrality dependence of $p_T$-integrated flow for ${0.2  < p_T < 5.0}$~GeV and $|\eta| < 0.8$ calculated from the \emph{three-dimensional} hybrid model compared to \mbox{ALICE} measurements {Adam:2016izf} using the  $Q$-cumulant method {Bilandzic:2010jr}.
The 3+1D hydrodynamics code used in this study only partially implements bulk viscous corrections and thus is not yet suitable for quantitative calculations involving finite bulk viscosity.
We therefore assert a QGP specific bulk viscosity $\zeta/s = 0$ which precludes direct comparison with the boost-invariant VISH2+1 hydrodynamics code {Song:2007ux, Shen:2014vra, Bernhard:2016tnd} and the corresponding shear and bulk viscosities determined by the previous Bayesian analysis {Bernhard:2016tnd}.
For the QGP specific shear viscosity, we choose constant QGP $\eta/s = 0.17$ and $0.19$ for relative- and absolute-skewness models respectively, which provide good descriptions of the data {Gale:2012rq, Niemi:2015qia}, although it is not a systematic best fit.
The resulting $v_2\{2\}$ and $v_3\{2\}$ agree with experimental data within $10\%$ and verify that the generating function rapidity extension recovers previous \trento\ initial condition results at midrapidity.

\begin{figure}
  \includegraphics{vn_cen}
  \caption{Elliptic and triangular flow cumulants $v_2\{2\}$ and $v_3\{2\}$ as a function of centrality calculated from 3+1D hybrid model simulations using constant specific shear viscosity $\eta/s=0.17$ and $0.19$ for relative- and absolute-skewness models respectively, zero bulk viscosity $\zeta/s=0$ and hydro-to-micro switching temperature $T_\text{sw}=154$~MeV.
  The initial condition parameters are selected from the Bayesian posterior.}
  \label{fig:vn_cen}
\end{figure}

We now proceed to calculate the pseudorapidity-dependent flows which provide a sensitive handle on the QGP transverse structure at different rapidity values.
The ALICE collaboration has measured $v_n\{2\}(\eta)$ and $v_2\{4\}(\eta)$ within the wide pseudorapidity interval $-3.5 < \eta < 5.0$ and extrapolated to zero $p_T$ {Adam:2016ows}. 
This extrapolation reduces integrated flow relative to measurements with a nonzero $p_T$ cut because it averages over low-$p_T$ particles which generally have less flow.
The same behavior occurs in hydrodynamic models, although models which mispredict mean $p_T$ also mispredict the corresponding change in flow produced by introducing a $p_T$ cut.
The hybrid model used in this study omits bulk viscous corrections and thus overpredicts mean $p_T$.
This means it cannot describe hydrodynamic flow measurements with different $p_T$ cuts using a single value of $\eta/s$.
To circumvent this issue, we use $\eta/s=0.25$--$0.28$ when comparing to ALICE measurements that are extrapolated to zero $p_T$. 
Future implementation of realistic bulk viscous corrections would eliminate such fine tuning.

The pseudorapidity-dependent flows are estimated using the cumulant approach {Bilandzic:2010jr}, where particles of interests (POI) are correlated with reference particles.
The differential flow is then calculated via,
\begin{eqnarray}
v_n^\prime\{2\} = \frac{d_n\{2\}}{\sqrt{c_n\{2\}}},\\
v_n^\prime\{4\} = \frac{-d_n\{4\}}{\left(-c_n\{4\}\right)^{3/4}},
\end{eqnarray}
where $d_n\{2\}$, $d_n\{4\}$ is the two- and four-particle cumulants between the POI and reference particles and $c_n\{2\}$ and $c_2\{4\}$ are the cumulants among reference particles.
For POI with $\eta > 0$ ($\eta < 0$), the reference particles are restricted to $-0.8 <\eta < 0$ ($0 <\eta < 0.8$) to avoid autocorrelations.
The results are shown in the left panel of Fig.~\ref{fig:vn_eta} for nine centrality classes.
The correlation functions $d_n(\eta)$ and $c_n(\eta)$ are symmetrized since the event-averaged pseudorapidity-differential flow for the Pb+Pb system should be invariant with respect to the substitution $\eta \rightarrow -\eta$.

Both models predict $v_2\{2\}$, $v_2\{4\}$ and $v_3\{2\}$ that decrease from mid to forward/backward rapidity and produce a triangle shaped structure as measured by ALICE.
Incidentally, the absolute-skewness model agrees with experiment slightly better at large pseudorapidity.
It has been realized that the slope of $v_n(\eta)$ which produces this triangular shape is highly sensitive to the hadronic shear viscosity {Denicol:2015bnf}, and thus Fig.~\ref{fig:vn_eta} corroborates that UrQMD provides a semi-quantitative description of hadronic viscosity below the QGP transition temperature.
However, for central to mid-central collisions, the slope of the decreasing $v_2$ as a function of pseudorapidity is underpredicted, resulting in a flatter $v_n(\eta)$ than the experiments.
The reason for this discrepancy may be complicated.
Apart from improving initial conditions, a realistic bulk viscosity and a temperature dependent specific shear viscosity should definitely affect the results.
Another reason could be the use of the QCD EoS and QGP transport coefficient $\eta/s$ in the limit of vanishing baryon chemical potential, which may not be a good approximation at large pseudorapidity even at LHC energies.

\begin{figure*}
  \includegraphics{vn_eta}
  \caption{Pseudorapidity dependence of anisotropic flow coefficients $v_2\{2\}$, $v_3\{2\}$ and $v_2\{4\}$ (blue circle, green triangle and orange diamond shaped symbols) calculated from the hybrid model with constant specific shear viscosity $\eta/s=0.25$ and $0.28$ for relative- and absolute-skewness models respectively (solid and open symbols), compared to data from ALICE (smaller black symbols) with $p_T > 0$~GeV (extrapolated) in different centrality bins. 
  The bands for each theory calculation point indicate $1\sigma$ statistical error, while experimental bands/bars denote $1\sigma$ systematic and statistical errors respectively.}
  \label{fig:vn_eta}
\end{figure*}



\begin{figure*}
  \begin{center}
  \includegraphics{evt_pln_decorr_near}
  \quad\quad\quad
  \includegraphics{evt_pln_decorr_far}
  \end{center}
  \caption{Left: The event-plane decorrelation for $n=2,3$ in different centrality bins with the reference particles from $3.0<|\eta^b|<4.0$.
  Right: The same quantities as the left panel but with the reference particles from $4.4<|\eta^b|<5.0$. 
  Theory bands indicate $1\sigma$ statistical error, while experimental bands/bars denote $1\sigma$ systematic and statistical errors respectively.}
  \label{fig:epd}
\end{figure*}


\subsection{Event-plane decorrelation}

Next, we study the event-plane decorrelation as a function of pseudorapidity using the calibrated relative-skewness model. 
The event planes are defined by the angles
\begin{equation}
  \Psi_n^\text{EP} = \frac{\text{atan2}(\langle \sin n \phi \rangle, \langle\cos n \phi \rangle)}{n},
\end{equation}
where the average is performed over particles of interest.
In general, the angles $\Psi_n^\text{EP}$ may change as a function of pseudorapidity due to longitudinal initial state fluctuations and finite particle effects.
As a consequence, two event-plane angles constructed from sets of particles separated by a finite rapidity gap will decorrelate as the rapidity gap increases.
This effect is important as it affects not only the calculation of soft observables involving a finite pseudorapidity gap or a large pseudorapidity interval, but also the interpretation of hard probe observables where particles from a rare hard process are often correlated with reference particles from different pseudorapidity bins.
It has been studied in a number of previous works including a longitudinally torqued fireball model with fluctuating sources {Bozek:2015bna}, AMPT calculations which studied its influence on flow observables {Jia:2014ysa, Xiao:2012uw}, as well as coarse-grained AMPT initial conditions that were embedded in 3+1D ideal hydrodynamic simulations {Pang:2015zrq}.

The decorrelations receive contributions from both random fluctuations during the evolution process and the systemic twist of the participant plane arising from initial longitudinal fluctuations {Bozek:2015bna}.
In the present parametric initial condition model, the participant plane twist arises naturally from local longitudinal fluctuations.
The transverse geometry at forward (backward) space-time rapidity is dominated by the projectile (target) participant density.
As a result, the participant plane gradually interpolates between the projectile and target densities, leading to a systemic twist in the beam direction.
The time evolution also contributes to decorrelation among the event-planes.
For example, early- and late-stage dynamics introduce additional fluctuations that partially randomize event-plane orientations.
Stochastic contributions from pre-equilibrium dynamics are neglected in the present study, but fluctuations in the hadronic phase are naturally accounted for by the UrQMD transport model.

The CMS collaboration has measured the event-plane decorrelations in Pb+Pb collisions using the $\eta$-dependent factorization ratio $r_n(\eta^a, \eta^b)$ {Khachatryan:2015oea}, defined as
\begin{align}
  r_n(\eta^a, \eta^b) &= \frac{V_{n\Delta}(-\eta^a, \eta^b)}{V_{n\Delta}(\eta^a, \eta^b)}, \\
  V_{n\Delta}(\eta^a, \eta^b) &= \langle\langle \cos(n\Delta\phi) \rangle\rangle,
\end{align}
where the double average means averaging over particles in each event and then averaging over all events in a given centrality class. 
The use of three $\eta$-bins ($\pm \eta^a$ and $\eta^b$) reduces short range correlations.
The ratio $r_n(\eta^a, \eta^b)$ reflects the fluctuation of event-plane angles separated by $\eta^a+\eta^b$ relative to the fluctuation of angles separated by  $|\eta^a-\eta^b|$ {Khachatryan:2015oea}.

We compare our calculation to the CMS measurements with both $3.0 < \eta^b < 4.0$ and $4.4 < \eta^b < 5.0$ and momentum cuts $p_T^b > 0$~GeV and ${0.3 < p_T^a < 3.0}$~GeV.
The $\eta$-dependent factorization ratios $r_2$ and $r_3$ for six centrality classes and different $\eta^b$ cuts are shown in Fig.~\ref{fig:epd}.
Both models predict a prominent $n=2, 3$ event-plane decorrelation in central collisions which decreases with increasing centrality.
For midcentral collisions, the nuclear geometry largely defines the $n=2$ participant plane---fluctuations and twisting are perturbations around this predominant direction---and hence $r=2$ decorrelation is reduced.
On the other hand, the $n=3$ event-plane receives little contribution from the nuclear geometry but is dominated mostly by fluctuations; it therefore has a similar slope over all six centralities.
In central collisions, the contribution from nuclear geometry is overwhelmed by fluctuations leading to similar $n=2$ and $n=3$ decorrelations.
The calculations describe the observed $n=2,3$ event-plane decorrelations with $3.0 < \eta^b < 4.0$ very well except the most central $0$--$5\%$ centrality, but systematically overpredict the magnitude of the decorrelations with $4.4 < \eta^b < 5.0$, especially for $0$--$10\%$ central collisions.
The reason is that the model, by construction, extends well-developed mid-rapidity initial conditions to finite pseudorapidity. 
Even though it is calibrated to multiplicity observables, it gradually loses its predictive power for fine-structure flow observables when moving far away from mid-rapidity.
Specifically, the model predicts decorrelations between the event-planes that are stronger for larger $\eta^b$ bins, while the experiment sees that the magnitude of decorrelation saturates when moving from $3.0<\eta^b<4.0$ to $4.4<\eta^b<5.0$.
Future improvements to the model at large pseudorapidity are clearly needed.
Nevertheless, the model's explanation of the event-plane decorrelations for $3.0 < \eta^b < 4.0$ remains nontrivial.
Both models were both calibrated with $\dnchdy$ and rms $a_1$ data.
These multiplicity observables do not constrain the transverse structure of the event at different pseudorapidities, and hence reproducing $r_2$ and $r_3$ means the calibrated initial condition models not only reproduce global longitudinal entropy deposition and fluctuations, but also capture features of the longitudinal dependence of transverse geometry within $|\eta| \lesssim 4$.


\begin{figure*}
  \includegraphics{smn}
  \caption{Top row: calculated $SC(4,2)$ and $SC(3,2)$ as functions of centrality compared to ALICE measurements, with the same 3D hybrid model set-up as used for Fig. \ref{fig:vn_cen}.
  We conduct calculations in two kinematic ranges: $|\eta|<0.8$ is the pseudorapidity cut used by current the ALICE measurement and $2.5<|\eta|<3.5$ is our prediction for the symmetric cumulants away from mid-rapidity in the Pb+Pb system.
  Bottom row: $SC(m,n)$ normalized by $\langle v_m^2\rangle\langle v_n^2\rangle$ for two two kinematic ranges.
  The left column and right column show results using relative- and absolute-skewness respectively.
   }
  \label{fig:smn} 
\end{figure*}


\subsection{Flow correlations}

Correlations between different anisotropic flow harmonics can be used to further constrain the initial state geometry {Niemi:2012aj}.
Experimentally, these correlations can be quantified using either event shape engineering {Schukraft:2012ah, Aad:2015lwa} or the symmetric cumulants $SC(m,n)$ {Bilandzic:2013kga}. 
Here we focus on the symmetric cumulants which are defined as,
\begin{align}
SC(m, n) &= \langle\langle \cos(m\phi_1+n\phi_2-m\phi_3-n\phi_4)\rangle\rangle \nonumber \\
\nonumber &- \langle\langle\cos[m(\phi_1-\phi_2)]\rangle\rangle\langle\langle\cos[n(\phi_1-\phi_2)]\rangle\rangle \label{eq:scmn}\\
&= \langle v_m^2 v_n^2 \rangle - \langle v_m^2\rangle\langle v_n^2\rangle.
\end{align}
The centrality dependences of $SC(4,2)$ and $ SC(3,2)$ at midrapidity have recently been measured by ALICE {ALICE:2016kpq}. 
A positive value of $SC(m,n)$ means that a large $v_m$ is more likely to be observed with a large $v_n$, while for negative values of $SC(m,n)$, a large $v_m$ favors small $v_n$.
The symmetric cumulants $SC(m,n)$ are nearly insensitive to nonflow effects while remaining sensitive to collective effects, initial geometry fluctuations $\langle \varepsilon_m^2 \varepsilon_n^2 \rangle - \langle \varepsilon_m^2 \rangle \langle \varepsilon_n^2 \rangle$ and the QGP specific shear viscosity {ALICE:2016kpq, Zhu:2016puf}.
To remove its dependence on the magnitudes of $\langle v_m^2\rangle$ and $\langle v_n^2\rangle$, we also calculate the normalized symmetric cumulants
\begin{equation}
  NSC(m,n) = SC(m,n)/\langle v_m^2\rangle\langle v_n^2\rangle.
\end{equation}
Here we use this tool to not only study the flow correlations at midrapidity, but also reveal its pseudorapidity dependence.

Fig.~\ref{fig:smn} shows the calculated symmetric cumulants compared to ALICE measurements using the relative- and absolute-skewness models with the same transport coefficients as in Fig.~\ref{fig:vn_cen}. 
We use the same centrality bins as ALICE experiments and the centrality averaged symmetric cumulants are performed with a multiplicity weight as discussed in {Gardim:2016nrr}.
We first calculate $SC(4,2)$ and $SC(3,2)$ at midrapidity $|\eta|<0.8$ (solid lines) to match the rapidity cuts of the ALICE measurement.
The negative $SC(3,2)$ is a result of initial eccentricity correlations, while the large positive $SC(4,2)$ is produced by nonlinear correlations between $v_2$ and $v_4$ during the medium evolution {Giacalone:2016afq, Qian:2016fpi, Bhalerao:2014xra, Zhou:2015eya}.
The resulting symmetric and normalized symmetric cumulants agree with the data quite well and support previous constraints on the QGP initial conditions at midrapidity {Bernhard:2016tnd}. 

Next, we shift our attention away from midrapidity and predict the symmetric (normalized symmetric) cumulants in the rapidity interval $2.5 < |\eta| < 3.5$ (dashed lines) which has not been measured.
In this calculation, we take two reference particles from $|\eta| < 0.8$ and two POI from $2.5 < |\eta| < 3.5$ and calculate
\begin{align}
SC^\prime(m, n) &= \langle\langle \cos(m\phi_1+n\phi_2-m\phi_3^\prime-n\phi_4^\prime)\rangle\rangle \\
\nonumber &- \langle\langle\cos[m(\phi_1-\phi_2^\prime)]\rangle\rangle\langle\langle\cos[n(\phi_1-\phi_2^\prime)]\rangle\rangle, \label{eq:scmn}
\end{align}
where the primed symbols represents the azimuthal angle of POI.
Magnitudes of both $SC^\prime(4, 2)$ and $SC^\prime(3, 2)$ are significantly suppressed at forward/backward rapidities, in accordance with the behavior of $v_2\{2\}(\eta)$ and $v_3\{2\}(\eta)$ as presented in the text in Fig.~\ref{fig:vn_eta}.
However the normalized symmetric cumulants $NSC^\prime(4,2)$ and $NSC^\prime(3,2)$ are consistent within uncertainty bands for different pseudorapidity cuts.
We observe that the normalized symmetric cumulant does not change as a function of psuedorapidity for either the relative- or absolute-skewness model and hence expect it to remain constant in nature as well.
Future comparison with available data should correspondingly impose strong constraints on our approach for modeling the three-dimensional initial conditions.


In summary, we have proposed a new method to extend arbitrary initial condition models defined at midrapidity to forward and backward pseudorapidity.
The method describes initial entropy deposition as a purely local function of nuclear participant densities, with the longitudinal profile reconstructed from generating-function cumulants.
The first three cumulants of the distribution (mean, standard deviation, and skewness) are included.
We set the mean proportional to the center-of-mass rapidity of local nuclear participant densities, and parametrize the standard deviation using a constant rapidity width.
Two models for the distribution's skewness are investigated: one where the skewness is proportional to the relative nuclear thickness difference, and one where it is proportional to the absolute difference.

We apply the method to extend the parametric \trento\ initial condition model which has been previously used to constrain QGP initial conditions and medium properties at midrapidity. 
The resulting three-dimensional models are then calibrated using Bayesian parameter estimation to fit p+Pb and Pb+Pb charged particle pseudorapidity densities $\dnchdy$ and the root-mean-square of the two particle pseudo-rapidity correlation's Legendre decomposition coefficient $a_1$ at the LHC.
After the calibration, both models provide comparable descriptions of experimental $\dnchdy$ and rms $a_1$ data.
Despite the apparent difference in the skewness ansatz, the calibrated models predict effectively the same behavior for local longitudinal entropy deposition as function of nuclear thickness in heavy-ion collisions.

Using the calibrated relative- and absolute-skewness initial condition models, we study pseudorapidity-dependent anisotropic flows, event-plane decorrelations and flow correlations in Pb+Pb collisions.
The model nicely describes integrated flows $v_2$ and $v_3$ at midrapidity as well as the pseudorapidity dependence of differential flow for different centrality classes.
The elliptic and triangular event-plane decorrelations with $3.0 < |\eta^b| < 4.0$ are well explained except for the most central collisions, but both models overpredict the decorrelations with the reference particles $4.4 < |\eta^b| < 5.0$.
This is because the model is an extension from mid-rapidity calculation and it gradually loses its accuracy at large forward/backward rapidity.
Both models give a satisfactory description of flow correlation $SC(3,2)$ and $SC(4,2)$ at midrapidity, which can be used to predict their values at forward/backward pseudorapidity where their values have not yet been measured.

The present work expands upon previous efforts to parametrize and constrain local initial condition properties using global final-state observables.
We show that these local properties are overconstrained by multiplicity observables alone and can be reverse engineered using systematic model-to-data comparison with quantitative uncertainty.
Specifically, it is a first attempt to use data-driven methods to infer what the entropy density distribution looks like at the hydrodynamic starting time in all three spatial dimensions.

It is clear that local forward/backward fluctuations are responsible for a variety of longitudinally sensitive phenomena beyond mere multiplicity fluctuations.
The general agreement of the present framework with pseudorapidity dependent flows and event-plane decorrelations corroborates the use of relativistic viscous hydrodynamics in describing the QGP dynamics away from midrapidity region.
The resulting knowledge can then be used to provide direct feedback for first-principle calculations of the QGP initial conditions, and can also be applied to studies where the QGP initial conditions act as a nuissance parameter, e.g.\ when modeling the propagation of hard probes through the medium in order to measure their response.

The present analysis would benefit from a number of future improvements.
For example, it would be interesting to add subnucleonic structure to the nuclear thickness functions in order to examine its effect on longitudinal rapidity fluctuations.
Also, in this work we assume that the multiplicity observables are insensitive to viscous effects and use ideal hydrodynamics in the model-to-data comparison process.
In the calculation of flow observables, we use an over-simplified constant specific shear viscosity and zero bulk viscosity, although there have been many works suggesting preference for a temperature dependent shear viscosity and finite bulk viscosity {Bernhard:2016tnd,Niemi:2015qia, Ryu:2015vwa}.
We leave these refinements to future work and hope the calibrated initial conditions presented in this study provide a more realistic description of the three-dimensional structure of relativistic heavy-ion collisions which will prove useful in constraining the properties of hot and dense QCD matter.
\chapter{In-medium transport model for hard partons}
\cite{ATLAS-Collaboration:2012iwa,Abelevetal:2014dna,STAR:upgrade-hf,Adare:2015kwa,CMS:2017dec}
\cite{Wang:1994fx,Zakharov:1996fv,Baier:1996sk,Zakharov:1997uu,Arnold:2002zm,Gyulassy:2003mc,Kovner:2003zj,Jeon:2003gi,CasalderreySolana:2007pr,Djordjevic:2008iz,Bass:2008rv,Schenke:2009gb,Majumder:2009zu,Majumder:2010qh,Armesto:2011ht,Zapp:2011ya,Ovanesyan:2011xy,Kang:2014xsa,Cao:2016gvr,Kauder:2018cdt,Cao:2017zih}.

Hard partons are predominately created in perturbative scatterings in the earliest stage of relativistic heavy-ion collision.
The distribution of the hard partons gets modified by its in-medium propagation and therefore whose final states carries information about the medium properties, as well as hard-soft interactions that are interested.

Among the many ways of describing the in-medium evolution of hard partons, 
transport approach has its unique advantage. 
Here, by transport approach, we are referring to a general class of transport model that does the time evolution of a semi-classical particle distribution function.
Transport models can be often formulated as simulation on the particle level, which provides easy coupling to local properties of a dynamically evolving and fluctuating medium, with exclusive information of the final states.

There are also challenges associated to the transport modeling in high energy collision.
First, there are different assumptions of the interactions between the hard parton and soft medium.
Two commonly assumed extremes are:
\begin{itemize}
\item[1] A weakly coupled picture: medium are consists of perturbative quasi-particles (scattering centers) whose distribution is close to local thermal equilibrium.
Hard partons scatters pertrubtively with these well separated scattering centers. The dynamics are described by a Boltzmann equation.
\item[2] Diffusion picture: interactions between the medium and the hard parton are frequent and soft, many body effects and non-perturbative effects can be important. The statistical effect of these interactions are modeled by a drag and a diffusion coefficient. The dynamics are solved using a Langevin equation.
\end{itemize}
These two commonly used approaches are not necessarily mutually exclusive, and can have different range of application. 
For example, the effect of a large part of soft momentum exchange processes in the perturbative calculation can be well modeled by a diffusion equation.
These drastically different assumptions on the interaction between the medium and the hard probe is largely due to our inadequate theoretical tools in describing the QGP medium in the strongly coupled regime.
On the one hand, this becomes the model uncertain intrinsic to transport approach, until one founds convincing way of interpreting the strongly coupled QGP medium from first principal.
On the other hand, this also becomes a chance to extract the properties of the medium through a systematic model-to-data comparison. 
An ultimate goal would be answering whether quasi-particle degrees of freedom is an effective description of sQGP using hard probes.

A second difficult associated to the semi-classical transport modeling is the treatment of quantum coherence effect.
Indeed, a quantum transition will always be bounded by the uncertainty principal: a process with momentum scale $Q$ can not be localized within a space (time) extend of $1/Q$.
While in the semi-classical transport model, one always specify a local point in space-time where the interaction takes place.
This is of course validate if the momentum scale $Q$ is high enough that $1/Q$ is much smaller than the resolution that the transport model concerns, e.g., characterized by the mean-free-path in Boltzmann equation.
However, soft and collinear divergence of QCD bremsstrahlung (or more generally, parton branching) processes generates an abundance of small-$Q$ events whose spatial extents can be much greater than the na\"ive expectation of mean-free-path. 
This happens for certain phase space of the vacuum parton shower as well as the medium-induced parton shower.
The vacuum parton shower are often solved as a vitality evolution with the  time variable integrated out, while the transport equation is an evolution in time with virtuality integrated out, we shall postpone the discussion to the next section.
This section focus on the medium induced part, where the QCD in-medium branching receives contribution from multiple scattering centers that leads to the QCD analog of the Landau-Pomeranchuk-Migdal (LPM) effect, and the radiation pattern is changed qualitatively.
When this happens, strictly speaking, the semi-classical transport equation is not the appropriate tool for these processes.
However, considering the advantages of the transport formulation, we would like to develop a minimum set of modification to the semi-classical transport that can mimic some quantum effect of in-medium parton branching.


As an overview of the section and our goals in developing the transport model,
We start with introduction of widely used transport equations: the (linearized ) Boltzmann equation and the Langevin equation.
Considering the elastic interaction first, we try to combine these two transport approach into a hybrid one by introducing a cut-off distinguishing hard and soft momentum transfer process.
After that, we introduce the theory related to QCD in-medium parton branching processes and discuss its various approximations and also the exact solutions.
With these theoretical insights, we developed a ``modified Boltzmann transport" approach in treating the in-medium parton branching approximately.
Finally, we compare the simulated results the transport model to the theoretical expectations to validate the implementation in different regime.
We will show that the modified transport approach can reasonable describes the longitudinal structure of the medium-induced splitting vertex for different channels $q\rightarrow q+g$, $g\rightarrow g+g$ and $g\rightarrow q+\bar{q}$.
Treatment of the heavy quark masses effect, and running coupling is investigated at the very end.

\cite{PhysRev.103.1811,Wang:1994fx,Zakharov:1996fv,Zakharov:1997uu,Baier:1996kr,Baier:1996sk}

\cite{Cao:2013ita,ColemanSmith:2012vr,Xu:2004mz,Zapp:2011ya,Gossiaux:2012cv,Park:thesis}.

\cite{Arnold:2002zm,Arnold:2008zu,Arnold:2009mr,Baier:1996kr,Baier:1998yf}. 

\section{The Boltzmann equation}
The Boltzmann equation evolves the transport of particles' distribution function under the effect of localized collision. 
By localization, it means that the time scale of the collision has to be much smaller compared to the mean free-path $\tau \ll \lambda$. 
Therefore, the collision probability can be evaluated using local particle distribution function.
Also the calculation of the few body collision processes is not significantly affected by the presence of the other particles, because the probability to interact with another particle during this collision is small, $P \approx \tau/\lambda \ll 1$.
At weak coupling, we will see in the next section that this is indeed the case for elastic collision, and soft and large radiation, but not high energy radiation where $\tau$ is actually much greater than the mean-free-path, and hence ``non-local".
Accordingly, the Boltzmann formulation and solvers needed to be modified to take into this non-local effect.
We postpone the related discussion later and shall assume in this subsection that the interactions are local. 

Considering two-body to two-body (elastic) and two-body to three-body (inelastic, including the reverse process) processes take the Boltzmann equation for particle specie $a$ takes the following form,
\begin{eqnarray}
\frac{\partial f^a}{\partial t} + \vec{v}\cdot\frac{\partial f^a}{\partial \vec{x}} + \frac{\partial E}{\partial \vec{x}}\cdot\frac{\partial f^{a}}{\partial \vec{p}} = - \sum_{b,c,d}\mathcal{C}_a^{a+b\leftrightarrow c+d}- \sum_{b,c,d,e}\mathcal{C}_a^{a+b\leftrightarrow c+d+e}
\end{eqnarray}
On the left hand side, the distribution function $f^{a}(t, \vec{x}, p)$ undergoes transport with velocity $\vec{v} = \partial E/\partial \vec{p}$, and a possible potential potential force $\vec{f} = -\partial E/\partial \vec{x}$.
On the right hand side of the equation, the $2\leftrightarrow 2$ and $2\leftrightarrow 3$ collision term are functions of the distribution functions, changing the distribution function through a gain term and a loss term in the one particle phase-space.
The summation of over $b,c,d,e$ iterates over all other particle species including $a$.
Using the elastic process as an example and neglecting degeneracy of the internal quantum number for simplicity, the collision term for a specific process is,
\begin{eqnarray}
\mathcal{C} &=& \mathcal{C}_\textrm{loss} + \mathcal{C}_\textrm{gain}\\
&=& \int f^a(p_1)f^b(p_2)[1+\epsilon^c f^c(p_3)][1+\epsilon^d f^d(p_4)] \overline{|M|^2}(p_1^a, p_2^b; p_3^c, p_4^d) d[PS]_{bcd} \\\nonumber
&-& \int f^c(p_3)f^d(p_4)[1+\epsilon^a f^a(p_1)][1+\epsilon^b f^b(p_2)] \overline{|M|^2}(p_3^c, p_4^d; p_1^a, p_2^b) d[PS]_{bcd} \\
&=& \int \left\{
f^a(p_1)f^b(p_2)[1+\epsilon^c f^c(p_3)][1+\epsilon^d f^d(p_4)] \right. \label{eq:collision-term:symmetry} \\\nonumber
&& \left.- f^c(p_3)f^d(p_4)[1+\epsilon^a f^a(p_1)][1+\epsilon^b f^b(p_2)]\right\}
\overline{|M|^2} d[PS]_{bcd} 
\end{eqnarray}
Where the crossing symmetry of the matrix-elements has been used in the last line of the equation ($\overline{|M|^2}(p_1^a, p_2^b; p_3^c, p_4^d) = \overline{|M|^2}(p_3^c, p_4^d; p_1^a, p_2^b) = \overline{|M|^2}$), and the phase-space integral is 
\begin{eqnarray}
d[\textrm{PS}]_{bcd} = \prod_{i\in {b,c,d}}\frac{dp_i^3}{2E_i (2\pi)^3} (2\pi)^4 \delta^{4}(p^a_1+p^b_2 - p^c_3-p^d_4).
\end{eqnarray}
The $epsilon = 0, -1, 1$ corresponds to classical, Fermi-Dirac, and Bose-Einstein statistics depending on the nature of the particle.

The first term in the integration represents the loss of type-$a$ particle in the phase-space around point $(x, p_1)$ due to elastic collisions, and the second term represents the gaining of type-$a$ due to the reverse processes.
The symmetry in the microscopic matrix-element is very important for the kinetic equation and ensure the detailed balance: the probability to transition from one microscopic state to anther equals its reverse process.
The detailed balance is also important for the reach of thermalization of the system. 
Assume the system has been evolved in a cell with fixed volume and there is no spatial variance of the distribution function.
Then using the form of the collision term in Eq. \ref{eq:collision-term:symmetry}, one can see that the static solution of the distribution function has to satisfy the relation,
\begin{eqnarray}
f^a f^b (1+\epsilon f^c) (1+\epsilon f^d) = f^c f^d (1+\epsilon f^a) (1+\epsilon f^b),
\end{eqnarray}
for the entire phase-space and every combination of particle species.
Therefore, the following combination is conserved for each reaction channel.
 \begin{eqnarray}
\frac{f^a}{(1+\epsilon^a f^a)} \frac{f^b}{(1+\epsilon^b f^b)}
= \frac{f^c}{(1+\epsilon^c f^c)} \frac{f^d}{(1+\epsilon^d f^d)}
\end{eqnarray}
The nature conservation one expect is the energy-momentum conservation, and therefore one solution to the previous equation is,
\begin{eqnarray}
\frac{f^a}{(1+\epsilon^a f^a)} = e^{-\beta \mu_a-\beta p\cdot u}
\end{eqnarray}
for every particle species with parameters $\mu, \beta$ and a four vector $u$ ($u^2 = 1$). 
So the static solution of the distribution is 
\begin{eqnarray}
f^a(p) = \frac{1}{ e^{\beta \mu_a+\beta p\cdot u} - \epsilon^a} \label{eq:thermal}\\
\mu_a +\mu_b = \mu_c + \mu_d \label{eq:chem}
\end{eqnarray}
The fist line is the thermal distribution as expected that characterizes the kinetic equilibrium of the system, and the second line is the requirement for reaching chemical equilibrium. 
And one can identify the $\beta$ and $\mu$ parameter as the inverse temperature and chemical potential. The $u$ vector is the flow velocity of the cell as can be seen from the average velocity,
\begin{eqnarray}
\left\langle \frac{p^\mu}{M} \right\rangle = \frac{\int f(p) \frac{p^\mu}{M} dp^3}{\int f(p) dp^3} = \frac{\int f(p) \frac{p^\mu}{M} dp^3}{\int f(p) dp^3} = u^\mu
\end{eqnarray}

\section{Linearization of the Boltzmann equation and the diffusion limit}
The full Boltzmann equation is a coupled differential-integral equation for all distribution function and has non-linearities. 
As a result, the analytic solutions for even simple form of interaction is almost impossible.
The numerical solution / simulation is also a highly non-trivial task.
However, under certain circumstances, a linearization of the Boltzmann equation is possible and greatly simplifies both analytic analysis as well as the numerical simulation.

The hard particles (jet partons, heavy flavors) are initial produced in perturbative processes with a large $p_T$ or a large mass $M \gg T$. 
The perturbative production cross-section drops fast with the increase of $p_T$ and $m_T$, so hard partons are very rare in actually event. 
Therefore the occupation number of hard parton is a small number $f_H \ll 1$.
One can neglect the quantum statistics terms in the Boltzmann equation for them $1+\epsilon f_H \approx 1$.
For the Boltzmann equation of hard partons, the collision terms with more than one uncorrelated hard particles in the initial state can also be neglected since these contributions is proportional to $f_H^2 \ll f_H$. 
Finally, we also assume that the response of the bulk of the particles to the hard particles is small, and shall neglect any collision terms that involve a hard parton in the Boltzmann equation for the bulk distribution function.
Under these approximations, one arrived at a set of equations that is linearized with respect to the hard partons:
\begin{eqnarray}
\frac{df_H}{dt} &=& -\mathcal{C}_H[f_H, f_{\textrm{bulk}}] \label{eq:hard-bulk-eq}\\
\frac{df_{\textrm{bulk}}}{dt} &=& -\mathcal{C}[f_{\textrm{bulk}}]
\end{eqnarray}
Here the collision term $\mathcal{C}_H$ is linear with respect to $f_H$.

For the bulk particles, one still have a complex set of coupled equation to solve. 
However, a great simplification can be made by observing that the time it takes for the bulk particles to reach local thermalization is much shorter than the hard particles relaxation time, so a zeroth order approximation would be using the local thermal distribution of Eq. \ref{eq:thermal}.
The space-time evolution of the temperature $T$, chemical potential $\mu$ and flow velocity $u$ can be obtained from a hydrodynamic solution that also assumes a close-to-local thermalization of the bulk medium.
Replacing the bulk medium distribution function by the thermal in Eq. \ref{eq:hard-bulk-eq}, one arrives at a closed (linearized) equation for the hard particles.
Here we write down the equation assuming both the classical statistics and the conservation of the hard parton's species, and only elastic collision term is shown for simplicity,
\begin{eqnarray}
\frac{\partial f^H}{\partial t} + \vec{v}\cdot\frac{\partial f^H}{\partial \vec{x}} &=& -\sum_{b} \int \left\{
f^H(p_1)f^b_{eq}(p_2) - f^H(p_3)f^b_{eq}(p_4)\right\}
\overline{|M|^2} d[\textrm{PS}]_{234} \\
&=& - \int \left\{
f^H(p_1) w(p_1; p_3) - f^H(p_3) w(p_3, p_1)\right\}\frac{dp_3^3}{2E_3 (2\pi)^3}
\end{eqnarray}
Where the $w(p; p')$ are the transition probability density for a particle with momentum $p$ into momentum state $p'$,
\begin{eqnarray}
w(p; p') = \sum_b\int f_{eq}^b(p_2) \overline{|M|^2}(p, p_2; p', p_4) d[\textrm{PS}]_{24}
\end{eqnarray}

This seems to be a great simplification to the original coupled equation. However, the use of locally equilibrium of the bulk particle distribution function is a strong assumption. 
The degree of local thermalization in realistic events is still an open question, especially at early stages. 
Even if the system is close to local thermal equilibrium, whether it can be understood in terms of the quasi-particle degrees of freedom that has been described using semi-classical distribution function is a different question.
In an extreme weakly coupled system $g\ll 1$, one expected the local pressure and energy density can be explained in terms of the fundamental degrees of freedom: quarks and gluons, with perturbative corrections.
But with a large $g$ estimated from phenomenology studies, such a perturbative description may not be most efficient way of understanding the bulk medium and non-perturbative physics can play an essential row. 
This interpretation of the medium in terms of microscopic degree-of-freedom seem to be a an unavoidable problem of the Boltzmann equations, however, it is possible to ``integrate out" the microscopic details of the medium if the interaction in the soft limit of interaction.
In the soft momentum transfer $q = p'-p$ limit  $|q| \ll |p|$, the change in distribution function can be expanded to the second terms in the $q$, and the linearized Boltzmann equation reduces to the Fokker-Planck type of equation,
\begin{eqnarray}
\frac{\partial f}{\partial t} + \vec{v}\cdot\frac{\partial f}{\partial \vec{x}} &=& - \int \left\{
f(p) - \left[f(p) +  \vec{q}\frac{\partial f}{\partial\vec{p}} + \frac{1}{2}\vec{q}\vec{q}\frac{\partial^2 f}{\partial\vec{p} \partial\vec{p}} \right]
\right\} w(p',p)\frac{dp_3^3}{2E_3 (2\pi)^3} \\
&=& - \int \left\{ \vec{q}\frac{\partial f}{\partial\vec{p}} - \frac{1}{2}\vec{q}\vec{q}\frac{\partial^2 f}{\partial\vec{p} \partial\vec{p}}
\right\} w(p',p)\frac{dp_3^3}{2E_3 (2\pi)^3} \\
&=&  -A(p) \frac{\partial f}{\partial\vec{p}} + \frac{1}{2}B(p)\frac{\partial^2 f}{\partial\vec{p} \partial\vec{p}}
\end{eqnarray}
The transport coefficient are defined as
\begin{eqnarray}
A(p) &=& \int w(p,p+q) q \frac{dp_3^3}{2E_3 (2\pi)^3}\\
B(p) &=& \int w(p,p+q) q q \frac{dp_3^3}{2E_3 (2\pi)^3}
\end{eqnarray}
which are the first and second moments of the transition probability.
One remark is that although the form of the Fokker-Planck equation can be derived as the diffusion (soft) limit of the linearized version of the Boltzmann equation, its range of applicability can be different.
This is because the transport coefficient is well defined in general regardless of whether one assumes quasi-particle type microscopic dynamics.
In fact, in our final model, we replace the soft sector of the Boltzmann equation with the Fokker-Planck equation so that the use of ``medium quasi-particles" are restricted to hard momentum transfer processes.
Moments beyond second order is dropped in deriving the Fokker-Planck equation, this is justified if the interaction is frequent enough so that within the smallest time scale that is concerned, there is already many interactions such that a statistical description of the effect of interaction is adequate in terms of the first (mean) and the second moments (standard deviation).
However, if the physical processes is rare within the considered time scale, when fluctuation characterized by higher moments of the transition probability is very important and a diffusion approximation will not be a good proxy in describing its effect.
For this reason, we will later show that the diffusion approximation  does not work for radiation with a large gluon energy $\omega \gg T$.

\section{The elastic process in the medium}
Elastic collision are usually refers to the processes where the number of hard partons are conserved.
In a quasi-particle picture of the QGP, the hard parton can collide with medium partons and transfer a certain amount of energy to the medium.
These processes can be calculated at leading order in the weakly coupled theory, where the collision cross-section is calculated using the dressed gluon propagator inside the medium,
\begin{eqnarray}
D^{\mu\nu}(\omega, k) = \frac{\delta^{\mu 0}\delta^{\nu 0}}{k^2 - \Pi_L(\omega, k)} + \frac{\hat{P}_T^{\mu\nu}}{\omega^2 - k^2 - \Pi_T(\omega, k)}
\end{eqnarray}
Due to the presence of the medium, the dressed propagator lost its Lorentz invariance and depends on complicated in-medium self energy $\Pi_T$ and $\Pi_L$.
And the resulting cross-section formula will be equally complicated.
Fortunately, it has been shown recently in [] that a simplification is possible at leading order in rewriting the elastic processes as large-angle scattering and small-angle diffusion.
In such an approach, a cut-off in momentum transfer $Q_\textrm{cut}$ is introduced, with the range formally satisfying $gT \ll Q_\textrm{cut} \ll T$.
For processes with momentum transfer to the medium larger than the cut-off  (hard-mode) the medium screening effect is neglected and vacuum matrix-element is used as a proxy.
While for processes smaller than the cut-off (soft-mode), the propagator receives significant contribution from the screen effect.
The soft processes happens frequent and only involves small momentum transfer; therefore, its statistical effect over a finite amount of time can be efficiently captured by the diffusion dynamics.
The resultant transverse and longitudinal diffusion constant has been calculated by [] to be,
\begin{eqnarray}
\hat{q}_L = g^2 C_R T m_\infty^2  \ln\left(1+\frac{Q_{\textrm{cut}}^2}{m_\infty^2}\right) \\
\hat{q} = g^2 C_R T m_D^2  \ln\left(1+\frac{Q_{\textrm{cut}}^2}{m_D^2}\right).
\end{eqnarray}
$m_\infty^2 = m_D^2/2$ is the asymptotic gluon thermal mass.

This separation allows the following modeling of the elastic interaction between hard parton and the medium,
\begin{eqnarray}
\frac{df}{dt} = \mathcal{D}[f] + \mathcal{C}^{2\leftrightarrow 2}[f].
\end{eqnarray}
So that the particle is continuously evolved by the diffusion process while undergoing occasionally large-angle scatterings.
Of course, we would like to verify that this separate cut-off dependence in the diffusion and scattering component indeed cancels for physical observations at sufficiently small coupling.

The theoretical assumption of a quasi-particle QGP and using leading order resummed propagator really replies on the weakness of the coupling constant $g$. 
While the phenomenological value of $g$ is in fact not small, and one may question whether the interaction between the hard parton and the medium receives a significant non-perturbative contribution.
Although the non-pertubative physics varies in different calculations, their dynamical effect are often modeled by a diffusion processes.

\section{Transport approach in the incoherent limit}
In this section, we describe the Monte Carlo model in detail. The partonic processes are categorized into elastic (particle number conserving) and inelastic processes (particle number non-conserving). 
The inelastic processes are further divided into parton-splitting and parton-jointing contribution. 
In this section, we shall first proceed using local and incoherent calculation of such processes and will discuss in great detail in the next section on including the LPM effect in a Boltzmann-like transport approach.

The elastic interaction between hard parton and medium is separate into a large-momentum transfer (hard) and a small-momentum transfer (soft) part.
The switching scale is chosen to be proportional to the Debye mass square $Q_{\textrm{cut}}^2 = c m_D^2$, where $c$ is a parameter.
Respectively, the inelastic processes are also separated into diffusion-induced slitting / jointing processes, and large-$Q$ matrix-element based inelastic scattering contribution.

The large-$Q$ collisions are solved by a linearized Boltzmann equation with using collision rates calculated with vacuum matrix-elements, while the soft interactions are approximated by a diffusion process and is solved by Langevin equations.
This combined Langevin and linearized Boltzmann dyanmics is solved on the particle level using a two step approach,
\begin{eqnarray}
\vec{x}(t+\Delta t) &=& \frac{\vec{p}}{E}\Delta t\\
\vec{p}_{\textrm{int}} &=& \vec{p} - \eta_D \vec{p} \Delta t + \vec{\xi}(t) \Delta t\\
\Delta t\frac{dR(T, \vec{v}, \vec{p}_{\textrm{int}})}{d\vec{p}^3} &\xrightarrow{\textrm{sampling}}& \vec{p}(t+\Delta t)
\end{eqnarray}
Where the particle first does free transport, and then its momentum be updated with diffusion dynamics to get an intermediate $\vec{p}_{\textrm{int}}$. 
Then, the particle under goes collision according to the reaction probability within $\Delta t$, and the final states are obtained by sampling the differential collision rate.
The diffusion dynamics consists of a thermal random force such that,
\begin{eqnarray}
\left\langle\xi_i(t)\xi_j(0)\right\rangle = \delta(t) \left(
\frac{p_i p_j}{p^2}\hat{q}_{S, L} + \left(
\delta_{ij}-\frac{p_i p_j}{p^2}
\right)\frac{\hat{q}_S}{2} 
\right)
\end{eqnarray}
Because we require that the diffusion dynamics only accounts for soft momentum transfer processes, its transport coefficient is obtained by integrating the leading order collision kernel upto the momentum $Q_{\textrm{cut}}$,
\begin{eqnarray}
\hat{q}_S = \int dq^2 \frac{\alpha_s m_D^2 T}{q^2 (q^2+m_D^2)} 
\label{eq:qS} \\
\hat{q}_{S,L} = \int dq^2 \frac{\alpha_s m_\infty^2 T}{q^2 (q^2+m_\infty^2)}
\label{eq:qSL} 
\end{eqnarray}
with the drag coefficient determined by the Einstein relation,
\begin{eqnarray}
\eta_D = \frac{\hat{q}_{S,L}}{2ET} - \frac{d\hat{q}_{S,L}}{dp^2} - \frac{2\hat{q}_{S,L} - 2\hat{q}_S}{2p^2}
\end{eqnarray}
For the large-$Q$ scattering processes, the collision rates for the $2\rightarrow 2$ and $2\rightarrow 3$ scatterings are obtained by integrating the vacuum matrix-element,
\begin{eqnarray}
R = \frac{d}{2E_1}\int  \frac{d^3p_2}{2E_2(2\pi)^3} f_0(p_2)2\hat{s} \int_{-\hat{s}}^{Q_{\textrm{cut}^2}}\frac{d\sigma}{d\hat{t}}d\hat{t}
\end{eqnarray}
The integration is restricted to large momentum transfer above $Q_{\textrm{cut}}$ and therefore we do not impose additional screening effect to regulate the matrix-element.
For the incoherent diffusion-induced splitting rate, we borrow the expression from \cite{Cao:2017hhk} stripping the time-dependent phase factor,
\begin{eqnarray}
R = \int d k_\perp^2 dx \frac{\alpha_s P(x) \hat{q}^S}{2\pi (k_\perp^2 + m_\infty^2)^2}
\end{eqnarray}
where a gluon thermal mass is added to screen the divergence.
For the reverse processes $3\rightarrow 2$  and $2\rightarrow 1$ processes, similar reaction rate can be written down.

The vacuum matrix-element is used for the large-$Q$ elastic and inelastic scatterings.
In this work, the $2\rightarrow 2$ matrix-element only includes the $\hat{t}$-channel contribution.
Regarding the $2\rightarrow 3$ matrix-element, in previous study \cite{Ke:2018tsh}, we used to employ an improved version of the original Gunion-Bertsch cross-section that works under the limits $k_\perp, q_\perp \ll \sqrt{s}$ and $x q_\perp \ll k_\perp$ \cite{PhysRevD.25.746,Fochler:2013epa,Uphoff:2014hza}.
In the present study, we keep improving the matrix-elements by following the derivation in \cite{Fochler:2013epa} while relaxing the condition $x q_\perp \ll k_\perp$.
Therefore the updated matrix-elements contain the correct vacuum splitting function in the collinear limit.
We summarize the matrix-elements here and have attached a derivation in the appendix,
\begin{eqnarray}
\overline{|M^2|}_{g+i\rightarrow g+g+i} &=& \overline{|M^2|}_{g+i\rightarrow g+i} P_{gg}^{g(0)}  D_{gg}^{g}\\
\overline{|M^2|}_{g+i\rightarrow q+\bar{q}+i} &=& \frac{C_F d_F}{C_A d_A}\overline{|M^2|}_{g+i\rightarrow g+i} P_{q\bar{q}}^{g(0)} D_{q\bar{q}}^{g}\\
\overline{|M^2|}_{q+i\rightarrow q+g+i} &=& \overline{|M^2|}_{q+i\rightarrow q+i} P_{qg}^{q(0)} D_{qg}^{q},
\end{eqnarray}
where $P_{bc}^{a(0)}(x)$ are vacuum splitting functions from parton $a$ to partons $b$ and $c$. Index $i$ represent a quark / anti-quark or a gluon.
The two body matrix-elements that enters the $2\rightarrow 3$ matrix-element is always required to be the $t$-channel contribution.
\begin{eqnarray}
P_{gg}^{g(0)}  &=& g^2  C_A\frac{1+x^4+(1-x)^4}{x(1-x)}\\
P_{qg}^{q(0)} &=& g^2  C_F\frac{1+(1-x)^4}{x}\\
P_{q\bar{q}}^{g(0)} &=& g^2  \frac{N_f}{2}\left(x^2+(1-x)^4\right)
\end{eqnarray}
The $D_{bc}^{a}$ contains the interference structure,
\begin{eqnarray}
D_{qq}^{g} &=& 
C_A(\vec{a}-\vec{b})^2 + C_A(\vec{a}-\vec{b})^2 \\\nonumber
&-& C_A (\vec{a}-\vec{b})\cdot (\vec{a}-\vec{c})
\\
D_{q\bar{q}}^{g} &=& 
C_F(\vec{a}-\vec{b})^2 + C_F(\vec{a}-\vec{b})^2 \\\nonumber
&-& (2C_F-C_A) (\vec{a}-\vec{b})\cdot (\vec{a}-\vec{b})
\\
D_{qg}^{q} &=& 
C_F(\vec{c}-\vec{a})^2 + C_F(\vec{c}-\vec{b})^2 \\\nonumber
&-& (2C_F-C_A) (\vec{c}-\vec{a})\cdot (\vec{c}-\vec{b})
\end{eqnarray}
with the vectors given by
\begin{eqnarray}
\vec{a} = \frac{\vec{k}_\perp - x\vec{q}_\perp}{(\vec{k}_\perp - x\vec{q}_\perp)^2};
\vec{b} = \frac{\vec{k}_\perp - \vec{q}_\perp}{(\vec{k}_\perp - \vec{q}_\perp)^2};
\vec{c} =  \frac{\vec{k}_\perp}{\vec{k}_\perp^2}.
\end{eqnarray}


Combining all these processes, we summarize our linearized-Boltzmann plus Langevin equation into,
\begin{eqnarray}
\frac{df}{dt} = \mathcal{D}[f] + \mathcal{C}_{1\leftrightarrow 2}[f] + \mathcal{C}_{2\leftrightarrow 2}[f] + \mathcal{C}_{2\leftrightarrow 3}[f].
\end{eqnarray}
The distribution function of the hard parton under goes soft diffusion and diffusion induced-radiation. 
Hard collision with the medium are included as $2\leftrightarrow 2$ and $2\leftrightarrow 3$ collision terms.
The next section devotes to the inclusion of LPM effect to such an incoherent transport equation.

\section{Monte Carlo solver of the linearized Boltzmann equation}
This subsection discuss the technical of solving the linearized Boltzmann equation using particle-based simulation.
This method starts from representing the distribution function by particle states. 
For linearized Boltzmann equation, it is sufficient to consider the dynamics of one such particle.
The particle undergoes free-transport in between subsequent collisions, and differential collision probability per unit time, the collision rate, is simply the collision integral, with the distribution function replaced by a delta-distribution $\delta^{3}(x-x'- v(t-t')) \delta^3(p-p')$.
For a two-body scattering, neglecting the quantum statistics, the collision rate in the rest frame of the medium is,
\begin{eqnarray}
R_a(p) = \sum_{b,c,d}\frac{1}{2E_a}\int \frac{dp_b^3}{(2\pi)^3 2E_b} f_0(p_b) \int d\Phi_m |M^2|_{ab\rightarrow cd}
\end{eqnarray}
Similar expression can also be obtained for $2\leftrightarrow 3$ processes.
In a short amount of time $\Delta t$, the probability to have no collision is $P_{0} = \exp(-\Delta t R)$.
The number of independent multiple collisions satisfy a Poisson distribution with mean $N = \Delta t R$. 
For a particle based simulation, one always need to control $\delta t$ small enough $(\Delta t \ll 1/R)$ so that effectively there is at most one collision happens within the time step.
Once a collision is sampled to happen, the full final states can be obtained by further sampling each scattering channel and the momentum phase-space differential rates.

The multi-dimensional phase-space sampling is performed sequentially for the initial state and final state phase-space.
For $2\rightarrow 2$ and $2\rightarrow 3$ body processes, we rewrite the integrated rate in the fluid cell rest frame as,
\begin{eqnarray}
R_{2m}(E_1, T) &=& \frac{d}{\nu} \frac{1}{2E_1}\int \frac{e^{-\beta E_2}dp_2^3}{(2\pi)^32E_2} 
\int d\Phi_m\overline{|M|^2}.
\end{eqnarray}
If vacuum matrix-element is used, the nested integration is a Lorentz invariant quantity, and we can choose to calculate it in the center-of-mass frame of the two-body collision, 
\begin{eqnarray}
\int d\Phi_m\overline{|M_{22}|^2} &=& 2E_12E_2v_{\textrm{rel}}\sigma \nonumber \\
 &=& 2(s-M^2)\sigma_{\textrm{CoM}}^{22}(\sqrt{s}, T)\nonumber \\
  &=& F_{2m}(\sqrt{s}, T)
\end{eqnarray}
where $\sigma$ is the cross-section of the process.
In practice, the values of the integrated rates and cross-sections are tabulated. 
The sampling of initial state $p_2$ determines the center-of-mass energy of the process $s = (p_1+p_2)^2 = 2(E_1 E_2 - p_1p_2 \cos\theta_{12})$.
Subsequently we sample the momentum-transfer $q$-differential cross-section with $\sqrt{s}, T$ as inputs, and final states are reconstructed given the initial state and $q$.
The sampling of $3\rightarrow 2$ body process is more difficult due to the larger dimensional of parameters to specify the initial state kinematics,
\begin{eqnarray}
R_{32}(E_1, T) = \frac{d}{\nu} \int \frac{e^{-\beta E_2}dp_2^3}{(2\pi)^32E_2} \frac{e^{-\beta k}dk^3}{(2\pi)^32k}
\int d\Phi_2\overline{|M|^2}.
\end{eqnarray}
The Lorentz invariant nested integral is a function of the initial 3-body state kinematics and temperature,
\begin{eqnarray}
\int d\Phi_2\overline{|M|^2} = F_{32}(\sqrt{s}, \sqrt{s_{12}}, \sqrt{s_{1k}}, T).
\end{eqnarray}
Where $s = (p_1+p_2+k)^2$ is the center of mass energy, $s_{12} = (p_1+p_2)^2$ and $s_{1k} = (p_1+k)^2$.
This requires four-dimensional table for the value of $F_{32}$ and a five-dimensional initial state sampling.
The tabulation of a high-dimensional rate and cross-section tables can be made manageable if a proper approximating function $A(x, y, \cdots)$ is proposed that captures the limiting behavior of the target function $T(x, y, \cdots)$.
Tabulating the ratio of $T/A$ would be extremely efficient and accurate with a moderate size of the table.

The sequential sampling breaks the original $m+n$ body phase-space sampling into two lower dimension sampling.
But as one should notice, the prerequisite is that the integration over the final state momentum of the matrix-element can be written into a Lorentiz invariant form and therefore only depends on the Mandelstam variables.
This appealing feature is certainly broken by the inclusion of either
quantum statistics or in-medium propagator in the matrix-elements. 
Because quantum statistics introduces factors like $1\pm f(p\cdot u)$ to the final momentum integral and the in-medium propagator is not Lorentz invariant, resulting in a $F_{nm}$ that depends on the relative velocity between the collision system and the medium rest frame.
This significantly increases the dimension and complexity of the problem. 
Fortunately, as we have shown in the last subsection, the screen effect can be neglected at leading order if we restrict the use of the matrix-element scatterings at large momentum transfer. 

\section{Parton-branching (inelastic) process and the Landau-Pomeranchuk-Migdal effect}
The inelastic process characterize the particle number changing processes in the Boltzmann equation.
In QCD, a direct calculation of such processes in the vacuum at the high energy limit is presented in ... .
One may expect that this is the leading order contribution to the inelastic channels of the transport equation.
However, a closer examination of the the matrix-element shows the problem of using this naive evaluation for in-medium transport.

A process takes a finite amount of time, which is the time it takes for its final states to loose coherence. 
Doing a Fourier transform for the matrix-element of the gluon radiation process and focus on the possible collinear divergence region in the phase space of the final states.
The light-cone energy difference between the initial and final states is 
\begin{eqnarray}
\delta E = \frac{k_\perp^2}{2k} + \frac{p_\perp^2}{2p} - \frac{{p'}_\perp^2}{2{p'}} = \frac{ [(1-x)\vec{k}_\perp - x\vec{p}_\perp]^2}{2x(1-x)E}
\end{eqnarray}
By the uncertainty principal, the coherence time for such transition on the order of $\tau_f = 1/\delta E$, termed the ``formation time" of the radiated gluon. 
Effects within the formation time should be calculated fully coherently.
However, the formation time can be sufficiently long for collinear branching that it is large compared to the elastic collision mean-free path $\lambda$.
which happens for small transverse momentum of the splitting that $k_\perp^2, p_\perp^2 < g^2x(1-x)ET$.
This means that the independent picture of collision induced radiations starts to breakdown, since the average number of collision within radiation becomes a relatively large number $N = \tau_f/\lambda >1$.
Zakarav and later by [BDMPS] showed that such multiple scatterings can be resummed for the calculation of medium induced radiation.
The probability of radiation,
\begin{eqnarray}
\nonumber
\frac{dP^{a}_{bc}}{d\omega} &=& \frac{\alpha_s P^{0,a}_{bc}(x)}{x^2(1-x)^2 E^2}\\
&&\mathfrak{Re}\int_0^\infty dt_1 \int_{t_1}^{\infty} dt_2 \nabla_{\mathbf{b}_1} \cdot\nabla_{\mathbf{b}_2} \left\{G(t_2, \mathbf{b}_2; t_1, \mathbf{b}_1) - G_0(t_2, \mathbf{b}_2; t_1, \mathbf{b}_1) \right\}|_{\mathbf{b}_1, \mathbf{b}_2 \rightarrow 0}
\end{eqnarray}
Where the $G$ is the propagator of the following Hamiltonian for the transverse dynamics of the splitting,
\begin{eqnarray}
\hat{H} &=& - \frac{\nabla^2_{\mathbf{b}}}{2x(1-x)E} - i \Gamma_3(\mathbf{b})\\
\Gamma_3(\mathbf{b}) &=& \frac{C_a-C_b+C_c}{2}\Gamma_2(\mathbf{b}) + \frac{C_a-C_c+C_b}{2}\Gamma_2(x\mathbf{b}) + \frac{C_b+C_c-C_a}{2}\Gamma_2((1-x)\mathbf{b})
\end{eqnarray}
and $G_0$ is the free propagator.
The interaction terms consists of the transverse broaden of the system $a\rightarrow b+c$. 
For a weakly coupled theory, this is
\begin{eqnarray}
\Gamma_2(b) = \frac{1}{\pi}\int \frac{d\mathbf{q}_\perp^2}{(2\pi)^2} \frac{g^2 T m_D^2 (1-e^{i\mathbf{b}\cdot\mathbf{q_\perp}})}{q_\perp^2(q_\perp^2+m_D^2)}
\end{eqnarray}
The term under the double time integral, which is denoted as $F(t_2, t_1)$ signature the quantum nature of this transition. 

There are two systematic ways to investigate the properties of the medium-induced radiation computation.
The first method is called the opacity expansion, which amounts to solve the propagator in a perturbation series of the number of interaction $\Gamma_3$. 
Another approach which investigate behavior of the solution assuming soft interaction (small-$q$) and approximate the collision kernel by a harmonic oscillator one $\Gamma_2(b) \approx \frac{1}{4}\hat{q}b^2$.
The is also known as the leading-log $1/\ln(N)$  approximation. 
Taking the residue potential $\Gamma(b) - \frac{1}{4}\hat{q}b^2$ as a perturbation, improvements at the next-to-leading-log level has also been investigated.

Though this leading order calculation has been written in this compact form, it is not trivial to include its effect (even approximately) in the semi-classical Boltzmann simulation.
We have made this point at the beginning of this section.
One can also see this problem more evidently by observing that it requires a finite time interval $t_1 \rightarrow t_2$ to compute the branching rate at time $t_1$, while the there is only one time variable in the Boltzmann equation. 
We will devote the next subsection to an approximation solution to this problem and modify the Boltzmann formulation accordingly on the particle level.
For the rest of this section, we shall elaborate the details of the current understanding in the opacity expansion and harmonic oscillator (deep-LPM) regime, which greatly facilitate the discussion of the section.

\subsection{Large medium}
For a large and static medium that approaches the infinite medium limit. 
Further simplification is possible that the calculation of the transition probability becomes a ``static" problem, and a branching rate $\Gamma$ can be defined as the branching probability per unit time.
This limit is known as the AMY formalisim \cite{Arnold:2002ja,Arnold:2002zm,Arnold:2003zc},
\begin{eqnarray}\label{eq:AMY-1}
\nonumber
\frac{d\Gamma_{a\rightarrow bc}}{dx} &=& \frac{1}{2E\nu_a} \frac{\alpha_s d_a P_{a\rightarrow bc}(x)}{x^2(1-x)^2}\int\frac{d^2\vec{k}}{(2\pi)^2}\vec{k}\cdot \mathfrak{Re} \vec{F}
\end{eqnarray}
where we have dropped the Bose enhancement and the Pauli blocking factors of the outgoing partons from the original formula.
The vector valued wave-function $\vec{F}(\vec{h}; p, x)$ satisfies the following integral equation,
\begin{eqnarray}\label{eq:AMY-2}
\nonumber
2\vec{k} &=& i\frac{\vec{F}(\vec{k})}{\tau_f(k)}  + g^2 \mathcal{C}[\vec{F}]
\end{eqnarray} 
$\vec{k}$, and $\tau_f(k)$ is the transverse scale and the formation time of the branching. $\mathcal{C}$ is still the $\Gamma_3$ operator in the momentum space representation,
\begin{eqnarray}
\mathcal{C}[f] &=& \int_{\bf q} \mathcal{A}(q_\perp^2)
\left\{  \frac{C_b+C_c-C_a}{2}\left(f_{\bf p}-f_{{\bf p}-{\bf q}}\right) \right.\\\nonumber
&& + \left. \frac{C_a+C_c-C_b}{2}\left(f_{\bf p}-f_{{\bf p}+x{\bf q}}\right) + \frac{C_a+C_b-C_c}{2}\left(f_{\bf p}-f_{{\bf p}+(1-x){\bf q}}\right)\right\}\\
\mathcal{A}(q_\perp^2) &=& \frac{g^2 m_D^2 T}{q^2\left(m_D^2+q^2\right)}
\end{eqnarray}
The exact solution has to be solved numerically.
Here we want to build the understanding of this formula in two extreme regime: the incoherent limit (Bethe-Heitler regime) and the deep-LPM regime.

{\bf The Bethe-Heitler regime}: The quantum interference can be neglected if the formation time is sufficiently short.
In such cases the amplitude under the double time integral has a delta-function like time structure, and the transition probability has a nice interpretation of integrating localized branching rate over a single time variable.
But the kinematic range for short enough formation time is very limited. 
The condition $\tau_f \ll \lambda$ translates to $\omega \ll T$.
For such case, one may solve for $F$ by treating $F/\tau_f$ as a large quantity.
The leading equations are then,
\begin{eqnarray}
2\vec{k}\tau_f(k) &=& - \mathfrak{Im} \vec{F} \\
\mathfrak{Re} \vec{F} &=& -g^2 \tau_f(k)\mathcal{C}[\mathfrak{Im} \vec{F}] 
\end{eqnarray}
Take quark splitting into a quark and a gluon as an example and neglecting the thermal masses, the resulting rate is then proportional to 
\begin{eqnarray}
R &\propto& 2g^2 \int d k^2 \int d q^2 \mathcal{A}(q_\perp^2) \left\{
\frac{C_A}{2} \frac{\vec{k}}{\vec{k}^2}\cdot\left[\frac{\vec{k}}{\vec{k}^2}-\frac{\vec{k}-\vec{q}}{(\vec{k}-\vec{q})^2}\right] \right.\\\nonumber
&&+\left. \frac{2C_F-C_A}{2} \frac{\vec{k}}{\vec{k}^2}\cdot\left[\frac{\vec{k}}{\vec{k}^2}-\frac{\vec{k}+x\vec{q}}{(\vec{k}+x\vec{q})^2}\right]
+\frac{C_A}{2} \frac{\vec{k}}{\vec{k}^2}\cdot\left[\frac{\vec{k}}{\vec{k}^2}-\frac{\vec{k}+(1-x)\vec{q}}{(\vec{k}+(1-x)\vec{q})^2}\right]
\right\}
\end{eqnarray}
though, this expression looks very different from the cross-section formula that we used in the incoherent rate in the Boltzmann equation, they are actually equivalent upon the integration of $dk^2$, we provide detailed explanation of this connection between the Bethe-Heitler approximation of the AMY rate equation and the incoherent rate computed with $2\rightarrow 3$ cross-section in appendix \ref{}.

In the high energy limit $E\gg T \gg \omega$ so that $x\ll 1$, this rate can be approximated by its $x\rightarrow 0$ limit, 
\begin{eqnarray}
\frac{dR}{dx} &=& \frac{4 E\alpha_s d_a P(x)}{\nu_a} \int \frac{dk^2}{(2\pi)^2} \int \frac{dq^2 \mathcal{A}(q_\perp^2)}{(2\pi)^2}  \sum_{\pm}
\frac{C_A}{2} \frac{\vec{k}}{\vec{k}^2}\cdot\left[\frac{\vec{k}}{\vec{k}^2}-\frac{\vec{k}\pm\vec{q}}{(\vec{k}\pm\vec{q})^2}\right] \\
&=& \frac{2 C_A E\alpha_s d_a P(x)}{\nu_a} \int \frac{dk^2}{(2\pi)^2} \int \frac{dq^2 \mathcal{A}(q_\perp^2)}{(2\pi)^2} 
\left[\frac{\vec{k}}{\vec{k}^2}-\frac{(\vec{k}-\vec{q})}{(\vec{k}-\vec{q})^2}\right]^2
\end{eqnarray}
where in the second step, the $k$ integration of the term with the ``$+$" sign has been shifted to an integration over $k-q$ to render the expression into the completing square form.
This form is known as the Gunion-Bertsch approximation of inelastic $2\rightarrow 3$ scattering, whose improved form has been employed in existing full Boltzmann simulation of partonic transport equation [].
To understand the physical meaning of the above expression, we can proceed to integrate and regulate soft divergence with a screening masses whenever needed.
Eventually we have,
\begin{eqnarray}
\frac{dR}{dx} \propto \alpha_s P(x) \times \alpha_s C_A T \propto \frac{\alpha_s P(x)}{\lambda_g}
\end{eqnarray}
Where the second factor $\alpha_s C_A T$ can be interpreted as the inverse of gluon-mean-free path. 
Now the physical meaning becomes clear: in the incoherent limit, certain amount of radiation $\alpha_s P(x)$ is triggered every mean-free-path from interaction with the collision centers.

Therefore, in the Bethe-Heitler regime, the total number of branching can be viewed as contributed by a incoherent sum $2\rightarrow 3$ processes localized at time $t$.
And such contribution can be easily incorporated into the Boltzmann equation given the incoherent branching rate.
However, the valid range for this approximation is at best $x E \sim$ a few times of the temperature.


{\bf The deep-LPM region (leading behavior)}: another useful approximation considers the limit $\tau_f$ is so large that many collisions contribute coherently to the branching. 
This corresponds to those energetic split $\omega \gg T$.
As a result, the transverse momentum $k$ of the branching should be large compared to the average momentum transfer to each scattering centers $q$.
In this limit, a diffusion approximation to the $\mathcal{C}$ operators is possible.
The finite difference between $F(k)$ and $F(k+O(q))$ is expanded in $\vec{q}$. 
The zeroth order cancels and the first order $\vec{q}$ contribution  vanishes due to the symmetric $q$ integration.
Keeping only second order terms in $\vec{q}$, the AMY equation is simplified to a diffusion type equation but with complex diffusion constant and a source term
\begin{eqnarray}
- \frac{1}{4} \nabla^2_{\vec{k}}\vec{F} + i\frac{k^2 + m^2_{\textrm{eff}}}{2x(1-x)E\hat{q}_3}\vec{F} = \frac{2\vec{k}}{\hat{q}_3}.
\end{eqnarray}
This approximation of the original collision operator is also known as the harmonic oscillator approximation, because in the impact-parameter space, this approximation leads to a quadratic form of $\Gamma_2 \propto b^2$.
Finally, $\hat{q}_3$ is the effective transport parameter for conveniences 
\begin{eqnarray}
\hat{q}_3(x, Q_0^2) &=& \alpha_s T m_D^2 \ln\left(1+\frac{Q_0^2}{m_D^2}\right) C_{abc}(x).\label{eq:qhat3}
\end{eqnarray}
This is obtained by doing the $\mathbf{q}$ integration of the expanded collision operator upto a cut-off scale $Q_0$, below which the small-$q$ approximation is considered to be valid.
The effective transport parameter also depends on the color structure of the splitting,
\begin{eqnarray}
C_{abc}(x) &=&  \frac{C_b+C_c-C_a}{2} + x^2 \frac{C_a+C_c-C_b}{2} +(1-x)^2\frac{C_a+C_b-C_c}{2}
\end{eqnarray}
Taking the momentum fraction of the ``$b$" particle to zero $x\rightarrow 0$, this color factor goes to $C_b$; similarly, $x\rightarrow 1$ which corresponds to ``$c$" particle taking vanishing fraction of the total energy, the color factor approaches $C_c$.
Therefore, in these extreme limits $x\rightarrow 0$ or $1$, the effect transport parameters looks like the daughter with softer momentum.
Of course, with finite $x, 1-x$, the color factor becomes a combination of the colors of the whole splitting system.

Neglecting the thermal mass, the solution to the this diffusion equation can be obtained analytically,
\begin{eqnarray}
\vec{F} = i 4x(1-x)E^2 \frac{\vec{k}}{k^2} \left[\exp\left(\frac{-i^{1/2}k^2}{\sqrt{2x(1-x)E\hat{q}_3}}\right)-1\right]
\end{eqnarray}
And the radiation rate can be obtained accordingly \cite{Arnold:2008zu},
\begin{eqnarray}\label{eq:AMY-LL}
\frac{dR_{bc}^{a,\textrm{LL}}}{dx} &=& \frac{\alpha_s P_{bc}^{a(0)}}{\pi\sqrt{2}}
\sqrt{\frac{\hat{q}_3(x, Q_0^2)}{2x(1-x)E}} \propto \frac{\alpha_s P(x)}{\langle \tau_f \rangle}
\end{eqnarray}
Such a result is often referred to as the leading log solution, because it is an expansion in terms of $1/\ln(N)$ with N the number of coherent collision centers.
An interesting scale $\sqrt{2x(1-x)E\hat{q}_3}$ shows up in such calculation which governs the typical transverse momentum of the splitting, or equivalently $\sqrt{\hat{q}_3/2x(1-x)E}$ which governs the rate at which the splitting happens.
A simple interpretation for these scales are obtained by considering the diffusion nature of this approximation.
The splitting parton has a formation time $\tau_f \sim 2x(1-x)/k^2$, within which many soft interactions contribute to the broadening of $k^2$.
From the change of variance in the diffusion dynamics, the average of $k^2$ is linearly proportional to the diffusion constant and time, $\langle k^2\rangle \sim \hat{q}_3\tau_f$.
Combining with the expression of formation time, one arrives at the above typical transverse momentum and typical formation time.

Compared to the na\"ive ``incoherent expectation", the actual radiation rate is reduced by a factor of $\lambda/\langle \tau_f \rangle$ on average. 
Therefore, in the deep-LPM regime, instead of triggering radiation every mean-free-path, a large collision centers contribute coherently and trigger radiation every ``formation time".
The average formation time scales as $\lambda \sqrt{\omega/T}$.
Considering that this approximation only works for $\omega \gg T$, we combine this result with the Bethe-Heitler regime and summarize the radiation pattern in a large medium as,
\begin{eqnarray}
\frac{dR}{dx} \sim \frac{\alpha_s P(x)}{\max\{\lambda, \tau_f\}}
\end{eqnarray}
This simple idea will be the foundation for developing the transport modeling of the parton branching processes in the section section.

{\bf The deep-LPM region (next-to-leading-log level)}:
The previously introduced leading (log) result has both a charming simplicity and a clear physical interpretation in explaining what happens in the deep-LPM region $\ln(N) \sim \ln(xE/T) \gg 1$.
Together with the Bethe-Heitler (incoherent) limit at $xE \lesssim T$, one can already build a pretty good understanding of medium-induced radiation in an infinite medium.

One thing that we omit to a detailed discussion is the upper bound $Q_0$ introduced to the $q$-integration in the leading-log approximation.
This cut-off scale, as a result of the small-$q$ simplification of the full model, is generally unknown and brings an uncertainty in this approximation. 
To get insights from these approximation on how to build a transport model proxy of the underlying theory, one need such an uncertainty under better control.
This issue has beend studied at the next-to-leading-log level by treating the large-$q$ part of the collision kernel as a perturbation to this approximation, authors of [] and more recently in [] have found that reasonable choice of $Q_0$ to be the order of $k_\perp^2$ itself.
A self-consistent determination of $Q_0$ is also possible by requiring a minimal contribution from the NLL correction.
The resultant results, takes a similar structure of the leading-log solution

And the NLL result is the LL solution replacing the unknown $Q_0$ with $Q_{1}$,
\begin{eqnarray}
Q_1^2  \approx \sqrt{\omega \hat{q}} \approx \sqrt{\omega \alpha_s C_R m_D^2 T \ln\frac{Q_0^2}{m_D^2}}
\label{eq:Q1}
\end{eqnarray}
or a self-consistent determination as in \cite{Arnold:2008zu},
\begin{eqnarray}
Q_1^2 &=& \sqrt{2 x (1-x) E \alpha_s T m_D^2}\\\nonumber
&\times & \left(
\frac{C_b+C_c-C_a}{2}\ln\frac{2\xi Q_1^2}{m_D^2} + \frac{C_a+C_c-C_b}{2} x^2 \ln\frac{2\xi Q_1^2}{x^2 m_D^2} \right.\\\nonumber 
&+& \left.\frac{C_a+C_b-C_c}{2} (1-x)^2 \ln\frac{2\xi Q_1^2}{(1-x)^2 m_D^2} \right)^{1/2}
\label{eq:Q1-sf}
\end{eqnarray}
with $\xi \approx 9.1$ a constant. 
This suggests that the optimal choice of the scale is on the the branching transverse momentum itself $\langle k^2 \rangle = \sqrt{2x(1-x)E\hat{q}_3}$, but with an improved logarithmic factor.
It has been shown that using this self-consistent $Q_1$ brings the approximation very close to the numerical solution of the full model in the deep LPM region.
In the next section, we shall make use of this NLL solution to validate the performance of the transport simulation.

\subsection{Thin medium: opacity expansion}
The realistic medium created in a heavy-ion collision are always finite and expanding.
Whether the medium is large enough for a certain branching should be determined by the comparison of the formation time and the size of the medium (or the expansion time scale).
For thin and dilute medium, there is only a few effective collision that contributes.
In such cases, systematic expansion over $L/\lambda$ has been developed can is known as the opacity expansion.
This can be obtained by solving the propagator with an perturbation series of the interaction potential $\Gamma_3$.
At leading order in the opacity expansion and apply soft approximation $x\ll 1$, the radiation rate is,
\begin{eqnarray}
\frac{dR}{dx} \propto g^2 P(x) C_A \int \frac{d\vec{q}^2 d\vec{k}}{(2\pi)^4} \frac{g^2 T m_D^2}{q^2(q^2 + m_D^2)} \frac{\vec{k}\cdot\vec{q}}{(\vec{k}-\vec{q})^2 k^2} \left[1-\cos\left(\frac{(\vec{k}-\vec{q})^2 t}{2x E}\right)\right]
\end{eqnarray}
It has a notable time-dependent $1-\cos(\omega t)$ modulation due to interference.
Therefore, there is an important finite-size effect for radiation spectrum in a thin medium, and so is the associated energy loss of the leading parton.
The finite size effect / path-length dependent rate is important for phenomenological study.

\subsection{Numerical solution}
To go beyond the above approximation and investigate how different limiting regimes are connected, one has to employ numerical approaches to study the exact propagation. 
At first sight, it looks the Green's-function can be easily solved in the impact-parameter space using numerical method developed for time-dependent  Schr\"odinger equation.
However, there are several difficulties in doing so.
First, this is a time consuming 2+1 dimension problem.
Second, the physical results is a finite difference of two Green's functions that are singular as $t_1 \rightarrow t_2$, so it can be sensitive to numerical regulation.
And third, the result is sensitive to the specify representation of the $\delta$-function initial condition.
For these reasons, we choose to employ the approach described in arXiv:1006.2379 to solve the problem in the momentum space.

Here is our detailed numerical implementation following the appendix of arXiv:1006.2379.
Going to momentum space, the cancellation between the in-medium propagator against the vacuum propagator is done automatically, upto difference related to the thermal masses and the induced radiation rate is,
\begin{eqnarray}
\frac{dR^{a}_{bc}(t)}{dx} = \frac{P^{a(0)}_{bc}(x)}{2\pi x(1-x)E} \mathfrak{Re} \int \frac{d\mathbf{p}^2}{(2\pi)^2} \int_0^t d\tau e^{\frac{-i\tau}{\tau_f(p)}} \vec{p}\cdot \vec{\Psi}(\vec{p}, \tau)
\end{eqnarray}
with the time evolution of the vector-valued wave function solved in the interaction picture with the initial condition,
\begin{eqnarray}
\frac{\partial \vec{\Psi}}{\partial \tau} &=& - e^{\frac{-i\tau}{\tau_f(p)}} \hat{C}\left[e^{\frac{i\tau}{\tau_f(p)}}\vec{\psi}(\vec{p}, \tau)\right]\\
\vec{\Psi}(\vec{p}, \tau=0) &=& \hat{C}\left[\frac{i\vec{p}}{p^2+m^2_{\textrm{eff}}}\right]
\end{eqnarray}
The $\hat{C}$ operation involves finite difference and two-dimensional integration over the transverse momentum $p$. 
Fortunately the integration over the azimuths angle of $p$ can be performed analytically at least for the leading order collision kernel with fixed coupling constant.
Reparametrize the vector function using the azimuthal symmetry argument $\vec{\Psi} = \vec{p}/(p^2+m^2_{\textrm{eff}})^2 \Phi(p^2)$.
The evolution equation for the scalar function $\Phi$ is,
\begin{eqnarray}
&&\frac{\partial \Phi(p^2)}{\partial \tau} = - \alpha_s T \sum_n c_n \int dq^2 \left\{\frac{\Phi(p^2)}{|p^2-q^2|} - \frac{\Phi(p^2)}{\sqrt{(p^2+q^2+m_n^2)^2 - 4p^2q^2}}\right. \\\nonumber
&&\left.- \exp\left(i\tau\frac{p^2-q^2}{2x(1-x)E}\right)\frac{\Phi(q^2)}{2p^2}\frac{(p^2+m^2_{\textrm{eff}})^2}{(q^2+m^2_{\textrm{eff}})^2} \left[\frac{p^2+q^2}{|p^2-q^2|} - \frac{p^2+q^2+m_n^2}{\sqrt{(p^2+q^2+m_n^2)^2 - 4p^2q^2}}\right]\right\}\\
&&\Phi(\tau=0)= i\sum_n c_n \int dq^2 \left\{\frac{p^2+m^2_{\textrm{eff}}}{|p^2-q^2|} - \frac{p^2+m^2_{\textrm{eff}}}{\sqrt{(p^2+q^2+m_n^2)^2 - 4p^2q^2}}\right. \\\nonumber
&&\left.-\frac{(p^2+m^2_{\textrm{eff}})^2}{2p^2(q^2+m^2_{\textrm{eff}})} \left[\frac{p^2+q^2}{|p^2-q^2|} - \frac{p^2+q^2+m_n^2}{\sqrt{(p^2+q^2+m_n^2)^2 - 4p^2q^2}}\right]\right\}
\end{eqnarray}
Here, the summation goes over the different pieces of three-body collision kernel, where the original integration variable $\vec{q}$ has been shifted to $\vec{p}-\vec{q}$, $\vec{p}+x\vec{q}$, and $\vec{p}+(1-x) \vec{q}$ accordingly.
The color factors are $c_1 = (C_b+C_c-C_a)/2$, $c_2 = x^2(C_a+C_c-C_b)/2$, $c_3 = (1-x)^2(C_a+C_b-C_c)/2$, and screening masses $m_1^2 = m_D^2$, $m_2^2 = x^2 m_D^2$, $m_3^2 = (1-x)^2 m_D^2$.
The azimuthal integration over $\phi_q$ has been performed analytically, and the resultant operator only involves an integration over $q^2$ from zero to infinity.
One may notice that one of the denominator $|q^2-p^2|$ can vanish.
But as $q^2$ tends to $p^2$, the subtracted term in the second line also approaches the expression in the first line, and therefore leaves the function to be integrated finite.

Now, the problem has been reduced to an initial value problem of a 1+1 D first order differential-integral equation, and can be solved quite efficiently using finite difference and numerical quadrature methods.


\subsection{Mass effect in medium-induced branching}
For radiation in the vacuum, the heavy quark mass is a natural regulator for the collinear divergence,
\begin{eqnarray}
\frac{dP}{dx dk_\perp^2} = \frac{\alpha_s P(x) k_\perp^2}{(k_\perp^2 + x^2 M^2)^2} = \frac{\alpha_s P(x)}{k_\perp^2}\left(\frac{\theta^2}{\theta^2 + \theta_M^2}\right)^2
\end{eqnarray}
Compared to light quark, the radiation off a heavy quark is suppressed within a typical angle $\theta_M = M/E$.
This is often referred to as the ``dead-cone" (mass) effect.

Inside a medium, the situation is more complicated. 
Because mass not only changes the propagator, but also shorten the formation time
\begin{eqnarray}
\tau_f = \frac{2x(1-x)E}{k_\perp^2 + m_{\textrm{eff}}^{'2}}, m_{\textrm{eff}}^{'2} = (1-x)m_\infty^2 + x^2 M^2
\end{eqnarray}
These two competing feature together contribution to the mass correction.
In principle, the above formula allows for the inclusion of heavy flavor, as long as $M\ll p$ still holds.

In the region where $p/M < 1/g$, the radiation process for heavy quark becomes sub-dominant compared to the elastic collisions.

\section{Adapting transport approach for medium-induced parton branchings in the deep-LPM region}
The LPM effects comes from that the parton splitting is not instantaneous.
In this section, we shall investigate the possible ways to approximate this effect by different types of modification on top of the particle based Boltzmann transport approach
We have tried three different approaches. 
We first introduce the most promising one, termed ``a modified Boltzmann transport", which is also the approach that is eventually adopted by the LIDO transport model.
Though we choose not to use the other two approaches, there were certain literatures that implemented similar ideas.
Therefore, we also briefly describe their basic ideas and comment on the limitations and potential problems.
After that, we compare the simulation of the ``a modified Boltzmann transport" to theoretical calculations introduced in the previous sections for different parton energy, coupling constants, and medium temperatures that are relevant for phenomenology applications.
Such a model validation is important as it tells whether the model is a good proxy of the underlying theory and quantifies the theoretical uncertainty when applying the model to phenomenology study and transport parameter extraction.

The approach is designed to work in a large medium and interpolates the deep-LPM region and the Bethe-Heitler region.
However, we also check its prediction for finite and expanding medium in the end.
The inclusion of running coupling effect and mass effect for the study of heavy-flavor is discussed at the end.

\subsection{A modified Boltzmann transport approach}
To see how to approximate the branching using a modification to the Boltzmann equation, we first go back to the branching probability formula introduced previously,
\begin{eqnarray}
\frac{dP^{a}_{bc}}{dx} &=& \int_0^\infty dt \frac{g^2 P(x)}{2\pi x (1-x) } \int_t^\infty dt'  F(t', t)
\label{eq:full-theory}
\\
F(t', t) &=& \mathfrak{Re} \int \frac{dp^2}{(2\pi)^2} e^{-it'/\tau_f} \vec{p}\cdot \Psi(\vec{p}, t')
\end{eqnarray}
Where we have abbreviate the transition amplitude by the notation $F(t', t)$ for convenience, but always keep in mind this is a quantity that depends on $E, \omega, g$ and complete medium properties along the path of the parton, including temperature and fluid velocity profiles $T(t), u^\mu(t)$.
The obstacle towards the Boltzmann formulation is the double time integral that signature the quantum transition, which also suggests a fundamental change to the semi-classical approach.

If the time-dependent wave-function is expanded in a series of the collision operator, then this transition amplitude includes the superposition of arbitrary number of multiple-interaction with the medium.
One realizes that there are also multiple scatterings of the branching partons in the Boltzmann simulation.
The difference is that the Boltzmann multiple-scatterings are independent from the branching processes; therefore, they only broaden the relative transverse momentum between the two daughter partons, but do not change the branching probability.
While what is expected from the leading-log approximation is that not only the momentum gets broadened, but also the branching probability gets reduced by a $\sim \lambda/\tau_f$ due to multiple scatterings.
Our modified approach follows this simple observation, and here is an overview.
\begin{itemize}
\item[1.] Assume an incoherent branching processes is generated at $t=t_0$. Do not treat the daughter partons as independent immediately.
\item[2.] Both mother and daughter partons receives elastic brodening from interacting with the medium, which also changes the formation time of the branching.
\item[3.] Evolve the branching system until $t-t_0 > \tau_f$. Then, reject this branching process with acceptance probability that is proportional to $\lambda/\tau_f$, which corrects for the fact that these multiple scatterings should contribute coherently.
\item[4.] Branching partons for those accepted processes are treated as independent objects from this point, rejected partons are
\end{itemize}

Now we shall explain it in detail.
Formally, this method can be understood as replacing the transition amplitude $F(t',t)$ using the following ansatz,
\begin{eqnarray}
F(t', t) \rightarrow \frac{1}{N}\sum_{i=1}^N \frac{b}{\tau_i(t)} \delta(t-t'- a \tau_i(t))
\end{eqnarray}
Here the function $F(t, t')$ is approximated by an ensemble ($N$) of branching systems.
Where each copy ``$i$" that evolves under the incoherent Boltzman dyanmics, will have a certainty probability to have a certain $\tau_i$ at time $t$.
The delta function expresses that the branching that start at time $t'$  is thought to be formed at time $t+a\tau_f$.
The additional factor $b/\tau_f$ corrects for that the probability for a branching to happen is different from the incoherent expectation.
This ansatz of representing a two-point function using information at $t$ and $t'$ of an ensemble of particles follows the same spirit in the Monte-Carlo solver of the traditional Boltzmann equation, where the distribution function is represented by an ensemble of particles at time $t$, $n(t, p, x) \approx \frac{1}{N}\sum_i \delta(x-x_i-v_i(t)) \delta(p-p_i)$.
This is indeed a crude proxy of the actually two-point function, and its validity has to be examined by comparing its prediction with the theoretical calculations.
Finally, $a$ and $b$ are dimensionless factors whose forms shall be determined later and will be tuned to achieve an optimal level of agreement to theoretical calculations.

Plug this simplified ansatz for $F(t, t')$ into the branching probability,
\begin{eqnarray}
\frac{dP^{a}_{bc}}{dx} &=& \frac{1}{N}\sum_i \int_0^\infty dt \frac{g^2 P(x)}{2\pi x (1-x)} \frac{b}{\tau_i} \\  
 &=& \frac{1}{N}\sum_i \int_0^\infty dt \frac{g^2 P(x)}{2\pi x (1-x) \tilde{\lambda}} \frac{b \tilde{\lambda}}{\tau_i(t=t'+\tau_i)}.\\
  &=& \frac{1}{N}\sum_i \int_0^\infty dt \frac{dR_{\textrm{incoh}}}{dx} \frac{b \tilde{\lambda}}{\tau_i(t=t'+\tau_i)}.
\end{eqnarray}
In the second line, we divide and multiplied back an effective mean-free-path $\tilde{\lambda} = m_D^2/\hat{q}_g$.
In doing so, the first factor is interpreted as the incoherent branching rate $R_{\textrm{incoh}}$, while the second factor is simple the acceptance factor for incoherent branching samples we introduced before.
The formation time can be determined self-consistently for each branching copy as it is evolved under the influence of elastic broadening.
It is determined at the momentum when
\begin{eqnarray}
t - t' < \tau_f(t). 
\end{eqnarray}
This iterative approach for determine $\tau_f$ was first developed and implemented by \cite{Zapp:2011ya}.
In the deep-LPM region where the number of rescattering is large, such a procedure reproduce the expected scaling of the average formation time $\left\langle\tau_f\right\rangle \propto \sqrt{\omega/\hat{q}}$.
This approach also generalizes to medium with non-constant temperature profile as the re-scatterings are performed locally as the probe propagate through the medium.
In cases where the formation time is short that the acceptance probability is bigger than unity, the acceptance is set to one and the incoherent rate is recovered as the branching with $\tau_f < \tilde{\lambda}$ has entered the Bethe-Heitler regime.
Therefore, this approach natural provides an interpolation of the deep-LPM for energetic branching and the Bethe-Heitler regime for soft branching in a large medium.

Now we will determine the form of $a$ and $b$ parameters with guidance from the theory in the deep-LPM region.
In the leading-log formula, the average inverse formation time is $\langle\tau_f^{-1}\rangle \sim \sqrt{\hat{q}_3 / 2x(1-x)E}$. 
One notice that the effective $\hat{q}_3$ is different from the $\hat{q}$ of a daughter parton, where we perform momentum broadening.
$\hat{q}_3$ is related to the gluon $\hat{q}$ by the process- and $x-$dependent factor $C_{abc}$ that has been defined before.
For this reason, we chose the $a$ parameter to be this color combination for each branching channel.
\begin{eqnarray}
a \rightarrow a_{abc}(x) = \frac{C_A}{C_{abc}(x)}
\end{eqnarray}
In this way, the formation time for different processes is determined by performing the rescattering broadening of the gluon.

From the previous theory discussion, we know that there is a logarithmic ambiguity in the cut-off scale $Q_0$ in $\hat{q}_3$, and can be determined at the next-to-leading-log level to be the same order as the branching transverse momentum.
Let us examine what is the $Q_0$ scale in our Botlzmann simulation and how to improve that.
When we include the large-$Q$ part of the interaction through the matrix-element in the Boltzmann simulation, the upper bound of the momentum transfer integration is nationally cut-off by the center-of-mass energy $\sqrt{s}$ of each independent collision,
\begin{eqnarray}
s = (p_1 + p_2)^2 = 2E_1 E_2 (1-\cos(\theta))
\end{eqnarray}
where $p_2$ is the momentum of the medium parton.
Since at high energy, the cross-section evolves slowly with $\sqrt{s}$, we can define the average $\sqrt{s}$ by simply averaging $p_2$ over the thermal distribution,
\begin{eqnarray}
2p_1\frac{ \int p_2^3 e^{-p_2/T}(1-\cos\theta) dp_2 d\cos \theta }{\int p_2^2 e^{-p_2/T} dp_2 d\cos \theta} =  6ET.
\end{eqnarray}
Therefore, the average $Q_0^2$ from the independent transport simulation is $6ET$, compared to NLL choice of $\sqrt{\hat{q} \omega}$
The prediction from such simulation would be systematically deviated from theory prediction in a logarithmic manner, varying energy, temperature and coupling constant.
To use the correct scale, we define a scale-dependent acceptance probability to correct the na\"ive choice of $Q_0 \sim \sqrt{6ET}$ with a $b$ parameter,
\begin{eqnarray}
b &=& 0.75\sqrt{\frac{\ln(\hat{Q}_1^2 )}{\ln(\hat{Q}_0^2 )}}.
\label{eq:NLL-b}
\end{eqnarray}
with $\hat{Q}_1^2$ and $\hat{Q}_0^2$ given by,
\begin{eqnarray}
\hat{Q}_1^2 &=& 1 + \frac{\sqrt{\omega\hat{q}}}{m_D^2} \approx 1 + \frac{\tau_f}{\tilde{\lambda}}\\
\hat{Q}_0^2 &=& 1 + \frac{6ET}{m_D^2}
\end{eqnarray}
The $0.75$ is a constant determined when tuning the simulation to theoretical calculations in the next section, and it will be the same throughout the entire work.
The origin of this logarithmic ambiguity comes from the perturbative large-$q^2$ tail the vacuum $\hat{t}$-channel matrix-element, which is of course a reasonable assumption based on perturbative physics. 
However, if one assumes an absence of such a slowly decaying tail in $q^2$, such as non-perturbative physics motivated coupling between hard parton and the medium, the logarithmic part in the $b-$parameter is not necessary.

\subsection{Mass effect}
Heavy quark's large mass $M\gg T$ and unique flavor make it an excellent probe and we would like to apply the aforementioned approach to study heavy flavor as well. 
Of course, both the theory and the method described should only work in the the limit that the parton energy is still large compared to the heavy quark mass.
In the region $p \lesssim M/g$ at weak coupling, the $M/E \ll 1$ breaks down and the elastic energy loss starts to dominate over radiate processes.

Considering that heavy quark introduces a mass correction to the Fermion propagator, it looks that the simplest change is to include the mass effect in both the formation time and the matrix-element,
\begin{eqnarray}
\tau_f = \frac{2x(1-x)E}{k_\perp^2} \rightarrow \frac{2x(1-x)E}{k_\perp^2 + x^2 M^2}
\end{eqnarray}
and 
\begin{eqnarray}
\overline{|M|^2}_{2\leftrightarrow 2}(m=0) \rightarrow \overline{|M|^2}_{2\leftrightarrow 2}(m=M)\\
\overline{|M|^2}_{2\leftrightarrow 3}(m=0) \rightarrow \overline{|M|^2}_{2\leftrightarrow 3}(m=M)
\end{eqnarray}
For elastic scatterings, this replacement using the massive version of the two-body matrix-element is justified because subsequent elastic collisions are incoherent at weak coupling limit.
For inelastic scatterings, again the problem comes from the coherence over multiple scattering centers.
At high energy, a heavy quark acquires an average transverse momentum $\hat{q} \tau_f$ larger than the typical transverse momentum of the few body matrix-element $\overline{|M|^2}_{2\leftrightarrow 3}$.
As a result, the mass-effect should less important compared to the scale $\hat{q} \tau_f$ than comparing to the transverse momentum acquired from a single collision center. 

To solve this problem in the simulation, we choose to use the dead-cone approximation for the radiation off a heavy quark.
The $2\rightarrow 3$ and $1\rightarrow 2$ branching of the heavy quark is sampled from the mass-less matrix-element, while the formation time is determined using the massive formula.
The key change is that the acceptance probability is modified by the dead-cone factor,
\begin{eqnarray}
\frac{b\lambda}{\tau_f} \rightarrow \frac{b\lambda}{\tau_f} \left(\frac{k_\perp^2}{k_\perp^2+x^2M^2}\right)^2
\end{eqnarray}
but note that the $k_\perp$ here is the branching transverse momentum after the elastic momentum broadening, and on average $\langle k_\perp^2 \rangle = \langle k_{0,\perp}^2 \rangle + \langle\tau_f\hat{q}\rangle$, where $\langle k_{0,\perp}^2 \rangle$ is the average transverse momentum sampled from the the $2\rightarrow 3$ matrix-element.
One may argue the accuracy of approximating the massive version of the complicated multiple scattering matrix-element using a dead-cone approximation.
To evaluate this procedure, we will also compare the radiation spectrum off the heavy quark to the exact solution with mass effect.


\subsection{Implementing the running of $\alpha_s$}
There are two places in the transport model where the running of the strong coupling constant might be important:
the coupling between the hard patron and the medium, and the coupling constant for the branching vertices.
These two types of processes are often happen at different scales.

For the $\hat{t}$-channel elastic interaction the scale would be the $\hat{t}$-channel momentum transfer, the typical scale is on the order of the screening mass $|\hat{t}| \sim m_D^2$.  
Using leading order running of $\alpha_s$ $n_f = 3$ and $\Lambda = 0.2$ GeV, 
\begin{eqnarray}
\alpha_s(Q^2) = \frac{4\pi}{9\ln\left(Q^2/\Lambda^2\right)}
\end{eqnarray}
the coupling constant will blow up with the scale getting close to the non-perturbative scale $\Lambda^2$, and applying leading order perturbative calculation to such regions is problematic. 
However, in phenomenological application, the temperature of the medium inevitably evolves down to the critical temperate and the scale of the elastic scattering processes is getting too close to $\Lambda^2$.
We therefore introduce a minimum scale in the running coupling, proptional to the temperature $Q_{\textrm{med}} = \mu \pi T$, to regulate the leading order running formula.
Of course, regulating $\alpha_s$ to a finite number using such a medium scale does not necessarily improve the accuracy of using the leading order  calculation in this temperature range. 
For example $\alpha_s(2\pi T)$ ranges from $0.28$ to $0.45$ ($g \sim 1.9 -- 2.4$) for temperature decreasing from $T=0.4$ GeV to $T_c$, which are extremely large values considering the next-to-leading-order correction to the probe-medium correction is $O(g)$.  
As a result, in the model-to-data comparison, we also try to parametrize the non-pertubative contribution by a diffusion processes in our model to prevent the attempt to explain the coupling to sQGP in a pure pertrubative framework.

Unlike the coupling between hard parton and the medium, the scale for the hard-to-hard splitting of partons are much harder than the screening mass.
In a static medium, this is because that splitting with a short formation time (compared to the mean-free-path) involves a large transverse momentum $k_\perp^2 > 2x(1-x)E /\lambda \sim m_D \omega/T$, while splitting with longer formation time receives multiple scattering contribution and the average $k_\perp^2$ scales like $\sqrt{2x(1-x)E\hat{q}} \sim m_D^2 \sqrt{\omega/T}$.
Therefore for splitting where both the daughter patrons are hard $xE, (1-x)E \gg T$, the running of the splitting vertex coupling is under better control than the probe-medium coupling.
For the case of an expanding and fluctuating medium, we shall use a realistic simulation to study the distribution of the actually couplings in the model to access the applicability of the perturbative approach.

Here is detailed implementation of running $\alpha_s$ in our transport model. 
From \cite{Arnold:2008zu}, the collision kernel $\mathcal{C}$ couplings are evaluated at $\hat{t}$-channel momentum transfer $q_\perp^2$.
This involve both the $\alpha_s$ in the large-$q$ matrix-element ($2\leftrightarrow 2$, and the elastic matrix-element factorized in $2\leftrightarrow 3$) as well as the $\alpha_s$ in the soft transport coefficients $\hat{q}_S$ and $\hat{q}_{S, L}$.
The running of the splitting vertex, taking into account the possible multiple scatterings, is a two-step implementation. 
First, the $\alpha_s$ for the splitting vertex in the few body matrix-element is evaluated at $k_\perp^2$.
Next, at the end of the elastic broadening for each splitting processes, the acceptance probability is multiplied by a running coupling factor
\begin{eqnarray}
p^{\textrm{running}} = p\times \frac{\alpha_s(k_{\perp,t_0+\tau_f}^2)}{\alpha_s(k_{\perp,t_0}^2)}.
\end{eqnarray}
Where $k_{\perp,t_0}^2$ is the transverse momentum when the splitting is generate from the few-body processes, while $k_{\perp,t_0+\tau_f}^2$ is the final transverse momentum including the elastic broadening, and is on average greater than $k_{\perp,t_0}^2$.

\section{Validate the modified transport approach to theoretical calculation: large and static medium}

\begin{figure}
\centering{
\includegraphics[width=.7\columnwidth]{spectrum.png}
}
\caption{The $q\rightarrow q+g$ splitting rate in an infinite medium from a quark with $E=1$ TeV,and a coupling constant $\alpha_s = 0.1$. The top plot shows the simulated spectrum $dR/d\omega$ (red-dashed line) and power law fit (green-dotted and blue-dash-dotted lines) in different gluon energy regions, separated by energy scales $\omega_{BH}\approx 2\pi T$. The middle plot compares to the simulation to NLL solution to the AMY equation, and the ratio is shown in the bottom plot}
\label{fig:spectrum}
\end{figure}


In this section, we first compare the splitting rate $dR/d\omega$ that comes out of the modified Boltzmann approach simulation to the NLL approximation in the infinite medium limit.
In practice, to define a Monte-Carlo transport simulation an infinite medium limit and an eikonal limit of parton propagation, an ensemble of parton of certain species are initialized at a fixed energy $E_0$ and will be let propagate in the same direction.
Each time when a parton scatters elastically or splits, its splitting kinematics are taken down $\omega, k_\perp, t_0, \tau_f$, then the mother parton's energy is reset back to its initial value.
For elastic re-scatterings in the implementation of the LPM effect, the parton's energy is re-scaled back to the value before scatterings without changing its direction.
The system is evolved for a sufficiently long time $t_{\max}$, and only branchings with branching time within $[t_{\min}, t_{\max}]$ are analyzed to focus on the infinite time behavior of the transport model.

We start by taking a detailed look at the $q\rightarrow q+g$ channel.
Its differential rate $dR/d\omega$ for a 1 TeV quark propagate through a medium of $T=0.5 GeV$ with coupling constant $\alpha_s = 0.1$ is shown in 
figure \ref{fig:spectrum}.
The vertical axis is the differential branching rate $dR/d\omega$, and the horizontal axis is the energy of the final state gluon $\omega$.
To better understand our result, we have put three ``landmark'' energy scales in the upper plot, which are the initial parton energy $E$, an estimate of the Bethe-Heitler energy $\omega_{\textrm{BH}}\hat{q}_g \lambda_g^2 \sim 2\pi T$, and the screening mass $m_g = m_D/\sqrt{2}$.
In the LPM regime $\omega_{\textrm{BH}} < \omega$, the spectrum falls off as a power law with fitted exponent $-1.50$ (the blue dash-dotted line), and the in the Bethe-Heitler regime above the screening mass $m_g < \omega < \omega_{\textrm{BH}}$, the fitted power law exponent is close to $-1$ (the green dotted line).
These exponents are in good agreement with the theoretically expectation that $dR_{\textrm{BH}}/d\omega \propto \omega^{-1}$ and $dR_{\textrm{LPM}}/d\omega \propto \omega^{-3/2}$.
The screening mass regulates the soft divergence of the spectrum below $m_g$.
One may notice a tiny increase of the spectrum when $\omega \rightarrow E$, this is region where the gluon takes a larger fraction of the initial quark's energy.

In the middle plot, we compare this result from simulation directly to the NLL solution of the AMY equation. 
As a remark, we have tuned the prefactor in the $b$-parameter to be $0.75$  by comparing to this theory prediction at $\alpha_s=0.1, E=1 \textrm{TeV}, T = 0.5 \textrm{GeV}$ for the $q\rightarrow q+g$ channel.
For the rest of the comparison with different coupling, parton energy, temperature, and channels, this parameter will {\bf not} be further tuned.
The simulation agrees with the NLL solution very well when $\omega \gg \omega_{\textrm{BH}}$ where the formula is valid.
The bottom plot show the ratio between the simulation and the theory, an level of $\pm 10\%$ agreement in the deep-LPM region is achieved in the static case.

\begin{figure}
\centering{
\includegraphics[width=.7\columnwidth]{channel_rate.png}
}
\caption{The splitting rate of $q\rightarrow q+g^*$, $g\rightarrow g+g^*$, and $g\rightarrow q^* + \bar{q}$ as a function of the parton energy labeled by the star. The mother parton with $E=1$ TeV evolves inside an infinite medium with $T=0.5$ GeV. The simulations (thick dashed lines) are compared to the NLL solutions (thin solid lines).}
\label{fig:channel_rate}
\end{figure}

Next, we would like to compare the simulation all the three channels in figure \ref{fig:channel_rate}.
The setup is the same as the figure \ref{fig:spectrum}.
The red, green and blue lines correspond to the differential branching rate of processes $q\rightarrow q+g^*$, $g\rightarrow g^*+g^*$ and $g\rightarrow q^*+\bar{q}$; the thin back lines are the NLL solution to the AMY equation.
The ``${}^*$'' sign denotes the final state parton whose energy is $\omega$.
For the case two final state gluons, both are taken into account in the simulation as they are identical particles.
We have discussed the feature the $q\rightarrow q+g$ in the previous paragraph. 
The spectrum shape of $g\rightarrow g+g$ process is very similar to the quark splitting channel in the range $\omega \ll E$, with a higher value.
The rate is symmetric with respect to $\omega = E/2$ due to its symmetric final states (though it is hard to tell from this double-log plot), so at large $\omega$, the rate goes up again.
The spectrum of $g\rightarrow q+\bar{q}$ is also symmetric with respect to $\omega = E/2$.
Though its final state consists of two different particle, the splitting function is still symmetric in this case.
We see that the simulation achieves a good agreement with the NLL solution in the deep-LPM region $\omega/T > 10$.

Next, we would like validate the simulation with different coupling constant and parton energies.
We choose both a relative small coupling $\alpha_s = 0.1 (g \approx 1.1)$ and a value closer to the phenomenology coupling $\alpha_s = 0.3 (g \approx 1.9)$, and vary the energy from $10$, $10^2$, to $10^3$ GeV.
The ratios between the simulation and the NLL solutions are shown in figure \ref{fig:sys-q2qg} and \ref{fig:sys-g2gg-g2qqbar}.
From these systematic comparison.
One see that the simulation reproduces the correct scaling in the LPM region, although due to the decreasing of the parton energy, this region also shrinks.
The overall performance of the modified Boltzmann transport in describing the inelastic processes in a large medium is good and under control.
One remaining problem is that the systematic deviation for the $g\rightarrow q+\bar{q}$ channel is bigger than the other two channels, as we did not include the backward region in its $2\rightarrow 3$ matrix-elements.



\begin{figure}
\centering{
\includegraphics[width=.5\columnwidth]{spectrum_E_q2qg.png}
}
\caption{Ratios of splitting rate $dR/\omega$ between the modified Boltzmann simulation and the NLL solution for $q\rightarrow q+g$ splitting. The quark energies are $E$ is 10, 100, and 100 GeV from top to the bottom plot. 
And two coupling constants are used: $\alpha_s = 0.1$ (red solid lines) and $\alpha_s = 0.3$ (blue dashed lines).
$\omega$ stands for the gluon energy.
The horizontal dashed lines denote $\pm 10\%$ deviation from unity. }
\label{fig:sys-q2qg}
\end{figure}

\begin{figure}
\centering{
\includegraphics[width=.5\columnwidth]{spectrum_E_g2gg.png}\includegraphics[width=.5\columnwidth]{spectrum_E_g2qqbar.png}
}
\caption{Same as figure \ref{fig:sys-q2qg}, but for $g \rightarrow g + \bar{g}$ (left) and $g \rightarrow q + \bar{q}$ (right)}
\label{fig:sys-g2gg-g2qqbar}
\end{figure}


Finally, we compare the running coupling calculation with the theory curve in Fig. \ref{fig:running} for the $g\rightarrow g+g$ channel.
The theory curves (black lines) are obtained combining Eq. \ref{eq:AMY-LL} and Eq. \ref{eq:q3running}.
Different line styles correspond to the variation of the $Q_0$ value around an initial guess $m_D (E/T \ln(E/T) )^{1/4}$ by a factor of $2$ above and below.
For this 1 TeV parton, the scale $Q_0$ is actually very large and the running of $\alpha_s$ is rather slow, which explains the theory curve is not very sensitive to a factor of $4$ change in $Q_0$.
The simulation was performed using the running coupling prescription described in Section \ref{section:running}.
The overall shape of the spectrum in the deep LPM region is again well described by the modified Boltzmann simulation. 

In the end of this section, we also validate the implementation of the running coupling constant.
As we have discussed in the previous section, the scale for the interaction vertex with the medium is chosen at the momentum transfer $|\hat{t}|$, and the scale for the branching vertex at its transverse momentum $k_\perp^2$, including the elastic broadening effect.
And the detailed implementation in the simulation has been discussed.
On the theoretical side, these effects can be included by changing the effective $\hat{q}_3$ in the NLL formula to 
\begin{eqnarray}
\hat{q}_3^{\textrm{running}} \approx \frac{4\pi}{9}\left(g^2(m_D^2) - g^2(Q_0^2)\right) 1.27 T^3 C_{abc}(x)
\label{eq:q3running}
\end{eqnarray}
And then evaluate the splitting $\alpha_s$ around an averaged scale (note that $k_\perp^2$ in the simulation fluctuates a lot),
\begin{eqnarray}
\langle k_\perp^2\rangle \sim m_D^2 \sqrt{E/T\ln(E/T)}
\end{eqnarray}  
We use $g\rightarrow g+g$ channel as an example in figure \ref{fig:running}.
The red line is from the modified transport simulation, and the different black lines are theory curves varying the choice of the branching vertex scale around the center value by a factor of 2.
The ratios are shown in the bottom plots, we see that the spectrum shape is still well described by our simulation in the running coupling case. 
Of course, there is an ambiguity (though rather small) is comparing the absolute magnitude due to different choice of scale in the theoretical formula.

\begin{figure}
\centering{
\includegraphics[width=.7\columnwidth]{running.png}
}
\caption{Top plot: comparison of modified Boltzmann simulation with the NLL solution  with running coupling.
The ratio between simulation and theory is shown in the bottom plot.}
\label{fig:running}
\end{figure}


\section{Finite / expanding medium and mass effect}
We have made clear before that this approach is designed for interpolating the Bethe-Heitler region and the deep-LPM region in a large medium, and from the validation in the previous section, it indeed works very well.
However, the medium created in heavy-ion collisions were never in the large and static limit, its finite time and spatial extend, local hot spots fluctuations and the fast radial expansion can all make significant impact on the hard parton propagation. 
Therefore, we need to investigate how our approach would behave in a few more complex scenarios: a finite medium and and expanding medium, before applying such a model to phenomenological usage.

\paragraph{A semi-infinite medium}
Consider the semi-infinite medium with the static temperature profile,
\begin{eqnarray}
T = \begin{cases}
0 , z<0\\
T_0, z>0
\end{cases}
\end{eqnarray}
and hard partons are created at $z=0$ and propagate into the medium.
Deep inside the medium, the medium induced radiation should be getting asymptotically close to the calculation in an infinite medium.
However, at the boundary, there is a complicated interference between medium scatterings centers and the hard production vertex.
For a thin medium where the path length is short compared to the formation time, these interference terms can be worked out in the ``opacity ($L/\lambda$) expansion", or by analyzing the propagator in the path-integral formalism with the semi-infinite temperature profile.
This boundary effect results in a path length dependence of the medium induced branching rate that starts from zero at $t=0$ and gradually approach the asymptotic value in a large medium.
The resulting parton energy loss rate is significantly reduced due to this effect, and scales quadratic with the path length $\Delta E \propto L^2$ near the boundary (at larger times, it transits to $\Delta E \propto L$).
This feature can have significant phenomenology impact in understanding the nuclear modification factor and the azimuthal anisotropy of the leading particles, heavy flavors and jets.

It is true that our approach is designed for a large medium, but it also displays certain finite size effect. 
Of course, it is can not reach an quantitative level of agreement with the theory at $L \lesssim \tau_f$ since the interference pattern is not implemented.
The cause of the finite size effect in our approach can be understood as following.
Remember that the branchings in the modified transport approach takes a finite amount of time, and those branching that becomes independent at time $t$ are actually initiated by a $2\rightarrow 3$ processes at from a wide range of scattering centers in the past $t' = t - \tau_f$.
Therefore, if the medium is semi-infinite, and there were no scattering centers before $t' = t-\tau_f < 0$, then the medium-induced contribution to the branchings at time $t$ will be reduced.
This reduction gets weaker and weaker when the condition $t-\tau_f > 0$ can be satisfied by more and more induced branchings and eventually, when $t\gg \langle \tau_f\rangle$, this boundary effect dies off in the simulation. 
We would like to check if this simulation boundary effect qualitatively mimic the interference physics that happens near the boundary.


\begin{figure}
\centering
\includegraphics[width=.8\columnwidth]{spectrum_L.png}
\caption{Comparison of the path-length dependent rate $dR/d\omega$ from the simulation using $\alpha_s = 0.3$ to the theoretical calculation for splitting $q\rightarrow q+g$ \cite{CaronHuot:2010bp}. The quark energy is $16$ GeV.}
\label{fig:spectra-L-alphas=0.3}
\end{figure}

In Figure \ref{fig:spectra-L-alphas=0.3}, the differential rate obtained from simulation is compared to the numerical solution of the full leading order calculation inside a finite medium.
The horizontal axis is the time of travel by the hard parton (path length divided by the speed of light), and each subplot shows how the branching rate changes as a function of time with different medium temperatures ($T=0.2$ GeV on the left, $T=0.5$ GeV on the right) and at different branching parton energy ($\omega=3$ GeV at the top, $\omega=8$ GeV at the bottom).
The theory curves are taken from the references [] for a 16 GeV parton with coupling constant $\alpha_s = 0.3$, and the red lines are our simulation.
The theory curve first increases linearly and then turn over to a constant value in the large medium limit for $t \gg \sqrt{2x(1-x)E/\hat{q}_3}$.
The simulation, as expected, reproduces the large time limit of the rate.
Moreover, we find that the current implementation also predicts the qualitative ``turn over'' of the spectra at finite path length.
The original paper only publish this calculation for a $16$ GeV quark. 
To validate if this qualitative agreement also holds at higher parton energies, we implement the numerical approach described in [] and compute the theoretical curves for $E=100$ GeV partons.
The comparison of simulation and numerical solutions are shown in figure \ref{fig:spectra-L-alphas=0.3-E100} and again, we found a qualitative agreement with the theoretical finite size effect.


\begin{figure}
\centering
\includegraphics[width=.8\columnwidth]{spectrum_L_100.png}
\caption{The same as figure \ref{fig:spectra-L-alphas=0.3-E100}, but the initial quark energy is $100$ GeV.}
\label{fig:spectra-L-alphas=0.3-E100}
\end{figure}


\begin{figure}
\includegraphics[width=\columnwidth]{spectrum_Bjorken.png}
\caption{The ratios of induced splitting rate in expanding medium to that of a stat medium, with expansion parameter $\nu = 3/4, 1$, and $3/2$. The analytic results are shown in solid lines and simulations denoted as symbols. The conpling constant $\alpha_s=0.3$, the expansion starts at $\tau_0 = 0.2$ fm/$c$ with an initial temperature $T_0 = 1$ GeV.}
\label{fig:Bjorken-BDMPS}
\end{figure}

\paragraph{An expanding medium}
Fast radial expansion is another important feature of the created quark-gluon plasma. 
It causes the temperature at the medium rapidity to decrease drastically in the early stages of the expansion and introduces another time scale in which the medium temperature changes notably.
With an simplified power-law changing tempature profile
\begin{eqnarray}
\tau_{\textrm{ex}} = \left(\frac{d\ln(T^3)}{d \tau} \right)^{-1},
\end{eqnarray}
to be understood as the time scale over which the local $\hat{q}\propto T^3$ changes notably.
For simplicity, parametrize the temperature profile as a power law fall-off function of the proper time at mid-rapidity,
\begin{eqnarray}
T(\tau; \nu)^3 = T_0^3\left(\frac{\tau_0}{\tau}\right)^{2-1/\nu},
\end{eqnarray}
where the static case is recovered when $\nu=1/2$, and $\nu=1$ corresponds to the temperature profile of a Bjorken flow.
The resultant expansion time scale is
\begin{eqnarray}
\tau_{\textrm{ex}} = \frac{\tau}{2-1/\nu}.
\end{eqnarray}
If this time scale is smaller than the formation time of the splitting, then the medium cannot be well approximated by having a constant temperature when we resumes the multiple scatterings.
Fortunately, in the current modified Boltzmann approach, since we propagate the splitting process in real time, the fast changing of the medium temperature does affect the multiple scatterings that contribute to this specific splitting.

We would like to compare the response of the modified Boltzmann approach to an expanding medium to theoretical calculations.
The theoretical formula we used is obtained in the BDMPS framework 
by the authors of \cite{Baier:1998yf} using the power law type temperature profile. 
The total splitting probability is,
\begin{eqnarray}
\frac{dP}{d\omega} &=& \frac{\alpha_s}{2\pi E}P_{q\rightarrow qg}(x)\mathfrak{Re}\int_{\tau_0}^{\tau_0+L}\frac{dt_f}{t_f}\int_{\tau_0}^{t_f}\frac{dt_i}{t_i} \frac{1}{\nu^2}\\
\nonumber
&& \left.\left[ I_{\nu-1}(z_i)K_{\nu-1}(z_f)-I_{\nu-1}(z_f)K_{\nu-1}(z_i)\right]^{-2}\right|_{\omega}^{\omega=\infty},\\
z_{i,f} &=& 2i\nu \sqrt{\frac{\hat{q}_g(1-x+C_F/C_A x^2)}{2(1-x)\omega}} \tau_0 \left( \frac{t_{i,f}}{\tau_0}\right) ^{1/2\nu}
\end{eqnarray}
for the $q\rightarrow q+g$ splitting.
For $\nu=1/2$, this expression reduces to the static BDMPS result \cite{Baier:1996kr}. 

As a remark, the BDMPS calculation considers the multiple-soft limit of the collision kernel and therefore does not include the logarithm that comes from the perturbative tail $1/q_\perp^4$. 
Accordingly, we turn off the large-$Q$ matrix-element scatterings and only retain diffusion plus diffusion-induced radiation components in our simulation.
Besides, $b=0.75$ is used without the logarithmic correction factor in Eq. \ref{eq:NLL-b}, and the same $\hat{q}_g = m_D^2 C_A\alpha_s T$ are input to the theory and the simulation.
To suppress other difference in the simulation and the theory, instead of making direct comparison of the spectrum $dP/d\omega$ to the BDMPS result, we compare the ratio of the splitting probability in an expanding medium to that of a static medium
\begin{eqnarray}
R_\nu = \frac{dP(T=T(\tau;\nu))/d\omega}{dP(T=T_0)/d\omega}
\end{eqnarray}
between simulation and theory to focus on the response to a fast dropping temperature profile compared to the static case.

The medium expansion starts at $\tau_0=0.2$ fm/$c$ with $T_0=1$ GeV and stops at $\tau = 20$ fm/$c$.
We take four choices of the expansion rate $\nu = 1/2, 3/4, 1, 3/2$, corresponding to a static medium, a slowly expanding medium, Bjorken flow, and a faster-than-Bjorken expansion respectively.
The ratio $R_\nu$ from both theory and simulation are shown in Fig. \ref{fig:Bjorken-BDMPS} for a 100 GeV quark with $\alpha_s=0.3$.
Again, for $\omega/T \gg 1$, the simulation displays the expected decreasing of medium-induced radiation due to the dropping of temperature.
In the future, we are looking forward to making direct comparison to the solution of Eq. \ref{eq:full-theory} with both varying temperature and adding medium flow effects.

\begin{figure}
\includegraphics[width=\columnwidth]{mass.png}
\caption{Comparison of the path-length dependent rate $dR/d\omega$ from the simulation using $\alpha_s = 0.3$ to the theoretical calculation for splitting $q\rightarrow q+g$ \cite{CaronHuot:2010bp}. The quark energy is $16$ GeV.}
\label{fig:spectra-L-alphas=0.3}
\end{figure}

\section{Hard matrix-elements}
The vacuum matrix-elements for two-body large-$Q$ elastic scatterings are,
\begin{eqnarray}
\overline{|M_{22,Qq}|^2} &=& \frac{64\pi^2\alpha_s^2}{9} \frac{(M^2-u)^2 + (s-M^2)^2 + 2 M^2 t}{t^2}
\nonumber
\\
\overline{|M_{22,Qg}|^2} &=& \pi^2 \left\{
32\alpha_s^2 \frac{(s-M^2)(M^2-u)}{t^2} \right.
\nonumber
\\
&+&\frac{64}{9}\alpha_s^2 \frac{(s-M^2)(M^2-u)+2M^2(s+M^2)}{(s-M^2)^2} \nonumber
\\
&+&\frac{64}{9}\alpha_s^2 \frac{(s-M^2)(M^2-u)+2M^2(u+M^2)}{(M^2-u)^2} \nonumber
\\
&+& \frac{16}{9}\alpha_s^2 \frac{M^2(4M^2 - t)}{(M^2-u)(s-M^2)} 
\nonumber
\\
&+& 16 \alpha_s^2 \frac{(s-M^2)(M^2-u)+M^2(s-u)}{t(s-M^2)}
\nonumber
\\
&-& \left. 16 \alpha_s^2 \frac{(s-M^2)(M^2-u)-M^2(s-u)}{t(M^2-u)}\right\}
\end{eqnarray}
In medium, the denominator of the squared gluon propagator is replaced by $t^2 \rightarrow t(t-m_D^2)$. 
For the radiation processes, we also include a gluon mass to regulate soft divergence $x^2M^2 \rightarrow x^2M^2 + (1-x)m_g^2$, where $m_g^2 = m_D^2/2$ is the squared asymptotic gluon mass. 

For large-Q $2\rightarrow 3$ inelastic processes $g\rightarrow q+\bar{q}$, $q+q\rightarrow q+g+q$ and $g+q\rightarrow g+g+q$.
In medium frame, one of them will be the hard parton with energy $E\gg T$, while the other a thermal parton with $E\sim T$.
We perform the calculation in the the center-of-mass frame of the two incoming parton and let the hard parton to be moving towards the $+z$ direction with momentum $p_1$, and the medium parton moving to the $-z$ direction with $p_2$.
The momentum transfer between the hard parton and the medium parton is denoted as $q$ and thought to be large enough $q\gg Q_{\textrm{cut}}$ so we neglect the thermal correction to its propagator.
The hard parton, splits into two daughter partons with momenta $k$ and $p_1 + q - k$, where we shall consider the kinematic region that the rapidity of $k$ in the center-of-mass frame is positive $y_k > 0$.
Of course, the medium parton also splits, but they will only contribute to the backward region $y_k<0$ under the approximation scheme we apply.

The approximations we made in deriving the matrix-elements the collinear condition $k_\perp^2, q_\perp^2 \ll x(1-x) \hat{s}$, with $s = 2p_1p_2(1-\cos\theta_{12})$, and $x = k^+/\sqrt{s} = k_\perp e^y_k /\sqrt{s}$.
Combining these condition, we see that the approximation indeed requires $y_k \gg \ln(k_\perp/\sqrt{s})$ so that $y_k$ cannot be arbitrarily small and $y_k>0>\gg -\ln(\sqrt{s}/k_\perp)$ is a reasonable range of application.
In side a medium $\hat{s}$ is $6 ET$ on average, and we expect this approximation to breakdown when either the typical values of $q_\perp^2$ becomes comparable to $x(1-x)6ET$ or when $y_k<0$ ($x < k_\perp/\sqrt{s} \sim k_\perp/\sqrt{6ET}$).
We shall briefly mention the treatment of the $y_k<0$ region in the end.

The light-cone momentum for $p_1$ , $p_2$ and $k$ can written down directly using $\sqrt{s}$, $x$ and $k_\perp$, then applying the above collinear condition, the expression for $q$ (and therefore $p_3$ and $p_4$) is obtained by kinematic constraint up to corrections of order $\{k_\perp, q_\perp^2\}/x(1-x)\hat{s}$.
\begin{eqnarray}
p_1 &=& (\sqrt{s}, 0, \vec{0})\\
p_2 &=& (0, \sqrt{s}, \vec{0})\\
k &=& (x\sqrt{s}, \frac{k_\perp^2}{x\sqrt{s}}, \vec{k}_\perp)\\
q &\sim& (-\frac{q_\perp^2}{\sqrt{s}}, \frac{q_\perp^2 + k_\perp^2/x - 
2\vec{q}_\perp \cdot \vec{k}_\perp}{(1-x)\sqrt{s}}, \vec{k}_\perp)
\end{eqnarray}
We follow the derivation in [] to choose the light-cone gauge with light-like vector $n = (0, 1, 0)$, and the gauge fixing condition $n\cdot A =0$ eliminates the ``+" component in the gluon (with momentum $p$) polarization vector, and is written down by further applying the transverse condition $\epsilon \cdot p = 0$ (up to a higher order correction to its normalization)
\begin{eqnarray}
\epsilon(p) &\sim& (0, \frac{2\vec{\epsilon}_\perp\cdot\vec{p}_\perp}{p^+}, \vec{\epsilon}_\perp)
\end{eqnarray}.
With these preparations, our approximation only $\hat{t}$ momentum exchange with the medium parton is included.
The matrix-element is factorized into an amplitude for the splitting process (approximated in the collinear limit) times the amplitude for two-body collision with the medium parton.
We shall only derive explicitly the cases where the medium parton is a quark, for colliding with medium anti-quark and gluon, it is sufficient to replace the $H+q\xrightarrow{\hat{t}} H+q$ amplitude by $H+\bar{q}\xrightarrow{\hat{t}} H+\bar{q}$ and $H+g\xrightarrow{\hat{t}} H+g$.
The connect of these results to the Bethe-Heitler limit of the AMY integral equation will be elucidated in the end.

\paragraph*{Gluon splitting to quark-anti-quark pair}
\begin{figure}
\centering
\includegraphics[width=.5\textwidth]{Large-Q-g2qqbar-A.pdf}\\
\vspace{1em}
\includegraphics[width=.49\textwidth]{Large-Q-g2qqbar-B.pdf}\hfill
\includegraphics[width=.49\textwidth]{Large-Q-g2qqbar-C.pdf}
\caption{Three diagrams $A$ (Top), $B$ (Bottom left), $C$ (Bottom right) that contribute to the large angle scattering induced gluon splitting into quark-anti-quark pair.}
\label{fig:feyn-g2qqbar}
\end{figure}

Three Feynman diagrams contribute to the kinematic region $y_k >0$ in the current approximation, as shown in figure \ref{fig:feyn-g2qqbar}.
We start from the amplitude for diagram $A$.
\begin{eqnarray}
i M_A &=& (-ig)^2(-g)f^{abc}(t^b)_{j'j}(t^c)_{i'i} \epsilon_\lambda^\mu(p_1) \\\nonumber
&&\frac{-i}{(p_1+q)^2}\left(g^{\rho\rho'}-\frac{n^{\rho}(p_1+q)^{\rho'}+n^{\rho'}(p_1+q)^\rho}{n\cdot (p_1+q)}\right) \bar{u}^s(p_1+q-k)\gamma_{\rho'}v^{s'}(k) \\ \nonumber
&&\frac{-i}{q^2}\left(g^{\nu\nu'}-\frac{n^{\nu}q^{\nu'}+n^{\nu'}q^\nu}{n\cdot q}\right) \bar{u}^{\sigma}(p_4)\gamma_{\nu'}u^{\sigma'}(p_2) \\ \nonumber
&& \left[g_{\mu\nu}(p_1-q)_\rho + g_{\nu\rho}(2q+p_1)_\rho + g_{\rho\mu}(-2p_1 -q)_\nu \right]
\end{eqnarray}
Next, express the projection matrix of the gluon propagator with momentum $p_1+q$ by the sum of tensor products of its polarization vectors, and identify the amplitude $iP_{A,\lambda'}^{ss'}$ for a gluon with polarization $\lambda'$ to split into the quark and anti-quark pair with spin $s$ and $s'$.
Also, use the high energy approximation of the current $\bar{u}^i(a)\gamma^\alpha u^j(b)$ by $(a+b)^\alpha \delta^{ij}$,
\begin{eqnarray}
i M_A &\approx& -g^3 f^{abc}(t^b)_{j'j}(t^c)_{i'i} \delta^{\sigma\sigma'} \epsilon^\mu(p_1) \\\nonumber
&&\frac{1}{(p_1+q)^2} \sum_{\lambda'=\pm}\epsilon_{\lambda'}^{\rho}(p_1+q)\underbrace{\epsilon_{\lambda'}^{*,\rho'}(p_1+q) \bar{u}^s(p_1+q-k)\gamma_{\rho'}v^{s'}(k)}_{iP_{A,\lambda'}^{ss'}} \\ \nonumber
&&\frac{1}{q_\perp^2}\left(g^{\nu\nu'}-\frac{n^{\nu}q^{\nu'}+n^{\nu'}q^\nu}{n\cdot q}\right) (2p_2-q)_{\nu'} \\ \nonumber
&& \left[g_{\mu\nu}(p_1-q)_\rho + g_{\nu\rho}(2q+p_1)_\rho + g_{\rho\mu}(-2p_1 -q)_\nu \right] \\
&=& -g^3 f^{abc}(t^b)_{j'j}(t^c)_{i'i} \frac{1}{(p_1+q)^2}\frac{1}{q_\perp^2} \sum_{\lambda'=\pm}iP_{A,\lambda}^{ss'} \delta^{\sigma\sigma'}  \\ \nonumber
&& \epsilon_\lambda^\mu(p_1)2p_2^{\nu} \epsilon_{\lambda'}^{\rho}(p_1+q) \left[g_{\mu\nu}(p_1-q)_\rho + g_{\nu\rho}(2q+p_1)_\rho + g_{\rho\mu}(-2p_1 -q)_\nu \right].
\end{eqnarray}
Finally, we evaluate the contraction in the second line using the expression for $p_1, q$ and $\epsilon$, and keep only terms that is leading in $q_\perp^2/s$ to get,
\begin{eqnarray}
i M_A \approx -g^3 f^{abc}(t^b)_{j'j}(t^c)_{i'i}\delta^{\sigma\sigma'}\frac{2s}{q_\perp^2} \frac{x(1-x)}{(\vec{k}_\perp-x \vec{q}_\perp)^2} iP_{A,\lambda}^{ss'}.
\end{eqnarray}

Diagram B and C are similar and we only write down diagram B in detail.
\begin{eqnarray}
i M_B &=& (-ig)^3 (t^bt^a)_{i'i}(t^b)_{j'j} \epsilon_\lambda^\mu(p_1) \\\nonumber
&&\frac{-i}{q^2}\left(g^{\nu\nu'}-\frac{n^{\nu}q^{\nu'}+n^{\nu'}q^\nu}{n\cdot q}\right) \\\nonumber
&&\bar{u}^s(p_1+q-k)\gamma_{\nu}\frac{i(\slashed{p_1}-\slashed{k})}{(p_1-k)^2}\gamma^{\mu}v^{s'}(k) \\ \nonumber
&&\bar{u}^{\sigma}(p_4)\gamma_{\nu'}u^{\sigma'}(p_2)
\end{eqnarray}
Again, represent the tensor structure of the fermion propagator by the sum of tensor products of the spinors, identify the splitting amplitude $iP_{B,\lambda'}^{ss'}$ and use the high energy limit of the current,
\begin{eqnarray}
i M_B &\approx& ig^3 (t^bt^a)_{i'i}(t^b)_{j'j}  \\\nonumber
&&\frac{-i}{q_\perp^2}\left(g^{\nu\nu'}-\frac{n^{\nu}q^{\nu'}+n^{\nu'}q^\nu}{n\cdot q}\right) (2p_2-q)_\nu' \\\nonumber
&&\frac{1}{2p_1\cdot k} \sum_\sigma \bar{u}^s(p_1+q-k)\gamma_{\nu} u^{\sigma}(p_1-k) \underbrace{\epsilon_\lambda^\mu(p_1)\bar{u}^{\sigma}(p_1-k) \gamma^{\mu}v^{s'}(k)}_{iP_{B,\lambda}^{\sigma s'}}\\
&\approx& ig^3 (t^bt^a)_{i'i}(t^b)_{j'j} \frac{-i}{q_\perp^2}\frac{1}{2p_1\cdot k} iP_{B,\lambda}^{ss'}\\\nonumber
&&\left(g^{\nu\nu'}-\frac{n^{\nu}q^{\nu'}+n^{\nu'}q^\nu}{n\cdot q}\right) (2p_2-q)_{\nu'} (2p_1-q+2k)_\nu 
\end{eqnarray}
Note that $iP_{B}$ is different from $iP_{A}$ as the initial splitting parton has a different transverse momentum from diagram $A$.
Finally, evaluate the contraction and get,
\begin{eqnarray}
i M_B &=& i g^3 (t^b t^a)){i'i} t^b{j'j} \delta^{\sigma\sigma'} \frac{2s}{q_\perp^2} \frac{x(1-x)}{k_\perp^2}  iP_{B,\lambda}^{ss'}
\end{eqnarray}
Diagram C can be obtained similarly,
\begin{eqnarray}
i M_C &=& -i g^3 (t^a t^b)){i'i} t^b{j'j} \delta^{\sigma\sigma'} \frac{2s}{q_\perp^2} \frac{x(1-x)}{(\vec{k}_\perp-\vec{q}_\perp)^2}  iP_{C,\lambda}^{ss'} 
\end{eqnarray}
To sum the contributions from all three diagrams, applying $f^{abc}t^c = -i[t^a, t^b]$ to $iM_A$ and the result is,
\begin{eqnarray}
i (M_A+M_B+M_C) &=& ig^3 \frac{2s}{q_\perp^2} (t^b)_{j'j} x(1-x)\\\nonumber
&&\left\{(t^a t^b)_{i'i} \left(\frac{iP_{A,\lambda}^{ss'} }{(\vec{k}_\perp-x \vec{q}_\perp)^2} - \frac{iP_{C,\lambda}^{ss'}}{(\vec{k}_\perp-\vec{q}_\perp)^2}\right) \right. \\\nonumber
&&\left.-(t^a t^b)_{i'i}\left(\frac{iP_{A,\lambda}^{ss'} }{(\vec{k}_\perp-x \vec{q}_\perp)^2} - \frac{iP_{B,\lambda}^{ss'}}{k_\perp^2}\right) \right\}
\end{eqnarray}

Now we have to address what those splitting amplitudes are.
Label the four momenta as $p_g = c$, $p_q = a$, $p_{\bar{q}} = b$.
And use the following representation for the spinors,
\begin{eqnarray}
u^s(p) = (\sqrt{p\cdot \sigma} \xi^s, \sqrt{p\cdot \bar{\sigma}} \xi^s)^T
v^s(p) = (\sqrt{p\cdot \sigma} \eta^s, -\sqrt{p\cdot \bar{\sigma}} \eta^s)^T
\end{eqnarray}
where $\sigma_{i=\{1,2,3\}}$ are Pauli matrices, $\sigma = (1_{2\times 2}, \vec{\sigma})$, and $\bar{\sigma} = (1_{2\times 2}, -\vec{\sigma})$.
The square root of the matrix is,
\begin{eqnarray}
\sqrt{p\cdot \sigma} =
\left.
\begin{bmatrix}
p^- & -p_L^\perp \\
-p_R^\perp & p^+
\end{bmatrix}\right.^{1/2} 
= \frac{1}{\sqrt{2(E\pm M)}}(p\cdot\sigma \pm \mathbf{1}M)\\
\sqrt{p\cdot \bar{\sigma}} =
\left.
\begin{bmatrix}
p^+ & p_\perp^- \\
p_R^\perp & p^-
\end{bmatrix}\right.^{1/2} 
= \frac{1}{\sqrt{2(E\pm M)}}(p\cdot\bar{\sigma} \pm \mathbf{1}M)\\
\end{eqnarray}
where $M$ is the mass of the particle, $p^\pm = E\pm p_z$, and $p_{R,L}^\perp = p_x \pm  i p_y$.
Currently, for now we only consider the massless case, the mass effect will be discussed in the end.

Neglecting the mass, the splitting amplitude reads,
\begin{eqnarray}
&&\epsilon_{\lambda, \mu}(c) \bar{u}_s(a)\gamma^\mu v_{s'}(b)\\
&=&\frac{1}{\sqrt{2a}\sqrt{2b}}(\xi^T_s a\cdot\sigma, \xi^T_{s} a\cdot \bar{\sigma})
\begin{bmatrix}
\epsilon\cdot\bar{\sigma} & 0 \\
0 & \epsilon\cdot\sigma
\end{bmatrix}
\begin{bmatrix}
b\cdot\sigma \eta_{s'}\\
b\cdot\bar{\sigma} \eta_{s'}
\end{bmatrix}
\\
&=&\frac{1}{2\sqrt{ab}}
\xi_s^T
\begin{bmatrix}
a^- & -a^\perp_L \\
-a^\perp_R & a^+
\end{bmatrix}
\begin{bmatrix}
0 & \sqrt{2}\delta_{\lambda R}\\
\sqrt{2}\delta_{\lambda L} & \frac{\sqrt{2}c^\perp_\lambda}{c^+}
\end{bmatrix}
\begin{bmatrix}
b^- & -b^\perp_L \\
-b^\perp_R & b^-
\end{bmatrix}
\eta_{s'}\\\nonumber
&-&
\frac{1}{2\sqrt{ab}}
\xi_s^T
\begin{bmatrix}
a^+ & a^\perp_L \\
a^\perp_R & a^-
\end{bmatrix}
\begin{bmatrix}
\frac{\sqrt{2}c^\perp_\lambda}{c^+} & -\sqrt{2}\delta_{\lambda R}\\
-\sqrt{2}\delta_{\lambda L} & 0
\end{bmatrix}
\begin{bmatrix}
b^+ & b^\perp_L \\
b^\perp_R & b^-
\end{bmatrix}
\eta_{s'}
\\
&=&\frac{1}{\sqrt{2ab}}
\xi_s^T
\begin{bmatrix}
-a^\perp_L b^- \delta_{\lambda L} - a^- b^\perp_L \delta_{\lambda R} + a^\perp_L b^\perp_R\frac{c^\perp_\lambda}{c^+} &
a^\perp_L b^\perp_L \delta_{\lambda L} + a^- b^+ \delta_{\lambda R} - a^\perp_L b^+\frac{c^\perp_\lambda}{c^+}
\\
a^+ b^- \delta_{\lambda L} + a^\perp_R b^\perp_R \delta_{\lambda R} - a^+ b^\perp_R\frac{c^\perp_\lambda}{c^+} &
-a^+ b^\perp_L \delta_{\lambda L} - a^\perp_R b^+ \delta_{\lambda R} + a^+ b^+\frac{c^\perp_\lambda}{c^+}
\end{bmatrix}
\eta_{s'}\\\nonumber
&-&\frac{1}{\sqrt{2ab}}
\xi_s^T
\begin{bmatrix}
-a^\perp_L b^+ \delta_{\lambda L} - a^+ b^\perp_R \delta_{\lambda R} + a^+ b^+\frac{c^\perp_\lambda}{c^+} &
-a^\perp_L b^\perp_L \delta_{\lambda L} - a^+ b^- \delta_{\lambda R} + a^+ b^\perp_L\frac{c^\perp_\lambda}{c^+}
\\
-a^- b^+ \delta_{\lambda L} - a^\perp_R b^\perp_R \delta_{\lambda R} + a^\perp_+ b^+\frac{c^\perp_\lambda}{c^+} &
-a^- b^\perp_L \delta_{\lambda L} - a^\perp_R b^- \delta_{\lambda R} + a^\perp_R b^\perp_L\frac{c^\perp_\lambda}{c^+}
\end{bmatrix}
\eta_{s'}
\end{eqnarray}
Keep the leading terms in the collinear limit which are products of $(+)(+)$ or $(+)(\perp)$ components of the momenta, and drop terms that are of order $(+)(-)$, $(\perp)(\perp)$ and $(\perp)(-)$,
\begin{eqnarray}
&&\epsilon_{\lambda, \mu} \bar{u}_s(a)\gamma^\mu v_{s'}(b)\\
&=& \frac{1}{\sqrt{2ab}}
\xi_s^T
\begin{bmatrix}
a^\perp_L b^+ \delta_{\lambda L} + a^+ b^\perp_R \delta_{\lambda R} - a^+ b^+\frac{c^\perp_\lambda}{c^+} & 0\\
0 & -a^+ b^\perp_L \delta_{\lambda L} - a^\perp_R b^+ \delta_{\lambda R} + a^+ b^+\frac{c^\perp_\lambda}{c^+}
\end{bmatrix}
\eta_{s'}
\end{eqnarray}
There are four combinations for the possible initial state polarization and final state spins
\begin{eqnarray}
\epsilon_{\lambda, \mu} \bar{u}_s(a)\gamma^\mu v_{s'}(b) = \frac{x\vec{a} - (1-x)\vec{b}}{\sqrt{2x(1-x)}}
\begin{cases}
x, \hfill \lambda=L, s=\uparrow\\
-(1-x), \hfill \lambda=L, s=\downarrow\\
(1-x), \hfill \lambda=R, s=\uparrow\\
-x, \hfill \lambda=R, s=\downarrow\\
\end{cases}
\end{eqnarray}
Where we have use $a^+ = (1-x)c^+, b^+ = xc^+$ and $c_\perp = a_\perp+b_\perp$.
Sum over the spins and average over polarization for the squared amplitude,
\begin{eqnarray}
\frac{1}{2}\sum_\pm |P|^2 = \frac{2(x^2 + (1-x)^2)}{x(1-x)} \left((1-x)\vec{a}_\perp-x\vec{b}_\perp\right)^2.
\end{eqnarray}
This result goes back to the standard splitting function if it is computed in the frame where $a_\perp = -b_\perp$. 
However, there is no such frame that $a_\perp = -b_\perp$ satisfies simultaneously for the splitting in diagram A, B and C, therefore different amplitude needs to be inserted for each diagram and we find,
\begin{eqnarray}
\frac{\sum_{\lambda, s, s', \sigma, \sigma', a, b}|M^2|_{g+q\rightarrow q+\bar{q}+q}}{2d_F 2d_A} &=& g^4 \frac{2C_F}{d_A}\frac{4s^2 x(1-x)}{q_\perp^4}  \\\nonumber
&\times& g^2\frac{(x^2+(1-x)^2)}{2} \left(C_F \vec{A}^2 + C_F \vec{B}^2 - (2C_F- C_A)\vec{A}\cdot\vec{B}\right)
\end{eqnarray}
Where the $\vec{A}$ and $\vec{B}$ is,
\begin{eqnarray}
\vec{A} &=& \frac{\vec{k}_\perp - x\vec{q}_\perp}{(\vec{k}_\perp - x\vec{q}_\perp)^2} -  \frac{\vec{k}_\perp - \vec{q}_\perp}{(\vec{k}_\perp - \vec{q}_\perp)^2} \\
\vec{B} &=& \frac{\vec{k}_\perp - x\vec{q}_\perp}{(\vec{k}_\perp - x\vec{q}_\perp)^2} -  \frac{\vec{k}_\perp}{\vec{k}_\perp^2}
\end{eqnarray}
Therefore the final squared matrix-element has been factorized into the two body scattering part (first line) and the collinear splitting part (second line) with the desired leading order QCD splitting function. 

\paragraph*{Quark splits to quark and gluon}
\begin{figure}
\centering
\includegraphics[width=.5\textwidth]{Large-Q-q2qg-A.pdf}\\
\vspace{1em}
\includegraphics[width=.49\textwidth]{Large-Q-q2qg-B.pdf}\hfill
\includegraphics[width=.49\textwidth]{Large-Q-q2qg-C.pdf}
\caption{Three diagrams $A$ (Top), $B$ (Bottom left), $C$ (Bottom right) that contribute to the large angle scattering induced a quark splitting into a quark and a gluon.}
\label{fig:feyn-q2qg}
\end{figure}

The Feynman diagrams to be included for $q+q\rightarrow q+g+q$ are shown in Figure \ref{fig:feyn-q2qg}.
The calculation uses exactly the same technique we used for the gluon splitting channel, and we directly presents the result.
\begin{eqnarray}
\overline{|M^2|}_{g+q\rightarrow g+g+q} &=& 
 g^4 \frac{C_F}{d_F}\frac{4s^2}{q_\perp^4}x(1-x) \\\nonumber
&\times&g^2\frac{1+(1-x)^2}{x}  
\left(C_F\vec{A}^2 + C_F\vec{B}^2 - \left(2C_F-C_A\right)\vec{A}\cdot\vec{B}\right)\\
\vec{A} &=& \frac{\vec{k}_\perp - \vec{q}_\perp}{(\vec{k}_\perp - \vec{q}_\perp)^2} -  \frac{\vec{k}_\perp - x\vec{q}_\perp}{(\vec{k}_\perp - x\vec{q}_\perp)^2} \\
\vec{B} &=& \frac{\vec{k}_\perp - \vec{q}_\perp}{(\vec{k}_\perp - \vec{q}_\perp)^2} -  \frac{\vec{k}_\perp}{\vec{k}_\perp^2}
\end{eqnarray}


\paragraph*{Gluon splitting to two gluons}
\begin{figure}
\centering
\includegraphics[width=.5\textwidth]{Large-Q-g2gg-A.pdf}\\
\vspace{1em}
\includegraphics[width=.49\textwidth]{Large-Q-g2gg-B.pdf}\hfill
\includegraphics[width=.49\textwidth]{Large-Q-g2gg-C.pdf}
\caption{Three diagrams $A$ (Top), $B$ (Bottom left), $C$ (Bottom right) that contribute to the large angle scattering induced gluon splitting into two gluons.}
\label{fig:feyn-g2gg}
\end{figure}

Finally, for $g+q\rightarrow g+q+g$, the Feynman diagrams are shown in Figure \ref{fig:feyn-g2gg}. 
The simplification of the two body collision amplitude can be done in a similar manner as the previous two channels. 
Here we only write down the splitting amplitude $g\rightarrow g+ g$ in detail for each diagram.
Suppressing the color index, we label the initial gluon with $\epsilon_1^\mu(p)$, and the two daughter gluons with $\epsilon_2^\nu(k)$ and $\epsilon_3^\rho(q)$.
The splitting amplitude is then (omitting the factor $-gf^{abc}$)
\begin{eqnarray}
iP &=& \epsilon^\mu_1\epsilon^\nu_2\epsilon^\rho_3
\left[
g_{\mu\nu} (p+k)_{\rho} +  g_{\nu\rho} (p+k)_{\mu} + g_{\rho\mu} (-q-p)_{\nu}
\right]\\
&=& -\vec{\epsilon}_{1,\perp}\cdot \vec{\epsilon}_{2,\perp} \left[(p+k)^+\frac{\vec{\epsilon}_{3,\perp}\cdot \vec{q}_\perp}{q^+} - \vec{\epsilon}_{3,\perp}\cdot (\vec{p}_\perp+\vec{k}_\perp)\right] \\\nonumber
&&-\vec{\epsilon}_{2,\perp}\cdot \vec{\epsilon}_{3,\perp} \left[(-k+q)^+\frac{\vec{\epsilon}_{1,\perp}\cdot \vec{p}_\perp}{p^+} - \vec{\epsilon}_{1,\perp}\cdot (-\vec{k}_\perp+\vec{q}_\perp)\right]
\\\nonumber
&&-\vec{\epsilon}_{3,\perp}\cdot \vec{\epsilon}_{1,\perp} \left[(-q-p)^+\frac{\vec{\epsilon}_{2,\perp}\cdot \vec{k}_\perp}{k^+} - \vec{\epsilon}_{2,\perp}\cdot (-\vec{q}_\perp-\vec{p}_\perp)\right]
\end{eqnarray}
There are four possible combinations of the polarization vectors, and their respective amplitude is computed as,
\begin{eqnarray}
iP = \sqrt{2}\left[x\vec{q}_\perp - (1-x)\vec{k}_\perp\right]\times 
\begin{cases}
\frac{1-x+x^2}{x(1-x)}, \hfill \lambda_1=\lambda_2=\lambda_3\\
-1, \hfill \lambda_1\neq\lambda_2=\lambda_3 \\
\frac{1}{x}, \hfill \lambda_1=\lambda_3\neq\lambda_2\\
\frac{1}{1-x}, \hfill \lambda_1=\lambda_2\neq\lambda_3
\end{cases}
\end{eqnarray}
Summing over the squared amplitude of all four cases and average over the initial gluon polarization, one indeed get the desired leading order QCD splitting function,
\begin{eqnarray}
2\frac{1+x^2+(1-x)^4}{x^2(1-x)^2} \left[x\vec{q}_\perp - (1-x)\vec{k}_\perp\right]^2.
\end{eqnarray}

Substitute the the amplitude in each diagram, the final squared matrix-element is 
\begin{eqnarray}
\overline{|M^2|}_{g+q\rightarrow g+g+q} &=&
g^4 \frac{C_A}{d_F}\frac{4s^2x(1-x)}{q_\perp^4} \\\nonumber
&\times&g^2\frac{1+x^4+(1-x)^4}{x(1-x)}   
\left(C_A\vec{A}^2 + C_A\vec{B}^2 - C_A\vec{A}\cdot\vec{B}\right)\\
\vec{A} &=& \frac{\vec{k}_\perp - x\vec{q}_\perp}{(\vec{k}_\perp - x\vec{q}_\perp)^2} -  \frac{\vec{k}_\perp - \vec{q}_\perp}{(\vec{k}_\perp - \vec{q}_\perp)^2} \\
\vec{B} &=& \frac{\vec{k}_\perp - x\vec{q}_\perp}{(\vec{k}_\perp - x\vec{q}_\perp)^2} -  \frac{\vec{k}_\perp}{\vec{k}_\perp^2}
\end{eqnarray}

\paragraph*{Regulating the $2\rightarrow 3$ squared matrix-elements}
The divergence in the $q$ integration is removed by the requirement that this few-body matrix-element only applies to processes with $q>Q_{\textrm{cut}}$.
The collinear divergence when $k$ approaching $q$, $xq$ is regulated by including a gluon thermal mass. 
In practice, these collinear region will be further suppressed by the LPM effect.
The cross-section is obtained by integrating over the final state phase-space, where we have chosen to parameterize the three particle final state in terms of $k_\perp^2$, the rapidity of $k$ in the center-of-mass frame $y_k$, and the solid angle of the recoil medium particle.

\paragraph*{Soft limit: the Gunion-Bertsch approximation}
The result we obtained for the $g\rightarrow g+g$ and $q\rightarrow q+g$ channel has a soft limit that goes back to the well known Gunion-Bertsch form. 
By soft limit, we require the radiated gluon energy to be small enough such that $xq_\perp \ll k_\perp$.
Then, the splitting amplitudes for both $g\rightarrow g+g$ and  $q\rightarrow q+g$ are simplified into the same form,
\begin{eqnarray}
\overline{|M|}^2_{22} x(1-x)g^2 \frac{2(1-x+O(x^2))}{x} C_A \left(\frac{\vec{k}_\perp}{k_\perp^2}-\frac{\vec{k}_\perp-\vec{q}_\perp}{(\vec{k}_\perp-\vec{q}_\perp)^2}\right)^2
\end{eqnarray}
Neglecting the $O(x^2)$ terms in the splitting function, the result is the same as the improved verison of the Gunion-Bertsch cross-section in the full Boltzmann partonic transport model BAMPS,
\begin{eqnarray}
\overline{|M|}^2_{22} 8\pi C_A\alpha_s (1-x)^2 \left(\frac{\vec{k}_\perp}{k_\perp^2}-\frac{\vec{k}_\perp-\vec{q}_\perp}{(\vec{k}_\perp-\vec{q}_\perp)^2}\right)^2
\end{eqnarray}

\paragraph*{Results for the backward ($y_k < 0$) region}
We have mentioned in the beginning of the derivation that the condition $k_\perp^2 < x(1-x)\hat{s}$ restricts the splitting to be happen only for the parton moving in the $+z$ direction in the center-of-mass frame ($y_k > 0$).
For splitting that happens in the backward region, one need to use another set of diagrams, where the splitting comes from the parton that moves in the $-z$ direction in the center-of-mass frame. 
The result is, of course, similar to the previous results, but with the definition of $x$ and $q$ changed to $x = k^-/\sqrt{s}$, and $q = p_1-p_3$.

To combine the results that is obtained in different regions of phase space ($y_k > 0$ and $y_k < 0$), we follow [] and defines,
\begin{eqnarray}
\bar{x} &=& \frac{(k + |k_z|)}{\sqrt{s}} = \frac{k_\perp e^{|y_k|}}{\sqrt{s}}\\ 
\bar{q} &=& \Theta(y_k)(p_2-p_4) + \Theta(-y_k)(p_1-p_3)
\end{eqnarray}
which replaces the original $x$ and $q$ in our formula, and the resultant matrix-elements can be used for both forward and backward regions.

\paragraph*{Relation to the Bethe-Heitler limit of the AMY formalisim}
Now we display the connection between the $2\rightarrow 3$ cross section and the Bethe-Heitler limit of the AMY equation.
In the Bethe-Heitler limit, the AMY integral equation can be solved approximately by treating $1/\tau_f$ as the leading factor. 
One get the splitting rate for each different channels (denoting $\vec{a}/a^2$ as $\vec{\phi}_{a}$), 
\begin{eqnarray}
R_{q\rightarrow q+g}^{BH} &\propto& g^2 P_{qg}^{q(0)}(x) \int d k^2 d q^2 \mathcal{A}(q^2) \left\{
C_A\vec{\phi}_k\cdot\left(\vec{\phi}_k-\vec{\phi}_{k-q}\right) \right.\\\nonumber
&&+\left. (2C_F-C_A) \vec{\phi}_k\cdot\left(\vec{\phi}_k-\vec{\phi}_{k+xq}\right)
+ C_A \vec{\phi}_k\cdot\left(\vec{\phi}_k - \vec{\phi}_{k+(1-x)q}\right)
\right\}
\\
R_{g\rightarrow g+g}^{BH} &\propto& g^2 P_{gg}^{g(0)}(x) \int d k^2 d q^2 \mathcal{A}(q^2) \left\{
C_A\vec{\phi}_k\cdot\left(\vec{\phi}_k-\vec{\phi}_{k-q}\right) \right.\\\nonumber
&&+\left. C_A \vec{\phi}_k\cdot\left(\vec{\phi}_k-\vec{\phi}_{k+xq}\right)
+ C_A \vec{\phi}_k\cdot\left(\vec{\phi}_k - \vec{\phi}_{k+(1-x)q}\right)
\right\}
\\
R_{g\rightarrow q+\bar{q}}^{BH} &\propto& g^2 P_{q\bar{q}}^{g(0)}(x) \int d k^2  d q^2 \mathcal{A}(q^2) \left\{
(2C_F-C_A)\vec{\phi}_k\cdot\left(\vec{\phi}_k-\vec{\phi}_{k-q}\right) \right.\\\nonumber
&&+\left. C_A \vec{\phi}_k\cdot\left(\vec{\phi}_k-\vec{\phi}_{k+xq}\right)
+ C_A \vec{\phi}_k\cdot\left(\vec{\phi}_k - \vec{\phi}_{k+(1-x)q}\right)
\right\}
\end{eqnarray}
with the collision kernel $\mathcal{A} = g^2 T m_D^2/q^2(q^2+m_D^2)$. These expression looks drastically different from the incoherent rate computed using the cross-section derived in the previous section, however, we would like to show that they are the same once integration over $dk^2$ is performed.
Therefore, the incoherent rate we used in the Boltzmann equation indeed recover the Bethe-Heitler cases of the AMY integral equation.

To show this, we start from the $2\rightarrow 3$ rate formula using the matrix-elements from equation. 
One notice that the $\vec{k}$ integration can be shifted for each term without changing the integral value, and we shall rearrange the expression into the form in equation ().
Starting from the $q\rightarrow q+g$ channel, the rate in our Boltzmann equation is,
\begin{eqnarray}
R_{q\rightarrow q+g} &\propto& g^2 P_{qg}^{q(0)}(x) \int  \frac{f(p_2)dp_2^3}{2E_2(2\pi)^3} d q^2 \frac{g^4}{q^4}\\\nonumber
&&  \int d k^2\left\{
C_F\left( \vec{\phi}_{k-q}-\vec{\phi}_{k-xq} \right)^2
+ C_F\left( \vec{\phi}_{k-q}-\vec{\phi}_{k} \right)^2\right.\\\nonumber
&&\left.
- (2C_F-C_A)\left( \vec{\phi}_{k-q}-\vec{\phi}_{k-xq} \right)\cdot \left( \vec{\phi}_{k-q}-\vec{\phi}_{k} \right)
\right\}
\end{eqnarray}
Focusing on the three products (squares) of $\vec{\phi}$s under the $dk^2$ integration, we are going to expand the first term in each product and then shift the argument of the first $\vec{\phi}$ to $k$, which reads,
\begin{eqnarray}
R_{q\rightarrow q+g} &\propto& g^2 P_{qg}^{q(0)}(x) \int  \frac{f(p_2)dp_2^3}{2E_2(2\pi)^3} d q^2 \frac{g^4}{q^4}\\\nonumber
&&  \int d k^2\left\{
C_F\vec{\phi}_{k}\left( \vec{\phi}_{k}-\vec{\phi}_{k+(1-x)q} \right)
- C_F\vec{\phi}_{k}\left( \vec{\phi}_{k-(1-x)q}-\vec{\phi}_{k} \right)\right.
\\\nonumber
&&+ C_F\vec{\phi}_{k}\left( \vec{\phi}_{k}-\vec{\phi}_{k+q} \right)
- C_F\vec{\phi}_{k}\left( \vec{\phi}_{k-q}-\vec{\phi}_{k} \right)
\\\nonumber
&&\left.
- (2C_F-C_A)\vec{\phi}_{k}\cdot \left( \vec{\phi}_{k}-\vec{\phi}_{k+q} \right)
+(2C_F-C_A)\vec{\phi}_{k} \cdot \left( \vec{\phi}_{k-(1-x)q}-\vec{\phi}_{k+xq} \right)
\right\}
\end{eqnarray}
Next we flip sign of $q$ under the integration $q\rightarrow -q$,
and meanwhile, insert a $-\vec{\phi}_k +\vec{\phi}_k$ in the brackets of the last term,
\begin{eqnarray}
R_{q\rightarrow q+g} &\propto& g^2 P_{qg}^{q(0)}(x) \int  \frac{f(p_2)dp_2^3}{2E_2(2\pi)^3} d q^2 \frac{g^4}{q^4}\\\nonumber
&&  \int d k^2\left\{
2C_F\vec{\phi}_{k}\left( \vec{\phi}_{k}-\vec{\phi}_{k+(1-x)q} \right)
+ 2C_F\vec{\phi}_{k}\left( \vec{\phi}_{k}-\vec{\phi}_{k+q} \right)
\right.
\\\nonumber
&&
- (2C_F-C_A)\vec{\phi}_{k}\cdot \left( \vec{\phi}_{k}-\vec{\phi}_{k+q} \right)
+(2C_F-C_A)\vec{\phi}_{k} \cdot \left( \vec{\phi}_{k+(1-x)q} -\vec{\phi}_k \right) \\\nonumber
&&\left.+(2C_F-C_A)\vec{\phi}_{k} \cdot \left(\vec{\phi}_k-\vec{\phi}_{k+xq} \right)
\right\}
\end{eqnarray}
After this manipulation, the first and the second terms cancels the $C_F$ part of the third and fourth terms, and the rate now reads,
\begin{eqnarray}
R_{q\rightarrow q+g} &\propto& g^2 P_{qg}^{q(0)}(x) \int  \frac{f(p_2)dp_2^3}{2E_2(2\pi)^3} d q^2 \frac{g^4}{q^4}\\\nonumber
&&  \int d k^2\left\{
C_A\vec{\phi}_{k}\cdot \left( \vec{\phi}_{k}-\vec{\phi}_{k+q} \right)
+C_A\vec{\phi}_{k} \cdot \left( \vec{\phi}_k - \vec{\phi}_{k+(1-x)q}\right) \right.\\\nonumber
&&\left.+(2C_F-C_A)\vec{\phi}_{k} \cdot \left(\vec{\phi}_k-\vec{\phi}_{k+xq} \right)
\right\}
\end{eqnarray}
which is the exactly the same integration as the one obtained from the Bethe-Heitler limit of the AMY equation.
The equivalence between these two expressions of the $g\rightarrow g+g$ channel and the $g\rightarrow q+\bar{q}$ channel can also be shown similarly.

\paragraph*{Mass effect in the $2\rightarrow 3$ squared matrix-elements}
As a remark, putting heavy flavor mass directly into the these matrix-elements are certainly legitimate if one only focus on $2\rightarrow 3$ processes.
But once we want to approximate the effect of multiple scatterings:
$(n \rightarrow n+1) \approx (2 \rightarrow 3)(2 \rightarrow 2)\cdots(2 \rightarrow 2)\times \textrm{corrections}$, it is not advantages to put the mass effect into the $(2 \rightarrow 3)$ part, but into the last step of corrections, which is the approach we used.
Nevertheless, for completeness and for future reference, we also present the $2\rightarrow 3$ matrix-elements with mass effect.

First, even with a heavy flavor mass, we are still working under the assumption that $M \ll E$, and will only keeps terms when $M$ is making direct comparison to $k_\perp, q_\perp$.
The kinematics are now changed to,
\begin{eqnarray}
p_1 &=& (\sqrt{s}, 0, \vec{0})\\
p_2 &=& (0, \sqrt{s}, \vec{0})\\
k &=& (x\sqrt{s}, \frac{k_\perp^2}{x\sqrt{s}}, \vec{k}_\perp)\\
q &\sim& (-\frac{q_\perp^2\sqrt{s}}{s-M^2}, \frac{x(\vec{q}_\perp-\vec{k}_\perp)^2 + (1-x)k_\perp^2 + x^2M^2}{x(1-x)\sqrt{s}}, \vec{k}_\perp)
\end{eqnarray}
The approximation for the matrix-elements still proceeds as before: identifying the splitting amplitude and factorized it from the two-body collision amplitude.

For the massive version of the splitting amplitude.
Previously, we only include the diagonal contribution from the amplitude $\epsilon_{\lambda, \mu}(c) \bar{u}_s(a) \gamma^\mu v_{s'} (b)$ of $g\rightarrow q+\bar{q}$, but for massive quarks, we also need to include the off-diagonal component that accounts for flipping helicity. 
Also, with a mass term, 
\begin{eqnarray}
\sqrt{p\cdot \sigma} &=& \frac{p\cdot \sigma + M}{\sqrt{2(E+M)}} \approx \frac{p\cdot \sigma + M}{\sqrt{2E}} \\
\sqrt{p\cdot \bar{\sigma}} &=& \frac{p\cdot \bar{\sigma} + M}{\sqrt{2(E+M)}} \approx \frac{p\cdot \bar{\sigma} + M}{\sqrt{2E}}
\end{eqnarray}
where we have omitted the mass in the denominator since it only involves corrections of order $M/E$.
From this one can see that the previous calculation can be used for the massive case with the substitution  $a^\pm \rightarrow a^\pm +M$ and $b^\pm \rightarrow b^\pm +M$.
Then, the splitting amplitude becomes,
\begin{eqnarray}
\epsilon_{\lambda, \mu} \bar{u}_s(a)\gamma^\mu v_{s'}(b)&=& \frac{1}{\sqrt{2ab}}
\xi_s^T A_{ss'} \eta_{s'}\\
A_{\uparrow\uparrow} &=&
\delta_{\lambda L} 2b_z a^\perp_L + \delta_{\lambda R} 2a_z b^\perp_R + \frac{c^\perp_\lambda}{c^+} (a^\perp_L b^\perp_R - a^+ b^+) \\
A_{\downarrow\downarrow} &=&
-\delta_{\lambda L}2a_z b^\perp_L - \delta_{\lambda R}2b_z a^\perp_R - \frac{c^\perp_\lambda}{c^+} (a^\perp_R b^\perp_L - a^+ b^+) \\
A_{\uparrow\downarrow} &=&
\delta_{\lambda L} 2a^\perp_L b^\perp_L + \delta_{\lambda R} (a^+b^-+a^-b^+) - \frac{c^\perp_\lambda}{c^+} (a^+b^\perp + a^\perp b^+) \\
A_{\downarrow\uparrow} &=&
 \delta_{\lambda L} (a^+b^-+a^-b^+) + \delta_{\lambda L} 2a^\perp_R b^\perp_R - \frac{c^\perp_\lambda}{c^+} (a^+b^\perp + a^\perp b^+) 
\end{eqnarray}
where we have included the off-diagonal components.


\section{Comment on other implementations of the inelastic processes}
Before we proceed to validate transport model to theoretical calculations.
We find it beneficial to discuss other inelastic processes implementations for reader's references. 
We shall term the two approaches the ``coherence factor" approach and the ``blocking radiation" approach.
Through out the discussion in this section, one shall found that the ``coherence factor" approach qualitatively agrees with the power counting of the LPM suppression, although it only includes the effect of one medium scattering centers and its formulation is logarithmic dependent on the infrared cut-off.
While the blocking radiation approach do not reproduce the power counting of of the LPM effect.
These points shall be analyzed quantitatively later in the result section by comparing the ``energy loss" of a fixed energy parton simulated with each of these approach.
Now we shall briefly discuss these two other approaches and comment on their advantages and problems.


\paragraph*{The coherence factor approach}
This approach is first implemented in the improved Langevin equation, borrowing the idea from the higher-twist calculation of the medium induced radiation.
The higher-twist formula is derived the medium induced radiation of a high virtual parton, by including the interference of the hard production vertex and one medium scattering center.
The radiation rate reads,
\begin{eqnarray}
\frac{dN_g}{dx dk_\perp^2 dt} = \frac{\alpha_s P(x)\hat{q}_g}{\pi k_\perp^4} 2\left(1-\cos\frac{t-t_0}{\tau_f}\right), \tau_f = \frac{2x(1-x)E}{k_\perp^2}
\end{eqnarray}
Here the radiation rate is a time-dependent one due to the interference between the hard production (at time $t_i$) vertex and the medium (at time $t$).
Note that the interference factor cancels the collinear divergence and the only divergence comes from soft emission $x\rightarrow 0$.

The use of this formula in generating the first radiation is unambiguous. 
However, the treatment of multiple emissions can be very different in different models.
For example, one can compute the average number of emission by integrating this formula along the trajectory of the hard parton through and then samples samples the fluctuating number of emission with a Poisson distribution.
But of course, this would assume the parton energy is not significantly changed during the process, and it is not clear how the presence of more than one scattering center would change this picture.
Here we would like to discuss an method in dealing with multiple emission using the high twist formula in a time evolution manner.
The algorithm goes as follows:
\begin{itemize}
\item[1.] Choose an infrared cut-off for the gluon energy $x_c \propto T/E$, and a small enough time step $\Delta t$, so that the average number of emission is much smaller than $1$ to suppress multiple emission within $\Delta t$,
\begin{eqnarray}
\langle N_g \rangle = \Delta t \int_{x_c}^1 dx \int dk_\perp^2 \frac{dN_g(t-t_0)}{dx dk_\perp^2 dt} \ll 1.
\end{eqnarray}
\item[2.] Sample N according to a Poisson distribution with $\langle N_g \rangle$. For $\langle N_g \rangle \ll 1$, it is sufficient to sample the two leading cases of $N=0, 1$, as the probability to have more than 1 emission is negligible ($P_{N>2} = 1-e^{-\langle N_g \rangle}-e^{-\langle N_g \rangle}/\langle N_g \rangle = O(\langle N_g \rangle^2) \ll P_1 \ll P_0$).
\item[3.] If $N=0$ then propagate the parton to the $t+\Delta t$. If $N=1$, then sample the emission gluon's $x$, and $\vec{k_\perp}$ by the differential rate. {\bf Meanwhile, $t_0$ is set to $t$}, so that the next emission's probability will accumulated from zero again.
\item[4.] Proceed for the next time step.
\end{itemize}
We found that the key step here is resetting the clock $t_0 = t$ for the parton after every emission.
As a result, from the second emission, the time difference that appears in the interference factor $t-t_0$ are the one measuring between two medium scattering centers.
Therefore, we will not interpret this procedure as the high-twist rate (interference between initial hard vertex and one medium collision center) starting from the second emission;
instead we understand it as an ansatz, from the second emission, to treat medium-induced emission in a large medium, as it do not require any information to the production vertex.

Having this in mind, one wonders if this ansatz reproduces any in-medium radiation features predicted in the theory, considering it only includes one medium scattering center in the formula.
Certainly, it is not immediately clear to see what this procedure predicts  expect through simulations. 
But if one pondering on the meaning of the ``clock resetting" procedure, then, one realizes that the typical time separation are a time scale within which the emission probability is of order one,
\begin{eqnarray}
1 \sim \int_{t_0}^{t} dt\int_{x_c}^1 dx \int dk_\perp^2 \frac{dN_g(t-t_0)}{dx dk_\perp^2 dt}
\end{eqnarray}
With this key observation, after a few step of algebra, we are able to learn some qualitative feature of this ansatz.
Taking the soft approximation $P(x) \sim 2/x$, $\tau_f\sim 2xE/k_\perp^2$, and perform the time integral first, then the $k_\perp$ integral with limits from $0$ to $xE$.
\begin{eqnarray}
1 &\sim& 4\alpha_s\hat{q}\Delta t \int_{x_c}^1 \frac{dx}{x} \int \frac{dk_\perp^2}{k_\perp^4}\left(1-\frac{\sin(\Delta t/\tau_f)}{\Delta t/\tau_f}\right)\\
&=& \alpha_s\hat{q}_g \Delta t^3 \int_{\frac{\Delta t E x_c}{2}}^{\frac{\Delta t E}{2}} 
\frac{du}{u^2} \frac{u^2 \mathrm{Si}(u) -2u + \sin(u) + u\cos(u)}{u^2}\\
&=& \alpha_s\hat{q}_g \Delta t^3 \left.\frac{u^3\mathrm{Ci}(u)-3u^2\mathrm{Si}(u) - u^2 \sin(u) +3u-\sin(u) - 2u\cos(u)}{3u^3}\right|_{\frac{\Delta t E x_c}{2}}^{\frac{\Delta t E}{2}} 
\end{eqnarray}
This final integral of $x$ (reparametrize by $u = xE\Delta t/2$) would have been logarithmic divergent if we had not cut it at $x_c$ at the lower bound.
The complicated integrated result has the following expansion at small $u$: $\frac{1}{18}(6\ln(u)+6\gamma_E - 17)$ and decay $0$ at infinite therefore a good proxy is to use the small-$u$ expansion but cut-off the upper bound of $u$ at its zero, and finally
\begin{eqnarray}
1 &\sim&  \frac{\alpha_s\hat{q}\Delta t^3}{3}\ln\frac{2}{ x_c E \Delta t } \propto (g^2 T \Delta t)^3 \ln\frac{2}{ x_c E \Delta t }
\end{eqnarray}
Now, it is clear that this procedure of implementing multiple emission inside the medium resets the clock in the interference factor every $1/g^2T$ up to certain logarithm dependence on the infrared cut-off, which is the order of the elastic collision mean-free-path.
Put this estimated $\Delta t$ back into the interference factor $2(1-\cos(\Delta t/\tau_f))$, one indeed find that the radiation spectrum will be strongly suppressed if the formation time is much greater than $\Delta t\sim \lambda_{el}$.

This suppression certainly mimic some property of the in-medium LPM effect, but is introduced by a very different mechanism.
Remember that the LPM effect is the suppression of single particle emission rate through multiple collision with the medium, without any information about how subsequent emissions are correlated. 
While the interference factor approach mimic the effect of the LPM suppression through correlation between subsequent emissions.
This will introduce several problems: 
\begin{itemize}
\item[1.] The correlation between subsequent emissions is in fact physics beyond leading order and requires new type of diagrams to be computed.
\item[2.] As we have seen, this procedure is affected by the choice of the infrared cut-off. Though its dependence is very weak, it is still a dependence that we try to avoid.
\end{itemize}

\paragraph*{The blocking radiation approach}
Another approach we also found in literature has been termed ``the blocking radiation approach" in the thesis, however, we would like to show that this approach is wrong in the sense that it suppresses radiation with $\tau_f > \lambda_{\textrm{inel}}^{\textrm{incoh}}$ but not $\tau_f > \lambda_{\textrm{el}}$, which is a power of $\alpha_s$ off.

In this approach, the splitting is also first generated through an incoherent processes at time $t_0$, and then the self-consistently determination the formation time by elastic broadening. 
But its LPM suppression is introduced by requiring no other radiation is allowed from this radiator within the time from $t_0$ to $t_0 + \tau_f$.
This clearly introduces a correlation between subsequent emission, while the LPM effect concerns only the one particle emission rate.
While in our approach, the suppression is implemented by accepting the process with probability $\sim \lambda_{el}/\tau_f$, and more importantly, other emissions are unaffected by the current ``imaginary splitting" as most of them would be rejected without causing any physical effect.

A closer investigation reveals bigger problem.
Moreover, this ``blocking radiation" approach effectively reduces every $\tau_f/\lambda_{inel}$ incoherent emission to one, resulting only an overall reduction in the radiation spectrum without changing its shape.
And the suppression factor $\lambda_{inel}/\tau_f$ is different from the expected one, this is in fact an order of $\alpha_s$ wrong as the mean-free-path of the incoherent radiation rate contains one more power of $\alpha_s$ than $\lambda_{el}$.
\chapter{Transport mode application to heavy-flavor evolution}
In this chapter, we shall start to focus on applying the LIDO transport model to the heavy-flavor section, and discuss in detail how a transport model fits into the complex dynamics that heavy flavor particles undergoes in the heavy-ion collision environment.

Heavy flavor is a charming probe for the medium created in heavy-ion collisions. 
Its large mass guarantees an negligible thermal production contribution at least for present LHC top beam energies for heavy-ion program (there are estimates that thermal contribution can play a row in future FCC collider).
Therefore, heavy flavors are almost always created in initial hard processes. 
By hard processes, it includes both the hardest few body collisions, as well as the associated high-virtuality parton evolution.
Their dilute population also suppresses the chances that they annihilates against their anti-particles during the medium evolution.
As a result, heavy flavors are created at relatively early stages of the heavy-ion collision and experienced the entire medium evolution and encodes valuable information about the medium.
On the theory side heavy flavor has a rich variety of physics interests. 
At high $p_T$, the evolution of heavy flavor particles merges into the context of jet dynamics and jet energy loss study; at intermediate $p_T$ the mass hierarchy predicted for the medium modifications;
and low $p_T$, heavy flavor is one of the key messengers for the thermalization processes inside QGP due to its long relaxation time compared to light partons.
On the experimental side, their unique flavors, masses, and decay modes also give us the chance to reconstruct heavy flavor (heavy meson / baryon) observables directly.

There are two types of heavy quarks that are most relevant for nowadays hard-probes study, the charm quark with mass $1.3$ GeV, and the bottom quark with mass $4.2$ GeV.
The reason top quark ($173$ GeV) is out of our discussion is due to its extremely short life time ($\sim 5\times 10^{-25} \approx 0.15$  fm/$c$ in the rest frame) so it barely interacts with the QGP before it decays predominantly into bottom quarks.
Even though there has also been proposal for taking advantage of this short time scale to probe the temporal structure of the QGP in its early stages, we shall focus on the charm and bottom flavor in this thesis.

To build a comprehensive simulation framework for the fate of heavy quarks in relativistic heavy-ion collisions requires a multi-component and multi-stage modeling.
Here we summarize this simulation framework in the following flow chart as a guide line for this section, figure \ref{fig:flowchart}.

\begin{figure}
\centering
\includegraphics[width=.8\textwidth]{flowchart.pdf}
\caption{hh}
\label{fig:flowchart}
\end{figure}

In the previous two chapters, we have introduced the left branch of this flow chart, which is a relativistic viscous hydrodynamics based simulation for the bulk medium evolution.
The right branch is a model for the hard parton (heavy flavor) evolution.
The heavy quarks are produced in hard processes.
The hard processes contains both the hard matrix-element as well as the virtuality evolution of the parton (the DGLAP evolution or termed vacuum-like evolution).
One complication in the presence of medium, one complication is that vacuum-like evolution starts to occupied the same space-time of the medium induced processes at low-virtuality.
And should one interfaces the two calculations is still an interesting topic under debate.
One of the many obstacles is that multiple emissions / branchings are treated very differently in the two calculations.
For the vacuum-like evolution, the evolution variable is the virtuality scale with the space-time information integrated out, while the transport model evolves the systems in time, with virtuality integrated out below a certain scale.
There are many recent progress in both theory developments and newly design event-generators to solve this problem.
In this chapter, we shall focus on one possible solution to interface the two in section 1.
The in medium propagation of parton requires the medium properties as input. 
To zeros order of approximation, we specify the medium with its flow velocity and the equilibrium temperature, although the viscous hydro provides far more off-equilibrium information. 
This is discussed in section 2.
The heavy flavor hadronization model is introduced in section 3, which is a previously developed model that interpolates high-$p_T$ fragmentation processes and low-$p_T$ in medium recombination production of heavy hadrons.
Finally, in section 4, we also briefly introduced how this simulation framework of open heavy flavor can be coupled to the quarkonium evolution, but for more details, please refers to the thesis work [].

\section{Initial production of heavy flavor}
The initial hard processes are computed using perturbative QCD based or related Monte-Carlo event generator.
The general set up for such a computation in proton-proton collision is the factorization theorem \ref{fig:factorization}.
Where the perturbative QCD calculation provides the physics at short distance (a hard scale $Q^2$): the partonic matrix-elements $\hat{\sigma}_{ij\rightarrow kl}$.
The partonic configuration inside the proton characterized by the parton-distribution function (PDF) and the hadronization of the final state parton into hadrons are non-perturbative inputs.
Although these non-perturbative objects general cannot be computed from first principal and has to be extracted from measurements, their scale evolution in $Q^2$ can be described in perturbative QCD, knowns as the DGLAP evolution equations.
This scale dependence comes from that the exclusive process  $i+j \rightarrow k+l$ described by the fixed order matrix-elements is always modified by the parton branching and virtual correction that is of order $\alpha_s \ln Q^2/\mu^2$.
These corrections takes into account the fact that the initial high-virtuality parton $i$ (or $j$) may also come from a splitting processes of low virtuality parton $i'$ (or $j'$) from the proton, including virtually correction. 
Similarly, the final state high virtuality parton $k$ (or $l$) may also splits into a low virtuality parton $k'$ (or $l'$) before it becomes a hadron, including virtual correction.
The same argument also applies to partons $i', j', k', l'$. 
Eventually, one gets a series of contribution where although each term contains an additional power of $\alpha_s$, but is magnified by $\ln Q^2/\mu^2$ if there is a large gap between the hard scale $Q^2$ and the PDF scale $\mu^2$ at which it is measured.
The DGLAP evolution equations resum these logarithmic contributions systematically to increase the predictive power of the perturbative calculation of the inclusive cross-section.
Moreover, a useful parton-shower picture can be built from the process and with a probabilistic interpretation and Monte Carlo technique, one can mimic the exclusive final states from these sequences of parton branching processes.

In our studies, we have tried to use both inclusive cross-section program as well as Monte-Carlo event generator to initialize the heavy quark production.
Next we will explain their merits and draw backs for our purposes.

\paragraph{Initialize from inclusive cross-section program}
We use FONLL (Fixed Order Next to Leading Log) to generate the inclusive production cross-section of heavy flavor at partonic level.
The FONLL program is a combination of the fixed order (NLO) massive matrix-elements and a massless resummation program.
It predicts the single inclusive differential cross-section $d^2\sigma/dydp_T$. 
Then, one can sample the initial heavy quark's momentum from this differential cross-section.
The advantages are
\begin{itemize}
\item[1.] Sampling / weighting the inclusive cross-section is fast.
\item[2.] Provide interface to nuclear PDFs.
\item[3.] First principal approach for proton-proton collision.
\end{itemize} 
However, there are also disadvantages, 
\begin{itemize}
\item[1.] It only predicts single particle distribution and one cannot initialize the correlation of the $Q-\bar{Q}$ pair. This is not a problem for open heavy flavor observables, but would be a problem for quarkonium study.
\item[2.] The space-time picture of the production process is lost. One do not know whether the virtuality evolution takes a finite amount of time. So we have to assume charms quarks are produced at proper time $t=0^{+}$, which is always before the in-medium transport. But later we will see that from event generator simulations, the evolution can take a finite amount of time, overlapping with the QGP evolution.
\item[3.] Lacking the exclusive final state. Again, it is not a problem for open-heavy flavor study. But to describe full jets, one really needs an exclusive partonic final state of the virtuality evolution to initilize the partonic transport model.
\end{itemize}

\paragraph{Initialize from Monte-Carlo event generator.}
\begin{figure}
\centering
\includegraphics[width=\textwidth]{factorization.png}
\caption{
%https://indico.cern.ch/event/680421/contributions/3096162/attachments/1697092/2731944/J.Huston_Introduction_to_QCD_from_an_LHC_perspective-02.pdf
}
\label{fig:factorization}
\end{figure}

Another approach to generate initial hard process is to use high energy Monte-Carlo event generator, such as Pythia [].
Pythia implements the leading order (LO) matrix-elements for hard QCD processes, including LO production of heavy flavor particles,
$g+g\rightarrow Q+\bar{Q}$ and $q+\bar{q}\rightarrow Q+\bar{Q}$.
A parton shower will be generated based on the hard processes.
In fact, in high energy collisions, the LO production of heavy flavor is only a fraction of the total heavy flavor cross-section, the rest of them are created in the parton showers from the so-called gluon splitting and flavor creation processes.
The former corresponds to a situation where the heavy flavor pair comes from a final state gluon splitting; and the latter produces the pair in initial state gluon splitting and is put-on shell by the hard scattering.
These contributions also mimic certain pair correlations with non back-to-back angular correlations.

The disadvantage is of course the event generator is not a first principal computation of the heavy flavor production.
However, there are many benefits of having an exclusive final state.
\begin{itemize}
\item Although the parton shower is evolved as a function of virtuality $Q^2$. A qualitative space-time picture can be built by measuring the formation time of each branching by $2x(1-x)E/k_\perp^2$. In this way, we will see that the vacuum radiation off a heavy quark can last for a long time.
\item With the space-time picture, it is easier to analyze what kind of vacuum branchings should be modified in the presence of a medium.
\item Can be used to initialize the full transport model to study jets. 
\end{itemize}

\paragraph{Heavy flavor production baseline} In the course of our study, we used both the FONLL program and the Pythia event generator to initialize the momentum space. 
It is important to check that first whether they predict similar heavy flavor production in nuclear collisions and proton-proton collisions, and second if they provide good descriptions of the experimental measurement in proton-proton collisions.
In the upper plot of figure \ref{fig:pythia-fonll}, we compare the $p_T$ differential cross-section of $p+p\rightarrow c$ from FONLL calculations (lines) and from Pythia simulations (symbols), and for Pb+Pb collision (red) and p+p collision (blue) at the LHC energy $\sqrt{s}=5.02$ TeV.
For p+p system, we use the CT10 parton distribution function and for Pb+Pb system, the nuclear modification to the parton distribution is using the EPS09 parametrizaiton.

We note that the absolute value of the cross-sections are different.
Fortunately, the observables that we are interested in nuclear collisions are always ratios such as the nuclear modification factor $R_{AA}$, and the momentum-space anisotropy of heavy meson $v_n$ which is dimensional less. 
Therefore the shape of the spectra is the key feature we would like to compare between the two calculation and we have rescaled the FONLL curve.
We see that their shapes are very similar.
When calculation $R_{AA}$, we are dividing the cross-section from nuclear collision to that of the proton-proton collision.
To see how much the modulation in $R_{AA}$ is contributed by the nuclear PDF effects, we take the ratio between the AA and pp results in the upper plots to get an ``$R_{AA}$" for the nuclear PDF effects in the bottom plots.
We can see that FONLL and Pythia simulation predicts consistent modulation: the initial production AA spectra of charm quark at low-$p_T$ is suppressed compared to the pp spectra, due to the shadowing effect of the small-$x$ gluon. 
At higher $p_T$, the ratio increase and slightly shoots over  unity, this is because the there is an anti-shadowing region of the gluon at larger-$x$.

\begin{figure}
\centering
\includegraphics[width=.8\textwidth]{pythia-vs-fonll.png}
\caption{}
\label{fig:pythia-fonll}
\end{figure}

\section{Interfacing vacuum shower with in-medium transport}
Interfacing the vacuum shower that evolves with virtuality and the transport equation that evolves with time is a difficult task. 
We shall provide a reasoning for the prescription we use following some of the recent developments [].
Start by considering a splitting of a hard parton which is created at the boundary of a brick medium.
This splitting has a formation time depending on its transverse momentum (or virtuality) $\tau_f \sim 2x(1-x)E/k_\perp^2$.
It is very likely that the radiated gluon (or the quark) interacts with a scattering center in the medium, denoted by the crossed circle.
If the virtuality of the splitting is very large, then the formation time is small.
The argument is that scatterings at time $t$ that is well separated from the formation time $\tau_f \ll t$ are independently from the initial vacuum-like splitting process; while scatterings within $\tau_f$ can change the branching probability.
So, we determine whether the vacuum branching should be modified in medium by the number of scatterings before $\tau_f$.

\begin{itemize}
\item For a branching with large virtuality (left of figure \ref{fig:vac-med-interface}) that $N = \tau_f/\lambda \ll 1$ (translates into $k_\perp^2 \gg \alpha_s  \omega T$). 
There chance for the medium interaction to modify the vacuum branching is negligible.
\item Hold the energy of the radiaion, while decrease its transverse momentum (middle of figure \ref{fig:vac-med-interface}) so that $N = \tau_f/\lambda \lesssim 1$ ($k_\perp^2 \gtrsim \alpha_s  \omega T$). 
Now, there is order one scatterings with the medium. 
The transverse momentum of the gluon is modified a little but still dominated by the initial virtuality.
The probability for the branching should also be modified.
For example, a higher twist expansion in terms of $1/k_\perp^2$ takes into account the effect of one interaction with the medium.
\item Further decreasing the initial virtuality of the branching (right of figure \ref{fig:vac-med-interface}) until $N = \tau_f/\lambda \gg 1$.
Now the final transverse momentum has to be determined self-consistently to be $k_\perp^2 \sim \sqrt{\hat{q}\omega} \sim \alpha_s\sqrt{\omega T^3}$. 
When this happens, the initial virtuality of the splitting is completely dominated by the medium effects (a medium scale at $\sqrt{\hat{q}\omega}$). 
And the branching probability should be replaced by a medium-induced calculation.
\end{itemize}
Summarizing the two extreme regions:
Vacuum branchings with $k_\perp^2 \gg \alpha_s \omega T$ is not modified, while branchings with $k_\perp^2 \sim \sqrt{\hat{q}\omega}$ should be treated as medium-induced process instead of vacuum radiation.
It is therefore natural to use $k_\perp^2$ and the fraction of $k_\perp^2$ that is contributed by the medium broadening $\Delta k_\perp^2$ to separate the medium-induced radiation and the vacuum-like radiation.
Since medium induced radiations treated by the transport equation always have $k_\perp^2 = \Delta k_\perp^2$,  our interfacing prescription is then to simply cut-out the vacuum branchings generated by Pythia in the region $k_\perp^2 <  \Delta k_\perp^2$ (this is also known as the vetoed region in the literature).

\begin{table}[h]
\centering
\caption{Treating Pythia branchings inside the medium}
\begin{tabular}{ccc}
\hline
\multirow{2}{*}{Formed inside medium} & $k_\perp^2 > C \Delta k_\perp^2$ & Unmodified\\
 & $k_\perp^2 < C \Delta k_\perp^2$ & Removed\\
Formed outside medium & Unmodified & \\
\hline
\end{tabular}
\label{tab:med-vac}
\end{table}

Since we are interested in heavy quarks,


\begin{figure}
\includegraphics[width=.35\textwidth]{largeQ.pdf}\includegraphics[width=.35\textwidth]{mediumQ.pdf}\includegraphics[width=.35\textwidth]{smallQ.pdf}
\caption{}
\label{fig:vac-med-interface}
\end{figure}


\section{Coupling transport dynamics to an evolving medium}

\section{Heavy-flavor hadronization and hadronic stage}

\section{Coupling open-heavy flavor to quarkonium evolution}

\chapter{Bayes parameter extraction of complex model}
In the past three chapter, we have discussed the modeling details of the heavy flavor transport in the environment of the relativistic heavy-ion collisions.
And it is evident that the simulation requires a complex model with complex inputs.
To use such a model to acquire quantitative understanding about the properties of the quark-gluon plasma, we apply advanced statistical tools known as the Bayes analysis.
Let me first list out the problems one encounters in the model-to-data comparison in the field of heavy-ion collisions.

\paragraph{Complex dynamical model with high-dimensional inputs}
Taking the bulk model as an example, it includes an initial condition, a dynamical model for pre-equilibrium stage, the viscous hydrodynamics, particlization of the hydrodynamic energy momentum tensor and the hadronic decays and rescatterings.
Each of the module takes both physical parameters and modeling parameters.
Eventually, one can have more than 10 parameters.
To generate one event with a model itself is already computationally intensive, while to be able to compare to experimental measurements that average over centrality with a good control of statistical uncertainty, $O(10^4)$ minimum-biased events needed to be generated.
As a result, a na\"ive linear grid sweep over the 10-dimensional parameter space is certainty not feasible.
To solve this problem of high-dimensional input complex models requires advanced parameter set design method and reliable interpolation schemes.

\paragraph{Model uncertainty and correlation among parameters}
We are mostly interested in those ``physical parameters / quantities'', for example, the transport coefficients $\eta/s(T)$ and $\zeta/s(T)$ of the medium.
Other parameters, such the switching time between the pre-equilibrium free-streaming dynamics and the hydrodynamics, and the temperature at which one particlize a hydrodynamic field description into a hadronic ensemble are less interested in the physical sense.
The appearance of these matching parameters are simply due to we do not have a single ``global model'' that works reasonably well under all circumstances.
However, one should not think less of these model parameters, because the quantify a significant fraction of the modeling uncertainty.
The model uncertainty can affect the interpretation of the experimental data, and any quantitative statement we draw about those physical quantities from a model-to-data comparison.
For example, the  bulk particles anisotropic flows, though proved to be very sensitive to the shear viscosity, are also sensitive to the initial eccentricity of the initial conditions.
Therefore, the exacted $\eta/s$ becomes correlated with the choice of initial condition models and its parameters.
The model uncertainty is hard problem, and one can either try to reduce it by improving the model or quantify it when compared to data to prevent an over-interpretation.

\paragraph{A high-dimensional model output / observables}
The inclusion of more observables also helps to break the model degeneracy and parameter correlation.
But it can be tricky to quantify the quality of agreement between model and data with a collection of observables (a high dimensional output).


The Bayesian analysis framework is introduced to the field by the earily works of and the PhD thesis of Jonah Bernhard [] who has successfully applied the tool to the extraction of the initial condition topology and temperature dependent QGP transport coefficient.
This chapter provides a concise description of the Bayesian analysis and for full details please refer to the thesis of Jonah Bernhard [].

We abstract the general task of a model-to-data comparison into the following form,
\begin{itemize}
\item A complex model $M$, with input parameters organized as a vector $p$ of dimensional $m$.
\item There are certain prior belief on the reasonable range of each parameters, known as the prior probability distribution $\mathrm{Prior}$.
\item The experimental data are organized as an observation vector $y_{\exp}$ of dimensional $n$
\item Goal: to infer the probability distribution of $p$, given the model $M$, the measurements $y_{\exp}$, and the prior belief $\mathrm{Prior}$. The resultant distribution is called the posterior probability distribution $\mathrm{Posterior}$.
\end{itemize}

\section{The design of parameters}
The first step is an understanding of the behavior of the model $M$.
This is done by sampling the high-dimensional input parameter space with a finite number of design parameter set.
Mathematically, this $N$ set of parameter vectors of length $n$ forms a so-called design matrix $\mathbf{D}$,
\begin{eqnarray}
\mathbf{D}_{N\times n} = 
\begin{bmatrix}
p_{11} & p_{12} & \cdots & p_{1n}\\
p_{21} & p_{22} & \cdots & p_{2n}\\
\vdots & \vdots & \ddots & \vdots \\
p_{N1} & p_{N2} & \cdots & p_{Nn}
\end{bmatrix}
\end{eqnarray}
where the first index is the label of different parameter set, and the second index labels different parameters.
We want an efficient sampling of the design points, so that the number of $N$ is manageable while still provide a representative coverage of the entire parameter space.
We used Latin-Hyper-Cube sampling method, it generate a random design but subject to the following constraint:
\begin{itemize}
\item The marginalized distribution on any parameters is a uniform distribution. This is different from a grid design, where the marginalized distribution are spiky delta function on the grid points.
\item The minimum distanced between any two point in the parameter space is maximized. This is different from a complete random design, where the points may from tight cluster or sparse regions.
\end{itemize}
Usually, for a well behaved model, the number of design points needed for a good interpolation increases linearly with the number of parameters, in contrast to the exponential increase with number of parameters for a grid design.

Once the design is made, the time consuming part is then to perform the full model calculation on each points and organize all the computed observables into an observation matrix,
\begin{eqnarray}
\mathbf{Y}_{N\times m} = 
\begin{bmatrix}
y_{11} & y_{12} & \cdots & y_{1m}\\
y_{21} & y_{22} & \cdots & y_{2m}\\
\vdots & \vdots & \ddots & \vdots \\
y_{N1} & y_{N2} & \cdots & y_{Nm}
\end{bmatrix}
\end{eqnarray}
where the first index is the label of different parameter set, and the second index labels different observables.
The design matrix $\mathbf{D}$ and the observations matrix $Y$ forms the training data to build a general interpolator for the mapping from the parameter space to the observable space $M: \vec{p} \rightarrow \vec{y}(\vec{p})$.

\section{Data reduction}
The model $M$ is an $n$-dimesional vector to an $m$-dimensional vector mapping. 
One can certainly construct an independent array of $m$ scalar mappings, and interpolate each of them using the training data.
However, this na\"ive construction does not make use of the intrinsic correlations / structures that is already presented in the training data, and can be very inefficient for practice usage.
For example, consider the observations contain two values of $R_{AA}$ and $v_2$ for $D$-meson, and usually the larger the $R_{AA}$ is, the smaller the $v_2$, and thus an anti-correlation is expected.
If one build interpolators for $R_{AA}$ and $v_2$ independently, their uncertainty is also going to be independent, and the intrinsic correlation is overlooked.
However, if one choose to interpolates the linearly combinations $a R_{AA} \pm b v_2$, then a wise choice of $a, b$ can largely reduces the correlation between these two ``new" observables.

The principal component analysis (PCA) is the systematic way to implement this idea.
The original vectors of observables are transformed into the principal-component (PC) space, with each PC a specific linear combination of the original observables, so that the covariances between the newly defined observables (the PCs) vanish.
Mathematically, this is the same as finding the singular value decomposition (SVD) of $\mathbf{\tilde{Y}}$. 
$\mathbf{\tilde{Y}}$ is the standardized observation matrix $\mathbf{Y}$,
\begin{eqnarray}
\tilde{y}_{ij} = \frac{y_{ij} - \mu_j}{\sigma_j}
\end{eqnarray}
with $\mu_j$ and $\sigma_j$ the mean and the standard deviation of column $j$.
Then the SVD proceeds as,
\begin{eqnarray}
\tilde{\mathbf{Y}} = \mathbf{U} \mathbf{\Sigma} \mathbf{V}
\end{eqnarray}
And the principal components are defined as the components after the $V$ transformation.
\begin{eqnarray}
z = \mathbf{V}y
\end{eqnarray}
It is evident that the covariance matrix of the $z$ observables are diagonalized,
\begin{eqnarray}
\mathrm{Var}(z_i, z_j) = \frac{1}{N}V_{ii'}\tilde{Y}_{ki'}V_{jj'}\tilde{Y}_{kj'} = \frac{1}{N}V\tilde{Y}^T\tilde{Y}V^T = \frac{1}{N}\mathbf{\Sigma}
\end{eqnarray}
so that the PCs are orthogonalized.
One more benefits is that, suppose the variance in $\Sigma$ has been ordered from maximum to minimum.
For data with pronounced structures, often the first few principal components take account the majority of the data variance.
And for practical usage, a truncation of the number of PCs already gives a good representation of the original data, and this greatly reduces the computations for large number of observables.
In the end, one can always go back from the PC space to the original physical space by $y = V^{-1} z$.

\section{Model emulator}
With limited information on a finite number of design points contained in the matrices $D$ and $M$, the original mapping is approximated by a model emulator (a surrogate model) using the Gaussian Process (GP).
The Gaussian Process provide a non-parametric interpolation for scalar function and also works for high dimensional input.
We shall let the readers refer to [] for the technique details related to the GP and only summarize the basic ideas.

Taking a uni-variate case as an example, given an array of input and an array of scalar output, the common way to interpolate the the data are, e.g., polynomial interpolation.
However, polynomial interpolation only uses local information of the grid, and its performance can be sensitive to the error of the output (e.g. statistical fluctuation in the measurement / simulation).
Moreover, it is hard to generate to a high-dimensional Lain-hypercube design because the design points are not arranged on a grid.
A GP does not make any assumption on the function form of the interpolation, but infer the output at a given input point based on how the output at the present point correlates with other points that has been given.
Mathematically, one assumes that the array of output $\vec{y}^*$ to be predicted at input $\vec{x^*}$ forms a multi-variate normal distribution with the output $\vec{y}_{\textrm{train}}$ at the training points $\vec{x}_{\textrm{train}}$,
\begin{eqnarray}
\begin{bmatrix}
\vec{y}^* \\
\vec{y}_{\textrm{train}}
\end{bmatrix}
\sim
\mathcal{N}\left(
\begin{bmatrix}
\mu^* \\
\mu_{\textrm{train}}
\end{bmatrix},
\begin{bmatrix}
\mathbf{\Sigma}(\vec{x}^*, \vec{x}^*)& \mathbf{\Sigma}(x^*, x_{\textrm{train}}) \\
\mathbf{\Sigma}(\vec{x}_{\textrm{train}}, \vec{x}^*)& \mathbf{\Sigma}(\vec{x}_{\textrm{train}}, \vec{x}_{\textrm{train}})
\end{bmatrix}
\right)
\end{eqnarray}
Here, without a loss of generality the mean vectors $\mu^*$ and $\mu_{\textrm{textrm}}$ are often set to zero.
The $\mathrm{\Sigma}$'s forms the co-variance matrix, and each of them has the same shape of the outer product of its two argument vectors.
Its matrix-element (the kernel function) often takes a squared exponential form,
\begin{eqnarray}
\Sigma_{ij} = k(x_i, x_j) = \sigma^2 \exp\left(-\frac{(x_i-x_j)^2}{2l^2}\right) 
\end{eqnarray}
Where the auto correlation is $\sigma^2$ at the same input, and correlation decays exponentially with squared separation of the input points.
In such a way, points that are close in inputs will also be close in outputs.

Because, we have known the the value of the training output for sure, one obtains the probability distribution of $\vec{y}^*$ by conditioning the training output on a set of fixed values,
\begin{eqnarray}
\vec{y}^* \sim &&\mathcal{N}\left(
\mathbf{\Sigma}(\vec{x}^*, \vec{x}_{\textrm{train}} )
\mathbf{\Sigma}^{-1}(\vec{x}_{\textrm{train}}, \vec{x}_{\textrm{train}} )\vec{y}_{\textrm{train}},\right.\\\nonumber
&&\left.
\mathbf{\Sigma}(\vec{x}^*, \vec{x}^*) - 
\mathbf{\Sigma}(\vec{x}^*, \vec{x}_{\textrm{train}} )
\mathbf{\Sigma}^{-1}(\vec{x}_{\textrm{train}}, \vec{x}_{\textrm{train}} )
\mathbf{\Sigma}(\vec{x}_{\textrm{train}},\vec{x}^*)
\right)
\end{eqnarray}
Note that the conditional multivariate normal distribution is still a normal distribution, with modified mean and covariance matrix.
One can easily check that if the predicting input approaches one of the training input, the distribution of the output approaches an delta function (as the limit of a narrow Gaussian) at the training output.

\paragraph{Inference with uncertainty quantification} The GP does not provide a single estimation of the out, as does the polynomial interpolation, but infer the probability distribution of the predicted outputs by giving both the mean estimation and the covariance.
This is a huge advantage of the Gaussian Process as it quantifies its own interpolation uncertainty.
The variances of the predicted outputs are the diagonal elements of the covariance matrix.

\paragraph{Hyperparameters and training} We have not discussed how to determine the value of the parameters in the kernel function $k$ yet, which are the variance $\sigma^2$ and the correlation length. 
The squared exponential form is not the only possible kernel function, more sophisticated choices with more parameters are designed for varies problems.
These extra parameters in the kernel function are known as hyper-parameters (denoted as a vector $\vec{\theta}$), and should in principle, also be treated as unknown parameters in the calibration.
However, a common practice to reduce the complexity by fixing these of hyper-parameters at a set of ``optimal values'' by minimize the loss function $\mathcal{L}$ of the training processes,
\begin{eqnarray}
\mathcal{L} = -\ln p(\vec{y}|\vec{\theta}) = \frac{1}{2}\ln \det \mathbf{\Sigma}(\vec{\theta})  + \frac{1}{2}\vec{y}^T \mathbf{\Sigma}(\vec{\theta})^{-1} \vec{y} + \frac{N}{2}\ln(2\pi)
\end{eqnarray}
where $\vec{y}$ is the (PCA transformed) training data, and $N$ is the number of training points.
By minimizing this quantity, an optimal level of interpolation is achieved.
This minimization process of fixing the hyper-parameters in the GP emulators is called a ``training'' processes.

\paragraph{Validation} Though the training process includes certain penalty for over-fitting the data, whether the trained GP has over-fitting problem can only be checked validation.
An validation can be done by performing the model calculation at novel points in the parameter space that is not ``learned" by the GP, and compare the trained GP's prediction $y_i \pm \sigma_i$ to the model calculation $y_{\textrm{validate}, i}$.
If an emulator is trained to work properly, then the standardized deviation $(y_i - y_{\textrm{validate}, i})/\sigma_i$ should be approximately a standard normal distribution.

\paragraph{Multivariate inputs and outputs} The GP formulation can be generalized to higher dimensional inputs easily by specifying a multidimensional kernel function.
For high dimensional outputs, one first applies the PCA analysis introduced in the previous section and the build individual GPs for each of the first $N_{PC}$ principal components that take most of the data's variance.

\section{Bayes' theorem and Markov chain Monte Carlo}
With the model emulator $M$ (we are using the same symbol as the model, but one should always remember that the emulator is only a fast surrogate of the original model and comes with uncertainty), we proceed to the essential of the statistical analysis, which is the Bayes' theorem.
The Bayes' theorem is the quantitative way to update the knowledge of model parameters with empirical observations,
\begin{equation}
\mathrm{Posterior}(\vec{p}|M, \vec{y}_{\textrm{exp}}) \propto \mathrm{Likelihood}(\vec{y}_{\textrm{exp}}|M, \vec{p})\times\mathrm{Prior}(\vec{p}).
\end{equation}
It states that the posterior probability distribution of parameters, given model and experimental measurements, is proportional to the likelihood to describe the experiments with the model using this set of parameters, times the prior belief of the distribution of the parameters.
The likelihood is function is often assumed to be a multivariate Gaussian,
\begin{eqnarray}
\mathrm{Likelihood}(\vec{p}) &=& (2\pi)^{-\frac{m}{2}} (\det|\Sigma|)^{-\frac{1}{2}} \exp\left\{-\frac{1}{2}\Delta \vec{y}^T \mathbf{\Sigma}^{-1} \Delta \vec{y}\right\}, \\ 
\Delta \vec{y} &=& \vec{y}(\vec{p}) - \vec{y}_{\textrm{exp}}
\end{eqnarray}
where the $\vec{y}(\vec{p})$ is the model emulators' prediction at parameter point $\vec{p}$, $m$ is the number of observables.
The prior distribution is often a multi-dimensional uniform distribution within a reasonable range. 
The covariance matrix contains various sources of uncertainty from both theory and experimental side.

\paragraph{A model dependent statement} first one notice that the posterior is always defined with a given model, and therefore even the extraction of theoretically well defined quantities can be affected by the use of different dynamical modeling.
The ultimate solution is of course the improvement of model's physical accuracy.
Or using a flexible model or models with different (but reasonable) assumptions to extract the same quantity and estimate the level of theoretical uncertainty.

\paragraph{The covariance matrix} covariance matrix is decomposed into different contributions which will be briefly introduced
\begin{eqnarray}
\mathbf{\Sigma} = \mathbf{\Sigma}_{\textrm{stat}} + \mathbf{\Sigma}_{\textrm{sys}} + \mathbf{\Sigma}_{\textrm{emu}} + \mathbf{\Sigma}_{\textrm{trun}}
\end{eqnarray}
\begin{itemize}
\item The statistical covariance  takes the diagonal form, $\mathbf{\Sigma}_{\textrm{stat}} = \delta_{ij}\sigma_{\textrm{stat}, i}^2$. And $\sigma_{\textrm{stat}, i}$ is the experimental statistical uncertainty.
\item The experimental systematic uncertainty can correlation across different $p_T$ bins and different centrality. However, due to the lack of information on how these correlation looks like, we parametrize the correlation to a Gaussian type in the $\ln(p_T)$ space, and a constant factor for different centrality,
\begin{eqnarray}
\mathbf{\Sigma}_{\textrm{stat}} = \sigma_{\textrm{sys}, i}\sigma_{\textrm{sys}, j} C_{ij} \exp\left\{-\frac{1}{2 L_{\textrm{corr}}^2} \left(\ln\frac{p_{T, i}}{p_{T, j}}\right)^2 \right\}.
\end{eqnarray}
$L_{\textrm{corr}}$ is a the correlation length. 
$C_{ij}$ is unity if the quantities $i$ and $j$ are from the same centrality of the same observables; it is zero if they are different observable. 
For $R_{AA}$, $C_{ij}$ is a constant between $0$ and $1$ for different centrality. This mimics the fact that $R_{AA}$ from different centrality always uses the same $p$-$p$ baseline measurements and therefore has a fraction of its uncertainty correlated.
For momentum anisotropy, $C_{ij}$ is still zero for between different centralities.
We can only guess the $\ln p_T$-correlation length and centrality correlation right now which introduces an ambiguity in the model-to-data comparison.
We hope that future measurements will provide more information on the covariance structure of the published systematic error bars.
\item The emulator covaraince $\mathbf{\Sigma}_{\textrm{emu}}$ is the prediction covariance of the GPs in the PC space and then transformed into the physical space.
\item Finally, the truncation covaraince $\mathbf{\Sigma}_{\textrm{trun}}$ take those less important principal components that are not being emulated by GPs into account. These fraction of the data variance is computed in t the PC space with a diagonal form and transformed back to the physical space.
\end{itemize}

\paragraph{Marginalize the posterior distribution} The resultant posterior distribution a function of $n$-parameter spaces.
To answer questions such as what is the probability distribution of one parameter folded with uncertainty from other parameter, one looks at the marginalized distribution with the other $n-1$ parameters integrated out.
This is done by a Markov chain Monte Carlo (MCMC) sampling of the posterior function and obtains an ensemble of $n$-dimensional walkers whose distribution thermalizes into the posterior distribution.
The maginalized distribution is then the distribution of the projected ensemble onto one dimension.
Similary, a marginalization of the joint distribution of two or more parameters can be obtained similarly.




\chapter{Results}
\label{chapter:results}
In this chapter, we apply the advanced statistical tools to the heavy-flavor transport model and extract the heavy quark transport coefficients.
I would like to present this in a two step processes to show the improvements of the lastest extraction.



\paragraph{A list of experimental data}
\begin{center}
\begin{table}[h]
\caption{ALICE dataset}\label{table:ALICE-obs} 
\begin{tabularx}{\columnwidth}{XXX}
\hline 
 Observables & Centrality & Reference\\ 
\hline 
$D$-meson $v_2$ & 30-50\% & {Acharya:2017qps}\\ 
\hline 
Event-engineered $D$-meson $v_2$ & 30-50\% & {Grosa:2017zcz}\\ 
\hline 
$D$-meson $R_{AA}$ & 0-10, 30-50, 50-80\% & {Acharya:2018hre}\\
\hline 
\end{tabularx}
\end{table}
\begin{table}[h]
\caption{CMS dataset}\label{table:CMS-obs} 
\begin{tabularx}{\columnwidth}{XXX}
\hline 
Observables & Centrality & Reference\\ 
\hline 
D${}^0$-meson $v_2$ & 0-10, 10-30, 30-50\% & {Sirunyan:2017plt}\\ 
\hline 
D${}^0$-meson $R_{AA}$ & 0-10\%, 0-100\% & {Sirunyan:2017xss}\\ 
\hline 
B${}^{\pm}$-meson $R_{AA}$ & 0-100\% & {Sirunyan:2017oug}\\ 
\hline 
\end{tabularx}
\end{table}
\end{center}

\section{Lessons from earlier extractions of $\hat{q}_Q$}
In an earlier publication [], we used a linearized Boltzmann model with the coherence factor approach to implement the LPM effect.
The heavy quark initial momentum dsitribution is obtained from the FONLL calculation.
We have already commented on the advantages and disadvantages of these choices.
Two different set of nuclear PDF $EPPS$ and $nCTEQ15$ are used to represent the uncertainty from the cold nuclear matter effect in the $\hat{q}$ extraction.

Regarding model parameters, the one parameter for the perturbative elastic and inelastic scatterings is controlled by $1/3 < \mu < 4$ in the running coupling. 
There is an additional pure diffusion process with a diffusion constant $\kappa_{NP}$ parametized to peak at low temperature and low energy, in order to mimic certain non-perturbative coupling between a low energy probe and the medium near $T_c$,
\begin{eqnarray}
\kappa_{NP} = T^3 \kappa_D \left(x_D + (1-x_D)\frac{1\textrm{ GeV}{}^2}{ET}\right).
\end{eqnarray}
The $0<\kappa_D<8$ parameter is the overall strength of the diffusion, and the $0<x_D<1$ controls the degree of energy-temperature dependence.
One can see that in the heavy quark limit $M\rightarrow \infty$, this parametrization becomes independent of mass.
An additional parameter is the in-medium energy loss starting time $\tau_0$ that is allowed to be tuned between $0.1$ fm/$c$ to $1.0$ fm/$c$ (before the onset of hydrodynamics).
The reason is we lack a quantitative description of the production of color charge in the initial stages.
This starting time is a simple approximation that interactions is only turned on after $\tau_0$ when the color carries is assumed to approach a Boltzmann distribution.


The design of the four dimensional parameter space  $(\tau_0, \mu, \kappa_D, x_D)$ has 80 design points.
The computation is carried on the distributed computing system Open Science Grid [] using about a million CPU hours.
The observabes on which we calibrated are listed in tables \ref{table:ALICE-obs} and \ref{table:CMS-obs}. 
Including, $p_T$ dependent $D$-meson nuclear modification factor $R_{AA}$ and $p_T$ dependent (event-shape-engineered) azimuthal anisotropy $v_2$.
CMS measurements of the $B^{\pm}$-meson $R_{AA}$ is also included to constrain the mass dependence of the transport coefficients.

\begin{figure*}
\includegraphics[width=.49\textwidth]{observables_design.pdf}
\includegraphics[width=.49\textwidth]{observables_posterior.pdf}
\caption{Left: the prior, i.e. the full range of calculations in parameter space. Right: the posterior, i.e. observables sampled from model emulators after calibration. In both figures, blue (green) lines are calculations with {\tt EPPS} ({\tt nCTEQ15np}) nuclear PDF.}\label{plots:deisgn_posterior_obs}
\end{figure*}

The prior and the posterior of the observables before and after the calibration is shwon in figure \ref{plots:deisgn_posterior_obs}.
blue stands for using {\tt EPPS} nuclear PDF and green stands for using the {\tt nCTEQnp}  nuclear PDF.
We found that the model after the calibration provide a good description of $R_{AA}$ and $v_2$ at the intermediate $p_T$ of the ALICE experiments.
But it does not reproduce the fast uprising shape of $R_{AA}$ at high-$p_T$ of the CMS experiment.
In addition, the model seem to under estimate the high-$p_T$ $v_2$ of the $30-50\%$ centrality bin measured by CMS.
The model is able to explain the correlation between the D-meson $v_2$ and the event-shape, though there are still large fluctuation in the data.
The use of different nuclear PDFs has a negligible effect on $v_2$, but does affect the $R_{AA}$ at small and large $p_T$.
Another thing worth noting that is that the $D$ and $B$ meson $R_{AA}$ are described at the same time.

\begin{figure}
\centering
\includegraphics[width=.7\textwidth]{posterior.pdf}
\caption{Marginalized postrior probability distribution of model parameters. Diagonal plots show the marginalization on a single parameter. Off diagonal plots show the pair correlation between parameters. Blue (Geen) lines and lower (upper) off diagonal plots correspond to the extraction using EPPS (nCTEQ15np) nuclear PDF.}\label{plots:posterior}
\end{figure}

The inferred posterior probability distribution of the parameters is shown in the figure \ref{plots:posterior}.
The diagonal plots show single parameterized distributions, and the off-diagonal ones displays the two-parameter correlations.
We split the results that use different nuclear PDFs into the upper (EPPS, green heat map and lines) and lower (nCTEQ15np, blue heat maps and lines) triangles.
One notices that results from different nuclear PDF are consistent within the uncertainty; therefore, from now on I shall not stress on any differences between these two set of results, but combine them into a single distribution to fold in the PDF uncertainty.
The favored parameters are $\mu \sim 0.6$ and $\kappa_D \sim 0.4$, indicating a large in-medium $\alpha_s$ and a small additional diffusion.
The typical value of the $\alpha_s$ is, in fact, so large that let one worried about the use of weakly-couple based approaches.
For example, $\alpha_s(0.6\pi T)$ at $T=300$ MeV is 0.67, corresponding to $g \approx 3$. 
And the screening mass $m_D \sim 3.6 T$ is even larger than the average energy of the thermal partons $3T$. 
In the discussion of the next section, we will see that this problem can be slightly alleviated, once we use the improved implementation of the LPM effect and implement a separation of soft-modes into the diffusion constant, though $g$ is still large.

\begin{figure}
\includegraphics[width=\columnwidth]{qhat_p_T.pdf}
\caption{Posterior range of the heavy quark transverse momentum broadening parameter $\hat{q}$ from Equation \ref{eq:qhat}. The results include the uncertainty from using different nuclear PDFs. Blue boxed region is for bottom quarks and red slashed region for charm quarks.}\label{fig:posterior_qhat}
\end{figure}

\paragraph{Transport coefficients} In this analysis, the heavy quark transport coefficient $\hat{q}$ is computed from adding up the momentum broadening from both the scattering and the parametric diffusion,
\begin{eqnarray}\label{eq:qhat}
\hat{q} &=& 2T^3\kappa_D\left(x_D + (1-x_D)\frac{\textrm{GeV}^2}{ET}\right) + \hat{q}_{\textrm{el}}.
\end{eqnarray}
In a perturbative definition of the transport coefficients, the inelastic process does not contribute to heavy quark transport coefficient at leading order. 
In figure \ref{plots:posterior_qhat}, the 90\% credible region of $\hat{q}$ is shown as a function of temperature at a fixed energy (left), and as function of energy at a fixed temperature (right).
Results for charm (red) and bottom (blue) quarks are labeled by different colors.
The mass difference only causes a small difference in $\hat{q}$.

\begin{figure}
\includegraphics[width=\columnwidth]{qhat_compare.pdf}
\caption{Posterior range of the heavy quark transverse momentum broadening parameter $\hat{q}$. The shaded region indicates a previous extraction using the improved-Langevin model [].}\label{fig:compare_qhat}
\end{figure}

\paragraph{Comparison to results from an improved-Langevin model}
The same transport coefficient is also extracted using the improved-Langevin model [].
It includes a diffusion modeling of the elastic interacton, a high-twist single gluon emission rate, and a similar routine to implement multiple radiations.
This model is then coupled to the same medium as the one used here and compared to the same set of observables as this work does.
The resultant posterior (for charm quark only) is shown as the shaded region in figure \ref{fig:compare_qhat}.
We see that the $\hat{q}$ extracted using the two models only overlap at the boundary of the credible region.
Their difference is comparable to the uncertainty band of either model, while both models provide a reasonable description of the data.
This suggests the theoretical uncertainty that comes from the assumption between the probe and the medium is a significant one.
The ability to tune a switching scale parameter in the new model intends to include this type of theoretical uncertainty.

\section{Calibration with the improved transport model}
Finally, we apply the improved model to the extraction of the heavy quark transport coefficients.
As a summary of the improvements:
\begin{itemize}
\item A more sophisticated implementation of the LPM effect to reduce modeling uncertainty of the radiative process;
\item An interpolation of the diffusion picture and the scattering picture to take into account modeling uncertainty.
\item Restrict the use of few-body matrix-elements to only large-$Q$ interactions.
\item Separating the high-virtuality evolution and the low-virtuality transport equation at a medium scale.
\end{itemize}

\paragraph{Model parameters}
In the new analysis, we try to include as many theoretical uncertainty as possible, therefore we have much more parameters than the two previous efforts.
They are listed in table \ref{table:new:prior}.
\begin{itemize}
\item The first parameter is again the energy loss starting time $\tau_i$.
In this analysis, we are comparing to data at two collision energies and the hydrodynamic starting time $\tau_0$ varies from $1.2$ fm/$c$ to $0.6$ fm/$c$.
To account for this differences, we use the ratio $\xi = \tau_i/\tau_0$ as the single parameter for both energies.
\item The second parameter is switching scale parameter $c$ in $Q_{\textrm{cut}}^2 = c m_D^2$, it is allowed to be vary between $0.1 and 10$.
\item The third parameter $R_v$ controls the matching condition between the vacuum-like radiation and the medium-induce radiation $\Delta k_\perp^2 = R_v k_\perp^2$.
In chapter \ref{chapter:coupling}, we have been simply using $\Delta k_\perp^2 = k_\perp$, i.e., $R_v = 1$.
But since this is only a qualitative argument, we allow $R_v$ to vary between 0 and 7. 
At $R_v = 0$, the vacuum-like radiation is completely forbidden once it interacts with the medium; for $R_v \gg 1$, the vacuum-like radiation is effectively unmodified.
\item The $\mu$ parameter controls the in-medium strong coupling $\alpha_s(\max\{Q, \mu\pi T\})$.
\item The rest of the six numbers $K,a,b,p,q, \gamma$ parametrizes a correction to the weakly coupled transport coefficient $\hat{q} + \Delta\hat{q}$, $\hat{q}_L + \Delta\hat{q}_L$,
\begin{eqnarray}
\Delta\hat{q} &=& \frac{K T^3}{\left[1+\left(a\frac{T}{T_c}\right)^p\right]\left[1+\left(b\frac{E}{T}\right)^q\right]}, \\
\Delta\hat{q}_L &=& \left(\frac{E}{M}\right)^\gamma \frac{\Delta\hat{q}}{2}
\end{eqnarray}
$0 < K < 15$ is the overall magnitude of the correction. 
The deviation from the $T^3$ dependence and the energy dependence are parametrized using two dimensionless combinations $T/T_c$, and $E/T$.
The $\gamma$ parameter varied from $-1$ to $1$ allows the correction to be anisotropic.
Note that such a construction goes back to an isotropic diffusion when velocity approaches zero ($E\rightarrow M$).
\end{itemize}
\begin{table}
\centering
\caption{Prior range of parameters}\label{table:new:prior}
\begin{tabular}{ccc}
\hline
Symbol & Description & Range \\
\hline
$\xi = \frac{\tau_0}{\tau_{\textrm{hydro}}}$ & Energy loss starting time & (.1, .9) \\
$c = \frac{Q_{\textrm{cut}}^2}{m_D^2}$ & Soft / hard switching scale & $(.1, 10.)$ \\
$R_v = \frac{k_\perp^2}{\Delta k_\perp^2}$ & Vacuum / Medium mathcing scale & $(0.14, \infty)$\\
$\mu$ & Running $\alpha_s$ stops at $Q = \mu\pi T$ & $(.6, 10)$ \\
$K$ & Magnitude of $\Delta \hat{q}/T^3$ & $(0, 20)$\\ 
$p$ & \multirow{2}{*}{$E$-dependence of $\Delta \hat{q}/T^3$} & $(-2, 2)$\\ 
$a$ &  & $(-1, 1)$\\ 
$q$ & \multirow{2}{*}{$T$-dependence of $\Delta \hat{q}/T^3$}  & $(-.5, 3)$\\ 
$b$ &   & $(-.5, 3)$\\ 
$\gamma$ & $\Delta \hat{q}_L = (E/M)^\gamma\Delta \hat{q}_L$  & (-1, 1)\\ 
\hline
\end{tabular}
\end{table}

\paragraph{Design and prior} We sampled 250 design points and 50 validation points. 
As a remark, we chosoe to give $\ln c, \ln R_v, \ln \mu, \ln a$ and $\ln b$ an uniform design and a uniform prior.
Therefore, the original parameter will have a non-uniform design and prior distribution.
The reason is that these parameters either causes a logarithmic slow change in the model and its prior uncertainty is large that it is allowed vary by orders of magnitude.
For example, the $\mu$ parameter enters the logarithmic running of $\alpha_s$ and we can rewrite the maximum possible $\alpha_s$ as,
\begin{eqnarray}
\alpha_{s,\max}(T) = \frac{2\pi}{9}\frac{1}{\ln(\mu) + \ln(\pi T/\Lambda_{\textrm{QCD}})}
\end{eqnarray}
Therefore, we choose assign a uniform prior to $\ln(\mu)$ so that $\alpha_s$ also varies notable within the prior range.
As for the $c$ parameter, it controls the boundary of the scattering and diffusion approximation, within a reasonable range of $c$, the diffusion dynamics and the scattering dynamics results in a similar amount of energy loss as has been shown in section [], and difference is seen in more subtle observables, therefore it is also assumed to be flat in the $\log$ space.
For $R_v, a$ and $b$, we did this to allow an order of magnitude change in the parameters as we want to test both the large and small number limits.

\begin{figure}
\centering
\includegraphics[width=.4\textwidth]{qhat_prior.png}\includegraphics[width=.4\textwidth]{ER_prior.png}
\caption{•}
\label{fig:new:design-qhat}
\end{figure}

Combining $\mu, K, p, q, a, b$ and $\gamma$, the prior region of the heavy quark transport parameters are plotted function of temperature and energy in figure \ref{fig:new:design-qhat}. 
On the left, 250 design's $\hat{q}$ as function of temperatures are shown  (using charm mass for demonstration).
Each subplot varys energy from $1.4$ GeV, $11.4$ GeV to $101.3$ GeV.
The prior range of $\hat{q}$ varies over a order of magnitude.
On the right of the figure, we show the ratio $2\hat{q}_L/\hat{q}$. 
This ratio is one if the transport parameters are isotropic.


The computations of the model on both the design points and the validation points are performed on the NERSC super-computing system using over two million CPU hours.
The prior observables are shown in figure \ref{fig:new:obs_prior_LHC} at LHC energy $\sqrt{s}$ = 5.02 TeV and in figure \ref{fig:new:obs_prior_RHIC} at RHIC energy $\sqrt{s} = 200$ GeV.
The dataset at the LHC energy is the same as the previous analysis.
THe dataset the RHIC energy are the central-peripheral ratio $R_{CP}$, and D meson $v_2$. 
$R_{CP}$ is defined as the $N_{\textrm{bin}}$ normalized ratio between the yield in a smaller centrality class $C$ to a larger centrality class $P$,
\begin{eqnarray}
R_{\textrm{CP}} = \frac{dN_\textrm{C}/dp_T N_{\textrm{bin,P}}}{dN_\textrm{P}/dp_T N_{\textrm{bin,C}}}
\end{eqnarray}
Using the nuclear data as a reference has the advantage of canceling certain theoretical uncertainties, such as the nuclear PDF (neglecting its impact-parameter dependence) and possible modifications to the initial production mechanism by the nuclear environment.
We found that even varying wildly the parameters, the very low-$p_T$ $R_{CP}$ at RHIC energy is not covered by the calculation. 
This indicates the model will have to be improved in this $p_T$ region, possible a more up-to-date hadronization model.
For now, we only include the STAR $R_{CP}$ above $5$ GeV in the calibration.

\begin{figure}
\centering
\includegraphics[width=\textwidth]{obs_prior_LHC.png}
\caption{•}
\label{fig:new:obs_prior_LHC}
\end{figure}

\begin{figure}
\centering
\includegraphics[width=\textwidth]{obs_prior_RHIC.png}
\caption{•}
\label{fig:new:obs_prior_RHIC}
\end{figure}

\paragraph{Emulator validation} 
The validation is performed by comparing the emulator trained on the 250 design points to the actual calculation on the 50 validation points.
We visualize the validation in figure \ref{fig:new:validation}.
In the top row, the emulated $v_2$ (left) and emulated $R_{AA}$ (right) are compared with the model calculations, and the data from different experiments and centrality has been labeled by different colors.
The emulated values strongly correlates with the true calculations around the the $y=x$ line.
Most points hit off the line, meaning the emulator is not 100\% accurate.
To see if the uncertainty is accounted for, we scatter plot the emulator's prediction uncertainty ($1\sigma$, $y$ axis) versus the absolute deviation between the prediction and the calculation (the $x$ axis).
The dashed line defines the shaded region where the true deviation is large than $\pm 3\sigma$.
We found that over $99\%$ of the prediction are within the $3\sigma$ region.
Therefore, for most cases the emulator correctly estimates its uncertainty  and therefore prevents over-fitting.

\begin{figure}
\centering
\includegraphics[width=.8\textwidth]{validation.png}
\caption{•}
\label{fig:new:validation}
\end{figure}

\paragraph{Co-variance matrix} Now we specify the covariance matrix to define the likelihood function, and marginalize the posterior distribution function using the MCMC technique.
The general structure of the covariance matrix has been introduced in the previous section, here we use a centrality decorrelation factor $C=0.5$, and chose two different $\ln p_T$ correlation length $0.3$ and $1$ to investigate the sensitivity of the calibration to the form of co-variance matrix

\begin{figure}
\centering
\includegraphics[width=\textwidth]{obs_posterior_LHC.png}
\caption{•}
\label{fig:new:obs_posterior_LHC}
\end{figure}

\paragraph{Posterior observables} The global level of agreement between the calibrated model and the data are shown in figure \ref{fig:new:obs_posterior_LHC} for LHC energy, and figure \ref{fig:new:obs_posterior_RHIC} for RHIC energy.
The both the $D$-meson and the $B$-meson $R_{AA}$ at the LHC energy is well described by the calibrated model, while $v_2$ is systematically below the data.
At the RHIC energy, the $v_2$ has a better agreement; the magnitude of $R_{CP}$ (note that we only calibrated on the three data points above $p_T=5$ GeV) is correct, though the shape is too flat compared to the data.

\begin{figure}
\centering
\includegraphics[width=\textwidth]{obs_posterior_RHIC.png}
\caption{•}
\label{fig:new:obs_posterior_RHIC}
\end{figure}

\begin{figure*}
\centering
\includegraphics[width=\textwidth]{posterior.png}
\caption{•}
\label{fig:new:posterior}
\end{figure*}

\paragraph{Posterior distribution of parameters} Figure \ref{fig:new:posterior} shows the posterior distribution of the parameters.
The $\ln\mu$ parameter has a evident peak around $.99$, which correspond to $\mu \approx 2.7$.
Note that the means the preferred in-medium coupling is smaller than the extracted one from the previous analysis.
This is because the improved implementation of the radiave process produces more energy loss than the old one when the coupling is large, so a smaller value of $\alpha_s$ is needed to explain the data.
The resulting posterior of $\alpha_s$ (figure \ref{fig:new:posterior-alphas}) suggests an effective in-medium coupling from 0.20 to 0.33 at $T=300$ MeV.
Though the preferred $\alpha_s \sim 0.3$ is smaller than the previous one, $g\sim 2$ is still large, but this extracted coupling constant is much closer to phenomenological values used by other studies [].
The energy loss starting is preferred to be about half of the hydrodynamic starting time.
The switching scale parameter do not have a strong preference as long as it is not too large, which is consistent with our model construction that physical processes should be weakly depends on this switching scale between diffusion and scattering modeling.
The matching parameter $R_v$ is not very well constrained, as we have shown it only affects very high-$p_T$ observables.

\begin{figure}
\centering
\includegraphics[width=.7\textwidth]{alpha_s_posterior.png}
\caption{•}
\label{fig:new:posterior-alphas}
\end{figure}

We plotted the 90\% credible range of the posterior transport parameters $\hat{q}, \hat{q}_L$ as the red bands in figure \ref{fig:new:posterior-qhat}.
The figure is arranged similarly to its prior plot.
The transport parameters is nicely constrained compare to its prior range (gray bands), and is comparable to the earlier extraction by the JET Collaboration \footnote{Note that the JET Collaboration extracts the light quark $\hat{q}$. However, at $p_T = 10$ GeV, the mass effect of the charm is small and these two numbers should be comparable.}.
We also present a first extraction of the longitudinal transport parameter. 
And it is quite anisotropic, similar to the pure weakly coupled expectation.

For the heavy quark spatial diffusion constant, since it is related to $\hat{q}$ in the zero momentum limit. 
Such an extraction is essentially an extrapolation of our parametrization, and can be sensitive to the detailed choice of the ansatz. 
Nevertheless, the extraction is compared to varies lattice calculations [] in  figure \ref{fig:new:posterior-Ds}.
Our extraction (red band for 90\% credible region) agrees with the lattice calculation in the static limit of the heavy quark; while calculation using dynamical charm quark gives a much lower value.


\begin{figure}
\centering
\includegraphics[width=.5\textwidth]{qhat_posterior.png}\includegraphics[width=.5\textwidth]{ER_posterior.png}
\caption{•}
\label{fig:new:posterior-qhat}
\end{figure}


\begin{figure}
\centering
\includegraphics[width=.7\textwidth]{Ds_posterior.png}
\caption{•}
\label{fig:new:posterior-Ds}
\end{figure}

\paragraph{Prediction with high-likelihood parameter set}


\chapter{Conclusion}
\label{chapter:conclusion}
In this dissertation, I have focused on understanding the transport properties of heavy flavor in the strongly coupled quark-gluon plasma applying model-to-data comparison methodology, aiming for model improvements and uncertainty quantification.

A prerequisite for the study is an ``accurate'' modeling of the physical ingredients to be tested.
It is not so trivial to model the heavy quark transport that is coupled to an event-by-event fluctuating and evolving medium.
On the one hand, this is because the finite medium-induced radiation formation time at high energy is much greater than the mean-free-path in semi-classical transport equations, and can be comparable to the medium evolution time scales.
On the other hand, there are two competing pictures regarding the heavy-quark-to-medium coupling: a weakly coupled picture modeled by scatterings, and a strongly coupled picture whose dynamics is often modeled by diffusion equations.
We developed a transport model for hard parton propagation in a near equilibrium plasma. 
An improved treatment of the LPM effect is implemented and it is shown to reduce to theoretical baseline calculations in idealized infinite static medium limit, and capture qualitative features in a finite and evolving medium.
The model also treats the large and small momentum transfer processes with different strategies of few-body scattering and diffusion (plus diffusion-induced radiation), which grants a flexible parametrization of diffusion-like deviations from the leading order weakly coupled approach.

The transport in a hot QGP stage is embedded in a more general ``transport'' picture including the initial production and high-virtuality evolution, hadronization near the transition temperature and hadronic dynamics and decay.
We identity a matching problem between the high-virtuality evolution and medium-induced evolution.
Currently, a unified formulation that smoothly connects the virtuality shower and the in-medium shower is still missing, and we use a separation of phase-space to terminate vacuum showers at a scale ($Q^2$) where they are likely to receive similar amounts of medium modification to the transverse momentum ($\Delta k_\perp^2 \sim Q^2$).
The exact location of the separation scale is then treated as an uncertainty of the model.

Finally, we apply a Bayesian analysis to infer the model parameter distribution by comparing to heavy flavor measurements at both RHIC and the LHC.
The model parameters include uncertainties such as the in-medium coupling strength, energy loss starting time, matching scale between vacuum and medium-induced shower, diffusion versus scattering model, as well as parametrized deviations from weakly coupled calculations.

\begin{figure}
\singlespacing
\centering
\includegraphics[width=.8\textwidth]{qhat_posterior_3D.png}
\caption[This figure shows the main result of this dissertation. The 90\%]{This figure shows the main result of this dissertation. The 90\% credible transport coefficient $\hat{q}/T^3$ extracted for the charm flavor is displayed on the two dimensional landscape of energy and temperature. The JET Collaboration extraction of light quark transport coefficients at $p=10$ GeV \cite{Burke:2013yra} (blue diamonds) and two lattice calculations of the momentum diffusion coefficient $\kappa$ ($\hat{q}=2\kappa$) \cite{Ding:2012sp,Banerjee:2011ra} are plotted for comparison.}
\label{fig:conclusion}
\end{figure}

We highlight the progress of this work in the conclusion figure \ref{fig:conclusion}.
It visualizes the $90\%$ credible region of the energy and momentum dependence of the heavy quark momentum diffusion transport parameter $\hat{q}$ scaled by $T^3$.
We found $\hat{q}/T^3$ gradually increases with $\ln E$ and displays an enhancement near the critical temperature.
Studying heavy flavor helps to connect the knowledge of in-medium transport properties at very high momentum (light quark limit) and very low momentum (static sources limit).
At relatively high momentum $p\sim 10$ GeV, it is consistent with the light quark transport parameter extracted by the JET Collaboration (blue).
At low momentum, it is consistent with lattice calculations in the heavy quark limit (black).
Future study with improved flavor dependence may be needed to understand the impact of using the ``heavy' limit in a dynamical model.
In the present calibration, the effective in-medium strong coupling constant is about $0.3$, and only contributes to a small fraction of the extracted $\hat{q}$ parameter.
The rest comes from the parametric contribution whose origin can be either perturbative or non-perturbative; either way, it suggests the necessity to model beyond leading order physics.

In conclusion, a transport model with perturbative parton evolution with a parametric probe-medium interaction term provides reasonable description to the open-heavy flavor observables measured at RHIC and LHC, while the level of accuracy needs to be improved.
Extracted heavy quark transport coefficients as function of energy and temperature are consistent with early phenomenological studies and lattice calculations.

The present model accuracy is still not enough to make the best use of future high-precision hard probe measurements in heavy-ion collisions.
We therefore list a few necessary points of improvements which may help to reduce or estimate the theoretical and modeling uncertainties.
\begin{itemize}
\item An interpolation formula between vacuum and medium-induced radiation: a calculation that connects virtuality evolution with in-medium time evolution will help to eliminated the matching scale uncertainty. Even though its effect is not strong for the present observables and $p_T$ range, it may impact more delicate jet observables.
\item Correlations among multiple emissions in the presence of a medium. We have been neglecting the correlation among subsequent emissions in the ``modified transport model''. In the infinite medium limit, this is because the probability of overlapping emissions scales as $\tau_{1,f} R(\omega_2) \sim \tau_{1,f} \alpha_s/\tau_{2,f}$ which is suppressed by $\alpha_s$. But this higher order effect can be important since the phenomenological $\alpha_s$ is not small. There are ongoing studies on this topic \cite{Arnold:2015qya,Arnold:2016kek,Arnold:2016mth,Arnold:2016jnq}.
\item Off equilibrium corrections to the linearized transport equation. One essential assumption in the linearized transport model is that medium partons follow a local thermal distributions, even though the hydrodynamics used includes viscous corrections. 
In fact, the viscous correction and the momentum space anisotropy can be very large at early times of the hydrodynamic evolution. 
One needs to understand how these off-equilibrium effects change the interpretation of the transport coefficients one extracts assuming full thermal equilibrium of partons.
\item Dynamical hadronization model and improved treatment of energy loss in the hadronic stage.
Our current hadronization model has the problem of pinching long distance physics into a sudden process. 
At low-$p_T$, the sudden recombination model breaks the detailed balance of the transport model and treating the recombination model in a dynamical way would be desirable.
At high-$p_T$, the problem is more severe, as the hadronization time scale is dilated by the large boost. 
Moreover, the hadronic system near $T_c$ is still very dense, and it is inconsistent to apply the vacuum fragmentation function at $T\sim T_c$.
One possible solution for those high-$p_T$ heavy quarks (the recombination process is negligible) is to continue their partonic transport into the hadronic phase, and finally apply the vacuum fragmentation function when the system is dilute enough.
Meanwhile, one can also study the energy loss in the dense hadronic system to extend the extracted transport parameter to the region below $T_c$.
\item A calibration with simultaneous tuning of the bulk and hard sectors. The bulk medium calibration is performed by a separate analysis. With future high precision hard probe measurements, a simultaneous calibration of of both soft and hard sector would be interesting.
For example, we found that the number of binary collision as a function of centrality is quite sensitive to the proton shape modeling in the Monte-Carlo Glauber model. 
The sensitivity of hard probe production to the number of binary collision may help to improve the proton shape modeling in the soft sector. 
In turn, a better calibrated medium may help to reduce the uncertainty in the hard parton energy loss study.
\end{itemize}

\begin{appendices}
\chapter{Few-body matrix-elements}
\label{app:ME}
This section provides the detailed $2\leftrightarrow 2$ and $2\leftrightarrow 3$ matrix-elements we used in the transport model.
The $2\leftrightarrow 2$ results are standard and we do not re-derive them here.
The $2\leftrightarrow 3$ cross-sections are more complicated and a detailed derivation is attached to show the approximations we made for readers reference.  

\subsection{$2\leftrightarrow 2$ processes}
The two-body scatterings between quarks, anti-quarks and gluons are standard and we quote the results from existing references \cite{RevModPhys.59.465}.
For a light parton scattering, we keep only $\hat{t}$-channel contribution, the $\hat{s}$ and $\hat{u}$ channel contribution are suppressed at high energy.
\begin{eqnarray}
\overline{|M_{q_1q_2\rightarrow q_1q_2}|^2} &=& \frac{64\pi^2 \alpha_s^2}{9} \frac{s^2+u^2}{t^2} \\
\overline{|M_{gg\rightarrow gg}|^2} &\approx& 72\pi^2 \alpha_s^2 \frac{-su}{t^2}
 \\
\overline{|M_{qg\rightarrow qg}|^2} &\approx& 16\pi^2 \alpha_s^2 \frac{s^2+u^2}{t^2}
\end{eqnarray}
For the heavy quark, since we are interested in its diffusion dynamics at low $p_T$, we uses the exact leading order matrix-element in the vacuum.
\begin{eqnarray}
\overline{|M_{Qq\rightarrow Qq}|^2} &=& \frac{64\pi^2\alpha_s^2}{9} \frac{(M^2-u)^2 + (s-M^2)^2 + 2 M^2 t}{t^2}
\nonumber
\\
\overline{|M_{Qq\rightarrow Qq}|^2} &=& \pi^2 \left\{
32\alpha_s^2 \frac{(s-M^2)(M^2-u)}{t^2} \right.
\nonumber
\\
&+&\frac{64}{9}\alpha_s^2 \frac{(s-M^2)(M^2-u)+2M^2(s+M^2)}{(s-M^2)^2} \nonumber
\\
&+&\frac{64}{9}\alpha_s^2 \frac{(s-M^2)(M^2-u)+2M^2(u+M^2)}{(M^2-u)^2} \nonumber
\\
&+& \frac{16}{9}\alpha_s^2 \frac{M^2(4M^2 - t)}{(M^2-u)(s-M^2)} 
\nonumber
\\
&+& 16 \alpha_s^2 \frac{(s-M^2)(M^2-u)+M^2(s-u)}{t(s-M^2)}
\nonumber
\\
&-& \left. 16 \alpha_s^2 \frac{(s-M^2)(M^2-u)-M^2(s-u)}{t(M^2-u)}\right\}
\end{eqnarray}

\begin{figure}
\singlespacing
\includegraphics[width=\textwidth]{feynman.pdf}
\caption[Elastic processes: The first diagram corresponds to heavy quark]{Elastic processes: The first diagram corresponds to heavy quark ($Q$) - light quark ($q$, $\bar{q}$) scattering. The last three diagrams contribute to heavy quark ($Q$) - gluon ($g$) scattering.}\label{plots:feyn-elastic}
\end{figure}

\subsection{$2\rightarrow 3$ matrix-elements}
Large-Q $2\rightarrow 3$ inelastic processes are $g + i \rightarrow q+\bar{q} + i$, $q+i\rightarrow q+g+i$ and $g+i\rightarrow g+g+i$, where $i$ stands for a medium parton, and the other symbols stands for hard partons.
In the medium frame, the hard parton has an energy $E\gg T$, while the medium thermal parton has $E\sim T$, and the typical center-of-mass energy is therefore $\sqrt{6ET}$.
We perform the calculation in the the center-of-mass frame of the two incoming partons and let the hard parton move towards the $+z$ direction with momentum $p_1$, and the medium parton moving to the $-z$ direction with $p_2$.
The hard parton then splits into two daughter partons with momenta $k$ and $p_1 + q - k$.
The momentum transfer $q$ between the hard parton and the medium parton is thought to be large enough $|q| > Q_{\textrm{cut}}$ so we neglect the thermal correction to its propagator.

Our derivation largely follows the work of \cite{Fochler:2013epa} while relaxing the soft approximation $xq_\perp \ll k_\perp$ in \cite{Fochler:2013epa}, and we only use the collinear approximation $k_\perp^2, q_\perp^2 \ll x(1-x) \hat{s}$ with $x = k^+/\sqrt{s} = k_\perp e^y_k /\sqrt{s}$.
Also, we only include the contributions with a $\hat{t}$-channel momentum exchange between the medium and the hard partons.
The collinear approximation requires $y_k \gg \ln(k_\perp/\sqrt{s})$ so that $y_k$ cannot be arbitrarily small and $y_k>0>\gg -\ln(\sqrt{s}/k_\perp)$ is a reasonable range of application.
Because $\hat{s}\sim 6 ET$, we expect this approximation to break down when either the typical values of $q_\perp^2$ becomes comparable to $x(1-x)6ET$ or when $y_k<0$ ($x < k_\perp/\sqrt{s} \sim k_\perp/\sqrt{6ET}$).
We shall briefly mention the treatment of the $y_k<0$ region in the end.

The light-cone momentum for $p_1$ , $p_2$ and $k$ can written down directly using $\sqrt{s}$, $x$ and $k_\perp$, then applying the above collinear condition, the expression for $q$ (and therefore $p_3$ and $p_4$) is obtained by kinematic constraint up to corrections of order $\{k_\perp, q_\perp^2\}/x(1-x)\hat{s}$.
\begin{eqnarray}
p_1 &=& (\sqrt{s}, 0, \vec{0})\\
p_2 &=& (0, \sqrt{s}, \vec{0})\\
k &=& (x\sqrt{s}, \frac{k_\perp^2}{x\sqrt{s}}, \vec{k}_\perp)\\
q &\sim& (-\frac{q_\perp^2}{\sqrt{s}}, \frac{q_\perp^2 + k_\perp^2/x - 
2\vec{q}_\perp \cdot \vec{k}_\perp}{(1-x)\sqrt{s}}, \vec{k}_\perp)
\end{eqnarray}
Using the light-cone gauge with a light-like vector $n = (0, 1, 0)$, the gauge fixing condition $n\cdot A =0$ eliminates the ``+" component in the gluon (with momentum $p$) polarization vector, and is obtained by applying the transverse condition $\epsilon \cdot p = 0$ (up to a higher order correction to its normalization)
\begin{eqnarray}
\epsilon(p) &\sim& (0, \frac{2\vec{\epsilon}_\perp\cdot\vec{p}_\perp}{p^+}, \vec{\epsilon}_\perp).
\end{eqnarray}
With these preparations, the matrix-element is factorized into an amplitude for the splitting process (approximated in the collinear limit) times the amplitude for two-body collision with the medium parton.
We shall only derive explicitly the cases where the medium parton is a quark, for colliding with medium anti-quark and gluon, it is sufficient to replace the $H+q\xrightarrow{\hat{t}} H+q$ amplitude by $H+\bar{q}\xrightarrow{\hat{t}} H+\bar{q}$ and $H+g\xrightarrow{\hat{t}} H+g$.
The connection of these results to the Bethe-Heitler limit of the AMY integral equation will be elucidated in the end.

\paragraph*{Gluon splitting to quark-anti-quark pair}
\begin{figure}
\singlespacing
\centering
\includegraphics[width=.5\textwidth]{Large-Q-g2qqbar-A.pdf}\\
\vspace{1em}
\includegraphics[width=.49\textwidth]{Large-Q-g2qqbar-B.pdf}\hfill
\includegraphics[width=.49\textwidth]{Large-Q-g2qqbar-C.pdf}
\caption[Three diagrams $A$ (Top), $B$ (Bottom left), $C$ (Bottom right) that]{Three diagrams $A$ (Top), $B$ (Bottom left), $C$ (Bottom right) that contribute to the large angle scattering induced gluon splitting into quark-anti-quark pair in the forward region of the center-of-mass frame.}
\label{fig:feyn-g2qqbar}
\end{figure}

Three Feynman diagrams contribute to the kinematic region $y_k >0$ in the current approximation, as shown in figure \ref{fig:feyn-g2qqbar}.
We start from the amplitude for diagram $A$.
\begin{eqnarray}
i M_A &=& (-ig)^2(-g)f^{abc}(t^b)_{j'j}(t^c)_{i'i} \epsilon_\lambda^\mu(p_1) \\\nonumber
&&\frac{-i}{(p_1+q)^2}\left(g^{\rho\rho'}-\frac{n^{\rho}(p_1+q)^{\rho'}+n^{\rho'}(p_1+q)^\rho}{n\cdot (p_1+q)}\right) \bar{u}^s(p_1+q-k)\gamma_{\rho'}v^{s'}(k) \\ \nonumber
&&\frac{-i}{q^2}\left(g^{\nu\nu'}-\frac{n^{\nu}q^{\nu'}+n^{\nu'}q^\nu}{n\cdot q}\right) \bar{u}^{\sigma}(p_4)\gamma_{\nu'}u^{\sigma'}(p_2) \\ \nonumber
&& \left[g_{\mu\nu}(p_1-q)_\rho + g_{\nu\rho}(2q+p_1)_\rho + g_{\rho\mu}(-2p_1 -q)_\nu \right]
\end{eqnarray}
Next, express the projection matrix of the gluon propagator with momentum $p_1+q$ by the sum of tensor products of its polarization vectors, and identify the amplitude $iP_{A,\lambda'}^{ss'}$ for a gluon with polarization $\lambda'$ to split into the quark and anti-quark pair with spin $s$ and $s'$.
Also, use the high energy approximation to replace $\bar{u}^i(a)\gamma^\alpha u^j(b)$ by $(a+b)^\alpha \delta^{ij}$, then
\begin{eqnarray}
i M_A &\approx& -g^3 f^{abc}(t^b)_{j'j}(t^c)_{i'i} \delta^{\sigma\sigma'} \epsilon^\mu(p_1) \\\nonumber
&&\frac{1}{(p_1+q)^2} \sum_{\lambda'=\pm}\epsilon_{\lambda'}^{\rho}(p_1+q)\underbrace{\epsilon_{\lambda'}^{*,\rho'}(p_1+q) \bar{u}^s(p_1+q-k)\gamma_{\rho'}v^{s'}(k)}_{iP_{A,\lambda'}^{ss'}} \\ \nonumber
&&\frac{1}{q_\perp^2}\left(g^{\nu\nu'}-\frac{n^{\nu}q^{\nu'}+n^{\nu'}q^\nu}{n\cdot q}\right) (2p_2-q)_{\nu'} \\ \nonumber
&& \left[g_{\mu\nu}(p_1-q)_\rho + g_{\nu\rho}(2q+p_1)_\rho + g_{\rho\mu}(-2p_1 -q)_\nu \right] \\
&=& -g^3 f^{abc}(t^b)_{j'j}(t^c)_{i'i} \frac{1}{(p_1+q)^2}\frac{1}{q_\perp^2} \sum_{\lambda'=\pm}iP_{A,\lambda}^{ss'} \delta^{\sigma\sigma'}  \\ \nonumber
&& \epsilon_\lambda^\mu(p_1)2p_2^{\nu} \epsilon_{\lambda'}^{\rho}(p_1+q) \left[g_{\mu\nu}(p_1-q)_\rho + g_{\nu\rho}(2q+p_1)_\rho + g_{\rho\mu}(-2p_1 -q)_\nu \right].
\end{eqnarray}
Finally, we evaluate the contraction in the second line using the expression for $p_1, q$ and $\epsilon$, and keep only terms that is leading in $q_\perp^2/s$ to get,
\begin{eqnarray}
i M_A \approx -g^3 f^{abc}(t^b)_{j'j}(t^c)_{i'i}\delta^{\sigma\sigma'}\frac{2s}{q_\perp^2} \frac{x(1-x)}{(\vec{k}_\perp-x \vec{q}_\perp)^2} iP_{A,\lambda}^{ss'}.
\end{eqnarray}

Diagram B and C are similar and we only write down diagram B in detail.
\begin{eqnarray}
i M_B &=& (-ig)^3 (t^bt^a)_{i'i}(t^b)_{j'j} \epsilon_\lambda^\mu(p_1) \\\nonumber
&&\frac{-i}{q^2}\left(g^{\nu\nu'}-\frac{n^{\nu}q^{\nu'}+n^{\nu'}q^\nu}{n\cdot q}\right) \\\nonumber
&&\bar{u}^s(p_1+q-k)\gamma_{\nu}\frac{i(\slashed{p_1}-\slashed{k})}{(p_1-k)^2}\gamma^{\mu}v^{s'}(k) \\ \nonumber
&&\bar{u}^{\sigma}(p_4)\gamma_{\nu'}u^{\sigma'}(p_2)
\end{eqnarray}
Again, we represent the tensor structure of the fermion propagator by the sum of tensor products of the spinors, identify the splitting amplitude $iP_{B,\lambda'}^{ss'}$ and use the high energy limit of the current,
\begin{eqnarray}
i M_B &\approx& ig^3 (t^bt^a)_{i'i}(t^b)_{j'j}  \\\nonumber
&&\frac{-i}{q_\perp^2}\left(g^{\nu\nu'}-\frac{n^{\nu}q^{\nu'}+n^{\nu'}q^\nu}{n\cdot q}\right) (2p_2-q)_\nu' \\\nonumber
&&\frac{1}{2p_1\cdot k} \sum_\sigma \bar{u}^s(p_1+q-k)\gamma_{\nu} u^{\sigma}(p_1-k) \underbrace{\epsilon_\lambda^\mu(p_1)\bar{u}^{\sigma}(p_1-k) \gamma^{\mu}v^{s'}(k)}_{iP_{B,\lambda}^{\sigma s'}}\\
&\approx& ig^3 (t^bt^a)_{i'i}(t^b)_{j'j} \frac{-i}{q_\perp^2}\frac{1}{2p_1\cdot k} iP_{B,\lambda}^{ss'}\\\nonumber
&&\left(g^{\nu\nu'}-\frac{n^{\nu}q^{\nu'}+n^{\nu'}q^\nu}{n\cdot q}\right) (2p_2-q)_{\nu'} (2p_1-q+2k)_\nu 
\end{eqnarray}
Note that $iP_{B}$ is different from $iP_{A}$ as the initial splitting parton has a different transverse momentum from diagram $A$.
Finally, we evaluate the contraction and get,
\begin{eqnarray}
i M_B &=& i g^3 (t^b t^a)){i'i} t^b{j'j} \delta^{\sigma\sigma'} \frac{2s}{q_\perp^2} \frac{x(1-x)}{k_\perp^2}  iP_{B,\lambda}^{ss'}
\end{eqnarray}
Diagram C can be obtained similarly,
\begin{eqnarray}
i M_C &=& -i g^3 (t^a t^b)){i'i} t^b{j'j} \delta^{\sigma\sigma'} \frac{2s}{q_\perp^2} \frac{x(1-x)}{(\vec{k}_\perp-\vec{q}_\perp)^2}  iP_{C,\lambda}^{ss'} 
\end{eqnarray}
To sum the contributions from all three diagrams, applying $f^{abc}t^c = -i[t^a, t^b]$ to $iM_A$ and the result is,
\begin{eqnarray}
i (M_A+M_B+M_C) &=& ig^3 \frac{2s}{q_\perp^2} (t^b)_{j'j} x(1-x)\\\nonumber
&&\left\{(t^a t^b)_{i'i} \left(\frac{iP_{A,\lambda}^{ss'} }{(\vec{k}_\perp-x \vec{q}_\perp)^2} - \frac{iP_{C,\lambda}^{ss'}}{(\vec{k}_\perp-\vec{q}_\perp)^2}\right) \right. \\\nonumber
&&\left.-(t^a t^b)_{i'i}\left(\frac{iP_{A,\lambda}^{ss'} }{(\vec{k}_\perp-x \vec{q}_\perp)^2} - \frac{iP_{B,\lambda}^{ss'}}{k_\perp^2}\right) \right\}
\end{eqnarray}

Now we have to address what those splitting amplitudes are.
Label the four momenta as $p_g = c$, $p_q = a$, $p_{\bar{q}} = b$.
And use the following representation for the spinors,
\begin{eqnarray}
u^s(p) = (\sqrt{p\cdot \sigma} \xi^s, \sqrt{p\cdot \bar{\sigma}} \xi^s)^T
v^s(p) = (\sqrt{p\cdot \sigma} \eta^s, -\sqrt{p\cdot \bar{\sigma}} \eta^s)^T
\end{eqnarray}
where $\sigma_{i=\{1,2,3\}}$ are Pauli matrices, $\sigma = (1_{2\times 2}, \vec{\sigma})$, and $\bar{\sigma} = (1_{2\times 2}, -\vec{\sigma})$.
The square root of the matrix is,
\begin{eqnarray}
\sqrt{p\cdot \sigma} =
\left.
\begin{bmatrix}
p^- & -p_L^\perp \\
-p_R^\perp & p^+
\end{bmatrix}\right.^{1/2} 
= \frac{1}{\sqrt{2(E\pm M)}}(p\cdot\sigma \pm \mathbf{1}M)\\
\sqrt{p\cdot \bar{\sigma}} =
\left.
\begin{bmatrix}
p^+ & p_\perp^- \\
p_R^\perp & p^-
\end{bmatrix}\right.^{1/2} 
= \frac{1}{\sqrt{2(E\pm M)}}(p\cdot\bar{\sigma} \pm \mathbf{1}M)\\
\end{eqnarray}
where $M$ is the mass of the particle, $p^\pm = E\pm p_z$, and $p_{R,L}^\perp = p_x \pm  i p_y$.
Currently, we only consider the massless case, because the mass effect is not implemented in the few-body matrix-elements in our model.
Neglecting the mass, the splitting amplitude is,
\begin{eqnarray}
&&\epsilon_{\lambda, \mu}(c) \bar{u}_s(a)\gamma^\mu v_{s'}(b)\\
&=&\frac{1}{\sqrt{2a}\sqrt{2b}}(\xi^T_s a\cdot\sigma, \xi^T_{s} a\cdot \bar{\sigma})
\begin{bmatrix}
\epsilon\cdot\bar{\sigma} & 0 \\
0 & \epsilon\cdot\sigma
\end{bmatrix}
\begin{bmatrix}
b\cdot\sigma \eta_{s'}\\
b\cdot\bar{\sigma} \eta_{s'}
\end{bmatrix}
\\
&=&\frac{1}{2\sqrt{ab}}
\xi_s^T
\begin{bmatrix}
a^- & -a^\perp_L \\
-a^\perp_R & a^+
\end{bmatrix}
\begin{bmatrix}
0 & \sqrt{2}\delta_{\lambda R}\\
\sqrt{2}\delta_{\lambda L} & \frac{\sqrt{2}c^\perp_\lambda}{c^+}
\end{bmatrix}
\begin{bmatrix}
b^- & -b^\perp_L \\
-b^\perp_R & b^-
\end{bmatrix}
\eta_{s'}\\\nonumber
&-&
\frac{1}{2\sqrt{ab}}
\xi_s^T
\begin{bmatrix}
a^+ & a^\perp_L \\
a^\perp_R & a^-
\end{bmatrix}
\begin{bmatrix}
\frac{\sqrt{2}c^\perp_\lambda}{c^+} & -\sqrt{2}\delta_{\lambda R}\\
-\sqrt{2}\delta_{\lambda L} & 0
\end{bmatrix}
\begin{bmatrix}
b^+ & b^\perp_L \\
b^\perp_R & b^-
\end{bmatrix}
\eta_{s'}
\\
&=&\frac{1}{\sqrt{2ab}}
\xi_s^T
\begin{bmatrix}
-a^\perp_L b^- \delta_{\lambda L} - a^- b^\perp_L \delta_{\lambda R} + a^\perp_L b^\perp_R\frac{c^\perp_\lambda}{c^+} &
a^\perp_L b^\perp_L \delta_{\lambda L} + a^- b^+ \delta_{\lambda R} - a^\perp_L b^+\frac{c^\perp_\lambda}{c^+}
\\
a^+ b^- \delta_{\lambda L} + a^\perp_R b^\perp_R \delta_{\lambda R} - a^+ b^\perp_R\frac{c^\perp_\lambda}{c^+} &
-a^+ b^\perp_L \delta_{\lambda L} - a^\perp_R b^+ \delta_{\lambda R} + a^+ b^+\frac{c^\perp_\lambda}{c^+}
\end{bmatrix}
\eta_{s'}\\\nonumber
&-&\frac{1}{\sqrt{2ab}}
\xi_s^T
\begin{bmatrix}
-a^\perp_L b^+ \delta_{\lambda L} - a^+ b^\perp_R \delta_{\lambda R} + a^+ b^+\frac{c^\perp_\lambda}{c^+} &
-a^\perp_L b^\perp_L \delta_{\lambda L} - a^+ b^- \delta_{\lambda R} + a^+ b^\perp_L\frac{c^\perp_\lambda}{c^+}
\\
-a^- b^+ \delta_{\lambda L} - a^\perp_R b^\perp_R \delta_{\lambda R} + a^\perp_+ b^+\frac{c^\perp_\lambda}{c^+} &
-a^- b^\perp_L \delta_{\lambda L} - a^\perp_R b^- \delta_{\lambda R} + a^\perp_R b^\perp_L\frac{c^\perp_\lambda}{c^+}
\end{bmatrix}
\eta_{s'}
\end{eqnarray}
Keep the leading terms in the collinear limit which are products of $(+)(+)$ or $(+)(\perp)$ components of the momenta, and drop terms that are of order $(+)(-)$, $(\perp)(\perp)$ and $(\perp)(-)$,
\begin{eqnarray}
&&\epsilon_{\lambda, \mu} \bar{u}_s(a)\gamma^\mu v_{s'}(b)\\
&=& \frac{1}{\sqrt{2ab}}
\xi_s^T
\begin{bmatrix}
a^\perp_L b^+ \delta_{\lambda L} + a^+ b^\perp_R \delta_{\lambda R} - a^+ b^+\frac{c^\perp_\lambda}{c^+} & 0\\
0 & -a^+ b^\perp_L \delta_{\lambda L} - a^\perp_R b^+ \delta_{\lambda R} + a^+ b^+\frac{c^\perp_\lambda}{c^+}
\end{bmatrix}
\eta_{s'}
\end{eqnarray}
There are four combinations for the possible initial state polarization and final state spins
\begin{eqnarray}
\epsilon_{\lambda, \mu} \bar{u}_s(a)\gamma^\mu v_{s'}(b) = \frac{x\vec{a} - (1-x)\vec{b}}{\sqrt{2x(1-x)}}
\begin{cases}
x, \hfill \lambda=L, s=\uparrow\\
-(1-x), \hfill \lambda=L, s=\downarrow\\
(1-x), \hfill \lambda=R, s=\uparrow\\
-x, \hfill \lambda=R, s=\downarrow\\
\end{cases}
\end{eqnarray}
Where we have used $a^+ = (1-x)c^+, b^+ = xc^+$ and $c_\perp = a_\perp+b_\perp$.
Sum over the spins and average over polarization for the squared amplitude,
\begin{eqnarray}
\frac{1}{2}\sum_\pm |P|^2 = \frac{2(x^2 + (1-x)^2)}{x(1-x)} \left((1-x)\vec{a}_\perp-x\vec{b}_\perp\right)^2.
\end{eqnarray}
This result goes back to the standard splitting function if it is computed in the frame where $a_\perp = -b_\perp$. 
However, there is no such frame that $a_\perp = -b_\perp$ satisfies simultaneously for the splitting in diagram A, B and C.
Therefore different amplitude needs to be inserted for each diagram and we find,
\begin{eqnarray}
\label{eq:gq2qqbarq}
\frac{\sum_{\lambda, s, s', \sigma, \sigma', a, b}|M^2|_{g+q\rightarrow q+\bar{q}+q}}{2d_F 2d_A} &=& g^4 \frac{2C_F}{d_A}\frac{4s^2 x(1-x)}{q_\perp^4}  \\\nonumber
&\times& g^2\frac{(x^2+(1-x)^2)}{2} \left(C_F \vec{A}^2 + C_F \vec{B}^2 - (2C_F- C_A)\vec{A}\cdot\vec{B}\right)
\end{eqnarray}
Where the $\vec{A}$ and $\vec{B}$ are,
\begin{eqnarray}
\vec{A} &=& \frac{\vec{k}_\perp - x\vec{q}_\perp}{(\vec{k}_\perp - x\vec{q}_\perp)^2} -  \frac{\vec{k}_\perp - \vec{q}_\perp}{(\vec{k}_\perp - \vec{q}_\perp)^2}, \\
\vec{B} &=& \frac{\vec{k}_\perp - x\vec{q}_\perp}{(\vec{k}_\perp - x\vec{q}_\perp)^2} -  \frac{\vec{k}_\perp}{\vec{k}_\perp^2}.
\end{eqnarray}
The final squared matrix-element has been factorized into the two body scattering part (first line) and the collinear splitting part (second line) with the desired leading order QCD splitting function. 

\paragraph*{Quark splits to quark and gluon}
\begin{figure}
\singlespacing
\centering
\includegraphics[width=.5\textwidth]{Large-Q-q2qg-A.pdf}\\
\vspace{1em}
\includegraphics[width=.49\textwidth]{Large-Q-q2qg-B.pdf}\hfill
\includegraphics[width=.49\textwidth]{Large-Q-q2qg-C.pdf}
\caption[Three diagrams $A$ (Top), $B$ (Bottom left), $C$ (Bottom right) that]{Three diagrams $A$ (Top), $B$ (Bottom left), $C$ (Bottom right) that contribute to the large angle scattering induced a quark splitting into a quark and a gluon in the forward region of the center-of-mass frame.}
\label{fig:feyn-q2qg}
\end{figure}

The Feynman diagrams to be included for $q+q\rightarrow q+g+q$ are shown in Figure \ref{fig:feyn-q2qg}.
The calculation uses exactly the same technique we used for the gluon splitting channel, and we present the result directly,
\begin{eqnarray}
\label{eq:qq2qgq}
\overline{|M^2|}_{g+q\rightarrow g+g+q} &=& 
 g^4 \frac{C_F}{d_F}\frac{4s^2}{q_\perp^4}x(1-x) \\\nonumber
&\times&g^2\frac{1+(1-x)^2}{x}  
\left(C_F\vec{A}^2 + C_F\vec{B}^2 - \left(2C_F-C_A\right)\vec{A}\cdot\vec{B}\right)\\
\vec{A} &=& \frac{\vec{k}_\perp - \vec{q}_\perp}{(\vec{k}_\perp - \vec{q}_\perp)^2} -  \frac{\vec{k}_\perp - x\vec{q}_\perp}{(\vec{k}_\perp - x\vec{q}_\perp)^2} \\
\vec{B} &=& \frac{\vec{k}_\perp - \vec{q}_\perp}{(\vec{k}_\perp - \vec{q}_\perp)^2} -  \frac{\vec{k}_\perp}{\vec{k}_\perp^2}
\end{eqnarray}


\paragraph*{Gluon splitting to two gluons}
\begin{figure}
\singlespacing
\centering
\includegraphics[width=.5\textwidth]{Large-Q-g2gg-A.pdf}\\
\vspace{1em}
\includegraphics[width=.49\textwidth]{Large-Q-g2gg-B.pdf}\hfill
\includegraphics[width=.49\textwidth]{Large-Q-g2gg-C.pdf}
\caption[Three diagrams $A$ (Top), $B$ (Bottom left), $C$ (Bottom right) that]{Three diagrams $A$ (Top), $B$ (Bottom left), $C$ (Bottom right) that contribute to the large angle scattering induced gluon splitting into two gluons in the forward region of the center-of-mass frame.}
\label{fig:feyn-g2gg}
\end{figure}

Finally, for $g+q\rightarrow g+q+g$, the Feynman diagrams are shown in Figure \ref{fig:feyn-g2gg}. 
The simplification of the two body collision amplitude can be done in a similar manner as the previous two channels. 
We only write down the splitting amplitude $g\rightarrow g+ g$ in detail.
Suppressing the color index, we label the initial gluon with $\epsilon_1^\mu(p)$, and the two daughter gluons with $\epsilon_2^\nu(k)$ and $\epsilon_3^\rho(q)$.
The splitting amplitudes are then (omitting the factor $-gf^{abc}$)
\begin{eqnarray}
iP &=& \epsilon^\mu_1\epsilon^\nu_2\epsilon^\rho_3
\left[
g_{\mu\nu} (p+k)_{\rho} +  g_{\nu\rho} (p+k)_{\mu} + g_{\rho\mu} (-q-p)_{\nu}
\right]\\
&=& -\vec{\epsilon}_{1,\perp}\cdot \vec{\epsilon}_{2,\perp} \left[(p+k)^+\frac{\vec{\epsilon}_{3,\perp}\cdot \vec{q}_\perp}{q^+} - \vec{\epsilon}_{3,\perp}\cdot (\vec{p}_\perp+\vec{k}_\perp)\right] \\\nonumber
&&-\vec{\epsilon}_{2,\perp}\cdot \vec{\epsilon}_{3,\perp} \left[(-k+q)^+\frac{\vec{\epsilon}_{1,\perp}\cdot \vec{p}_\perp}{p^+} - \vec{\epsilon}_{1,\perp}\cdot (-\vec{k}_\perp+\vec{q}_\perp)\right]
\\\nonumber
&&-\vec{\epsilon}_{3,\perp}\cdot \vec{\epsilon}_{1,\perp} \left[(-q-p)^+\frac{\vec{\epsilon}_{2,\perp}\cdot \vec{k}_\perp}{k^+} - \vec{\epsilon}_{2,\perp}\cdot (-\vec{q}_\perp-\vec{p}_\perp)\right]
\end{eqnarray}
There are four possible combinations of the polarization vectors, and their respective amplitude is computed as,
\begin{eqnarray}
iP = \sqrt{2}\left[x\vec{q}_\perp - (1-x)\vec{k}_\perp\right]\times 
\begin{cases}
\frac{1-x+x^2}{x(1-x)}, \hfill \lambda_1=\lambda_2=\lambda_3\\
-1, \hfill \lambda_1\neq\lambda_2=\lambda_3 \\
\frac{1}{x}, \hfill \lambda_1=\lambda_3\neq\lambda_2\\
\frac{1}{1-x}, \hfill \lambda_1=\lambda_2\neq\lambda_3
\end{cases}
\end{eqnarray}
Summing over the squared amplitude of all four cases and averaging over the initial gluon polarization, one gets the desired leading order QCD splitting function,
\begin{eqnarray}
2\frac{1+x^2+(1-x)^4}{x^2(1-x)^2} \left[x\vec{q}_\perp - (1-x)\vec{k}_\perp\right]^2.
\end{eqnarray}

Substituting the amplitude in each diagram, the final squared matrix-element is 
\begin{eqnarray}
\label{eq:gq2ggq}
\overline{|M^2|}_{g+q\rightarrow g+g+q} &=&
g^4 \frac{C_A}{d_F}\frac{4s^2x(1-x)}{q_\perp^4} \\\nonumber
&\times&g^2\frac{1+x^4+(1-x)^4}{x(1-x)}   
\left(C_A\vec{A}^2 + C_A\vec{B}^2 - C_A\vec{A}\cdot\vec{B}\right)\\
\vec{A} &=& \frac{\vec{k}_\perp - x\vec{q}_\perp}{(\vec{k}_\perp - x\vec{q}_\perp)^2} -  \frac{\vec{k}_\perp - \vec{q}_\perp}{(\vec{k}_\perp - \vec{q}_\perp)^2} \\
\vec{B} &=& \frac{\vec{k}_\perp - x\vec{q}_\perp}{(\vec{k}_\perp - x\vec{q}_\perp)^2} -  \frac{\vec{k}_\perp}{\vec{k}_\perp^2}
\end{eqnarray}

\paragraph{Regulating the $2\rightarrow 3$ squared matrix-elements}
The divergence in the $q$ integration is removed by the requirement that this few-body matrix-element only applies to processes with $q>Q_{\textrm{cut}}$.
The collinear divergence when $k$ approaches $q$ or $xq$ is regulated by including a gluon thermal mass. 
In practice, the collinear region will be further suppressed by the LPM effect.
The cross-section is obtained by integrating over the final state phase-space, where we have chosen to parameterize the three particle final state in terms of $k_\perp^2$, the rapidity of $k$ in the center-of-mass frame $y_k$, and the solid angle of the recoil medium particle.

\paragraph{Soft limit: the Gunion-Bertsch approximation}
The result we obtained for the $g\rightarrow g+g$ and $q\rightarrow q+g$ channel have a soft limit that goes back to the well known Gunion-Bertsch form. 
In the soft limit, we require the radiated gluon energy to be small enough such that $xq_\perp \ll k_\perp$.
Then, the splitting amplitudes for both $g\rightarrow g+g$ and  $q\rightarrow q+g$ are simplified into the same form,
\begin{eqnarray}
\overline{|M|}^2_{22} x(1-x)g^2 \frac{2(1-x+O(x^2))}{x} C_A \left(\frac{\vec{k}_\perp}{k_\perp^2}-\frac{\vec{k}_\perp-\vec{q}_\perp}{(\vec{k}_\perp-\vec{q}_\perp)^2}\right)^2
\end{eqnarray}
Neglecting the $O(x^2)$ terms in the splitting function, the result is the same as the improved verison of the Gunion-Bertsch cross-section \cite{Fochler:2013epa} used in the full Boltzmann partonic transport model BAMPS \cite{Xu:2004mz},
\begin{eqnarray}
\overline{|M|}^2_{22} 8\pi C_A\alpha_s (1-x)^2 \left(\frac{\vec{k}_\perp}{k_\perp^2}-\frac{\vec{k}_\perp-\vec{q}_\perp}{(\vec{k}_\perp-\vec{q}_\perp)^2}\right)^2
\end{eqnarray}

\paragraph{The backward ($y_k < 0$) region}
We have mentioned in the beginning of the derivation that the condition $k_\perp^2 < x(1-x)\hat{s}$ restricts the splitting to be happen only for the parton moving in the $+z$ direction in the center-of-mass frame ($y_k > 0$).
For splittings that happen in the backward region, another set of diagrams contribute, where the splitting comes from the parton that moves in the $-z$ direction in the center-of-mass frame.
Also one needs a different gauge $A^- = 0$.
The derivation is similar to the previous ones, but with the definition of $x$ and $q$ changed to $x = k^-/\sqrt{s}$, and $q = p_1-p_3$.

To combine the results that are obtained in different regions of phase space ($y_k > 0$ and $y_k < 0$), we follow \cite{Fochler:2013epa} and defines,
\begin{eqnarray}
\bar{x} &=& \frac{(k + |k_z|)}{\sqrt{s}} = \frac{k_\perp e^{|y_k|}}{\sqrt{s}}\\ 
\bar{q} &=& \Theta(y_k)(p_2-p_4) + \Theta(-y_k)(p_1-p_3)
\end{eqnarray}
which replaces the original $x$ and $q$ in our formula, and the resultant matrix-elements can be used for both forward and backward regions.



\paragraph*{Relation to the Bethe-Heitler limit of the AMY formalisim}
Now we show the connection between the $2\rightarrow 3$ cross section obtained here and the Bethe-Heitler limit of the AMY equation.
In the Bethe-Heitler limit, the AMY integral equation can be solved approximately by treating $1/\tau_f$ as the leading factor. 
One obtains the splitting rate for each different channels (denoting $\vec{a}/a^2$ as $\vec{\phi}_{a}$), 
\begin{eqnarray}
R_{q\rightarrow q+g}^{BH} &\propto& g^2 P_{qg}^{q(0)}(x) \int d k^2 d q^2 \mathcal{A}(q^2) \left\{
C_A\vec{\phi}_k\cdot\left(\vec{\phi}_k-\vec{\phi}_{k-q}\right) \right.\\\nonumber
&&+\left. (2C_F-C_A) \vec{\phi}_k\cdot\left(\vec{\phi}_k-\vec{\phi}_{k+xq}\right)
+ C_A \vec{\phi}_k\cdot\left(\vec{\phi}_k - \vec{\phi}_{k+(1-x)q}\right)
\right\}
\\
R_{g\rightarrow g+g}^{BH} &\propto& g^2 P_{gg}^{g(0)}(x) \int d k^2 d q^2 \mathcal{A}(q^2) \left\{
C_A\vec{\phi}_k\cdot\left(\vec{\phi}_k-\vec{\phi}_{k-q}\right) \right.\\\nonumber
&&+\left. C_A \vec{\phi}_k\cdot\left(\vec{\phi}_k-\vec{\phi}_{k+xq}\right)
+ C_A \vec{\phi}_k\cdot\left(\vec{\phi}_k - \vec{\phi}_{k+(1-x)q}\right)
\right\}
\\
R_{g\rightarrow q+\bar{q}}^{BH} &\propto& g^2 P_{q\bar{q}}^{g(0)}(x) \int d k^2  d q^2 \mathcal{A}(q^2) \left\{
(2C_F-C_A)\vec{\phi}_k\cdot\left(\vec{\phi}_k-\vec{\phi}_{k-q}\right) \right.\\\nonumber
&&+\left. C_A \vec{\phi}_k\cdot\left(\vec{\phi}_k-\vec{\phi}_{k+xq}\right)
+ C_A \vec{\phi}_k\cdot\left(\vec{\phi}_k - \vec{\phi}_{k+(1-x)q}\right)
\right\}
\end{eqnarray}
with the collision kernel $\mathcal{A} = g^2 T m_D^2/q^2(q^2+m_D^2)$. These expressions look drastically different from the incoherent rate computed using the cross-section derived in the previous section, however, we would like to show that they are the same once integration over $dk^2$ is performed.
Therefore, the incoherent rate we used in the Boltzmann equation indeed recovers the Bethe-Heitler limit of the AMY integral equation.

To show this, we start from the $2\rightarrow 3$ rate formula using the matrix-elements from equations \ref{eq:gq2qqbarq}, \ref{eq:qq2qgq} and \ref{eq:gq2ggq}. 
For the $q\rightarrow q+g$ channel, the rate in our Boltzmann equation is,
\begin{eqnarray}
R_{q\rightarrow q+g} &\propto& g^2 P_{qg}^{q(0)}(x) \int  \frac{f(p_2)dp_2^3}{2E_2(2\pi)^3} d q^2 \frac{g^4}{q^4}\\\nonumber
&&  \int d k^2\left\{
C_F\left( \vec{\phi}_{k-q}-\vec{\phi}_{k-xq} \right)^2
+ C_F\left( \vec{\phi}_{k-q}-\vec{\phi}_{k} \right)^2\right.\\\nonumber
&&\left.
- (2C_F-C_A)\left( \vec{\phi}_{k-q}-\vec{\phi}_{k-xq} \right)\cdot \left( \vec{\phi}_{k-q}-\vec{\phi}_{k} \right)
\right\}
\end{eqnarray}
Focusing on the three products (squares) of $\vec{\phi}$s under the $dk^2$ integration, we are going to expand the first term in each product and then shift the argument of the first $\vec{\phi}$ to $k$, 
\begin{eqnarray}
R_{q\rightarrow q+g} &\propto& g^2 P_{qg}^{q(0)}(x) \int  \frac{f(p_2)dp_2^3}{2E_2(2\pi)^3} d q^2 \frac{g^4}{q^4}\\\nonumber
&&  \int d k^2\left\{
C_F\vec{\phi}_{k}\left( \vec{\phi}_{k}-\vec{\phi}_{k+(1-x)q} \right)
- C_F\vec{\phi}_{k}\left( \vec{\phi}_{k-(1-x)q}-\vec{\phi}_{k} \right)\right.
\\\nonumber
&&+ C_F\vec{\phi}_{k}\left( \vec{\phi}_{k}-\vec{\phi}_{k+q} \right)
- C_F\vec{\phi}_{k}\left( \vec{\phi}_{k-q}-\vec{\phi}_{k} \right)
\\\nonumber
&&\left.
- (2C_F-C_A)\vec{\phi}_{k}\cdot \left( \vec{\phi}_{k}-\vec{\phi}_{k+q} \right)
+(2C_F-C_A)\vec{\phi}_{k} \cdot \left( \vec{\phi}_{k-(1-x)q}-\vec{\phi}_{k+xq} \right)
\right\}
\end{eqnarray}
Next, flip  the sign of $q$ under the integration,
and meanwhile, insert a $-\vec{\phi}_k +\vec{\phi}_k$ in the brackets of the last term,
\begin{eqnarray}
R_{q\rightarrow q+g} &\propto& g^2 P_{qg}^{q(0)}(x) \int  \frac{f(p_2)dp_2^3}{2E_2(2\pi)^3} d q^2 \frac{g^4}{q^4}\\\nonumber
&&  \int d k^2\left\{
2C_F\vec{\phi}_{k}\left( \vec{\phi}_{k}-\vec{\phi}_{k+(1-x)q} \right)
+ 2C_F\vec{\phi}_{k}\left( \vec{\phi}_{k}-\vec{\phi}_{k+q} \right)
\right.
\\\nonumber
&&
- (2C_F-C_A)\vec{\phi}_{k}\cdot \left( \vec{\phi}_{k}-\vec{\phi}_{k+q} \right)
+(2C_F-C_A)\vec{\phi}_{k} \cdot \left( \vec{\phi}_{k+(1-x)q} -\vec{\phi}_k \right) \\\nonumber
&&\left.+(2C_F-C_A)\vec{\phi}_{k} \cdot \left(\vec{\phi}_k-\vec{\phi}_{k+xq} \right)
\right\}
\end{eqnarray}
After this manipulation, the first (second) term cancels the $C_F$ part of the fourth (third) term, 
\begin{eqnarray}
R_{q\rightarrow q+g} &\propto& g^2 P_{qg}^{q(0)}(x) \int  \frac{f(p_2)dp_2^3}{2E_2(2\pi)^3} d q^2 \frac{g^4}{q^4}\\\nonumber
&&  \int d k^2\left\{
C_A\vec{\phi}_{k}\cdot \left( \vec{\phi}_{k}-\vec{\phi}_{k+q} \right)
+C_A\vec{\phi}_{k} \cdot \left( \vec{\phi}_k - \vec{\phi}_{k+(1-x)q}\right) \right.\\\nonumber
&&\left.+(2C_F-C_A)\vec{\phi}_{k} \cdot \left(\vec{\phi}_k-\vec{\phi}_{k+xq} \right)
\right\}
\end{eqnarray}
which is the same integration as the one obtained from the Bethe-Heitler limit of the AMY equation (neglecting the screen mass in $\mathcal{A}$ when $q^2 \gg m_D^2$)
The equivalence between these two expressions for the $g\rightarrow g+g$ channel and the $g\rightarrow q+\bar{q}$ channel can also be shown similarly.

%\paragraph*{Mass effect in the $2\rightarrow 3$ squared matrix-elements}
%For completeness, we briefly outline the derivation of $2\rightarrow 3$ cross-section with mass effect.
%As a remark, putting the heavy flavor mass directly into the these matrix-elements is certainly legitimate if one only focus on $2\rightarrow 3$ processes.
%But once we want to approximate the effect of multiple scatterings:
%$(n \rightarrow n+1) \approx (2 \rightarrow 3)(2 \rightarrow 2)\cdots(2 \rightarrow 2)\times \textrm{corrections}$, it is not advantageous to put the mass effect into the $(2 \rightarrow 3)$ part, but better to be included into the last step of corrections, which is the approach we used.


%First, we still work under the assumption that $M \ll E$, and will only keep terms when $M$ is making direct comparison to $k_\perp, q_\perp$.
%The kinematics are now changed to,
%\begin{eqnarray}
%p_1 &=& (\sqrt{s}, 0, \vec{0})\\
%p_2 &=& (0, \sqrt{s}, \vec{0})\\
%k &=& (x\sqrt{s}, \frac{k_\perp^2}{x\sqrt{s}}, \vec{k}_\perp)\\
%q &\sim& (-\frac{q_\perp^2\sqrt{s}}{s-M^2}, \frac{x(\vec{q}_\perp-\vec{k}_\perp)^2 + (1-x)k_\perp^2 + x^2M^2}{x(1-x)\sqrt{s}}, \vec{k}_\perp)
%\end{eqnarray}
%For the splitting amplitude, off diagonal elements of $\epsilon_{\lambda, \mu}(c) \bar{u}_s(a) \gamma^\mu v_{s'} (b)$ also need to be included for helicity flipping process. 
%Moreover,
%\begin{eqnarray}
%\sqrt{p\cdot \sigma} &=& \frac{p\cdot \sigma + M}{\sqrt{2(E+M)}} \approx \frac{p\cdot \sigma + M}{\sqrt{2E}} \\
%\sqrt{p\cdot \bar{\sigma}} &=& \frac{p\cdot \bar{\sigma} + M}{\sqrt{2(E+M)}} \approx \frac{p\cdot \bar{\sigma} + M}{\sqrt{2E}}
%\end{eqnarray}
%where we have omitted the mass in the denominator since it only involves corrections of order $M/E$.
%From this one can see that the previous calculation can be used for the massive case with the substitution  $a^\pm \rightarrow a^\pm +M$ and $b^\pm \rightarrow b^\pm +M$.
%Then, the splitting amplitude becomes,
%\begin{eqnarray}
%\epsilon_{\lambda, \mu} \bar{u}_s(a)\gamma^\mu v_{s'}(b)&=& \frac{1}{\sqrt{2ab}}
%\xi_s^T A_{ss'} \eta_{s'}\\
%A_{\uparrow\uparrow} &=&
%\delta_{\lambda L} 2b_z a^\perp_L + \delta_{\lambda R} 2a_z b^\perp_R + \frac{c^\perp_\lambda}{c^+} (a^\perp_L b^\perp_R - a^+ b^+) \\
%A_{\downarrow\downarrow} &=&
%-\delta_{\lambda L}2a_z b^\perp_L - \delta_{\lambda R}2b_z a^\perp_R - \frac{c^\perp_\lambda}{c^+} (a^\perp_R b^\perp_L - a^+ b^+) \\
%A_{\uparrow\downarrow} &=&
%\delta_{\lambda L} 2a^\perp_L b^\perp_L + \delta_{\lambda R} (a^+b^-+a^-b^+) - \frac{c^\perp_\lambda}{c^+} (a^+b^\perp + a^\perp b^+) \\
%A_{\downarrow\uparrow} &=&
% \delta_{\lambda L} (a^+b^-+a^-b^+) + \delta_{\lambda L} 2a^\perp_R b^\perp_R - \frac{c^\perp_\lambda}{c^+} (a^+b^\perp + a^\perp b^+) 
%\end{eqnarray}
\end{appendices}


\nocite{*}
\singlespacing
\bibliographystyle{unsrt}
\begingroup
    \setlength{\bibsep}{10pt}
    \bibliography{lit}
\endgroup


\biography
\doublespacing
Weiyao Ke reveived a B.S. in Physics from Peking University in 2014. His published or currently in preparation works during his PhD include:

\begin{itemize}
\singlespacing
\item Weiyao Ke, Yingru Xu, and Steffen A. Bass Modeling of quantum-coherence effects in parton radiative energy loss. arXiv:1810.08177.
\item Weiyao Ke, Yingru Xu, and Steffen A. Bass A linearized Boltzmann-Langevin transport model for heavy quark transport in hot and dense QCD matter. Phys. Rev. C {\bf 98}, 064901 (2018).
\item Weiyao Ke, J. Scott Moreland, Jonah E. Bernhard, and Steffen A. Bass Constraints on rapidity-dependent initial conditions from charged-particle pseudorapidity densities and two-particle correlations. Phys. Rev. C {\bf 96}, 044912 (2017).
\end{itemize}
\end{document}
